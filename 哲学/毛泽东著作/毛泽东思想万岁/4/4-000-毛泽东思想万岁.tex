\documentclass[b5paper,oneside,12pt]{ctexbook}
\usepackage[hmargin=0.3in,vmargin=0.5in]{geometry} 
\usepackage[]{multirow}
\usepackage[perpage,hang]{footmisc} %脚注

\pagestyle{plain} %整书页眉页脚设置
\setlength{\marginparsep}{2pt}
\setlength{\marginparwidth}{20pt}

\ctexset{chapter/numbering=false}
\ctexset{
    section={numbering=false, afterskip = 0ex},
    subsection={format=\large\heiti\centering,numbering=false,beforeskip=1ex,afterskip = 1.75ex}
}
\newcommand\datesubtitle[1]{{\centering\large #1\par\vspace{1ex}}}  %自定义日期副标题格式,为了保险,最好使用两层大括号

% 靠右对齐,右边距2字
\newcommand{\kaoyouerziju}[1]{{\raggedleft #1 \hspace{2em} \par}}
% 楷体,右边距5字
\newcommand{\kaitiqianming}[1]{{\raggedleft\large\kaishu\ziju{1} #1 \hspace{5em} \par}}

% 一人带职位
\newcommand{\yirendaizhiwei}[2]{
    {\setlength{\tabcolsep}{0em}
    {\raggedleft\begin{tabular} {cc}%
        #1 & \quad{} #2 \hspace{4em} \\ 
        \end{tabular} \\[1pt]}}
}

% 右下定宽
\newcommand{\youxiadingkuan}[1]{
    \begin{list}{}{
        \setlength{\topsep}{0pt}        % 列表与正文的垂直距离
        \setlength{\partopsep}{0pt}     % 
        \setlength{\parsep}{\parskip}   % 一个 item 内有多段,段落间距
        \setlength{\itemsep}{0pt}       % 两个 item 之间,减去 \parsep 的距离
        \setlength{\itemindent}{0pt}%
        \setlength\parindent{0pt}
        \setlength\listparindent{0pt}
        \setlength{\leftmargin}{0.4\linewidth}
        \setlength{\rightmargin}{2em}
    }
    \item[] #1
    \end{list}
}


\usepackage{etoolbox}

% 下面是修改了脚注样式
% 一些LATEX内部命令含有@字符,如\@addtoreset,如需使用这些内部命令,就需要借助于另两个命令\makeatletter和\makeatother.
\makeatletter
% 补丁,改脚注文本前面的序号的字体,去掉其上标样式 
\patchcmd{\@makefntext}
    {\@makefnmark}
    {\hbox{\normalfont\@thefnmark}}
    {}{}

% 给脚注编号前后添加 〔〕
\renewcommand\thefootnote{{〔\arabic{footnote}〕}} 

%% 开启 footmisc 的 hang 选项
\setlength{\footnotemargin}{1.25em}     % 整个脚注文本的左边距,加此边距,来显示脚注序号。
\setlength{\skip\footins}{1\baselineskip} % 脚注线 和 脚注内容 间距 
% \setlength{\footnotesep}{\skip\footins} % 两个脚注文本之间的间距
\renewcommand{\hangfootparskip}{0pt}
\renewcommand{\hangfootparindent}{2em}

% \patchcmd{\@makefntext}
% {\ifFN@hangfoot\bgroup}
% {\ifFN@hangfoot\bgroup\def\@makefnmark{\normalfont\@thefnmark}}
% {}{}

\makeatother

\usepackage{calc} % 可以在命令中计算长度
\usepackage[]{hyperref} % 放在 footmisc 后面

% 引用样式:使用 latex 原始的 list 环境
\newenvironment{yinyong}{%
    \begin{list}{}{\parsep\parskip
        \setlength\topsep{0pt}
        \setlength\itemindent{2em}%
        \setlength\parindent{2em}
        \setlength\listparindent{2em}
        \setlength{\leftmargin}{2em}
        \setlength{\rightmargin}{2em}
        \kaishu
    }
    \item[]
}{
  \end{list}
}

\title{毛泽东思想万岁\\1958.1—1960}
\author{毛泽东}
\date{}

\begin{document}

\frontmatter
\maketitle
\tableofcontents

\mainmatter

\input{4-001-1958.1.3-在杭州会议上的讲话(一)(一九五八年一月三日).tex}
\input{4-002-1958.1.4-在杭州会议上的讲话(二)(一九五八年一月四日).tex}
\input{4-003-1958.1.11-在南宁会议上的讲话(一)(一九五八年一月十一日).tex}
\input{4-004-1958.1.12-在南宁会议上的讲话(二)(一九五八年一月十二日).tex}
\input{4-005-1958.1.12-关于报纸工作给刘建勋、韦国清同志的一封信(一九五八年一月十二日).tex}
\section[给《文艺报》编委会的一封信关于1958年第二期《再批判》栏的按语(一九五八年一月十九日)]{给《文艺报》编委会的一封信关于1958年第二期《再批判》栏的按语}
\datesubtitle{(一九五八年一月十九日)}


即送北京《文艺报》×××、×××、×××三同志:

看了一点,没有看完,你们就发表吧。按语较沉闷。政治性不足,你们是文学家,文也不足。不足以唤起读者的注目。近来文风有了改进,就这篇按语说来,则尚未。题目太长,《再批判》三字就够多了。请你们斟酌一下。我在南方,你们来信刚才收到,明天就是付印日期,匆匆送上。

祝你们胜利!
<p align="right">毛泽东
一月十九日下午</p>


\input{4-007-1958.1.28-在最高国务会议上的讲话(一九五八年一月二十八日).tex}
\input{4-008-1958.1-在最高国务会议上讲话要点(一九五八年一月二十八、三十日).tex}
\input{4-009-1958.1.31-在中央政治局会议上讨论教育工作时的指示(一九五八年一月三十一日).tex}
\input{4-010-1958.2.8-反浪费反保守是当前整风运动的中心任务(一九五八年二月八日).tex}
\input{4-011-1958.1.31-工作方法六十条(草案)(一九五八年一月三十一日).tex}
\input{4-012-1958.3.9-在成都会议上的讲话(一)(一九五八年三月九日).tex}
\input{4-013-1958.3.1-在成都会议上的讲话(二)(一九五八年三月十日).tex}
\input{4-014-1958.3.2-在成都会议上的讲话(三)(一九五八年三月二十日).tex}
\input{4-015-1958.3.22-在成都会议上的讲话(四)(一九五八年三月二十二日).tex}
\input{4-016-1959.3.25-在成都会议上的讲话(五)(一九五九年三月二十五日).tex}
\input{4-017-1958.3.26-在成都会议上的讲话(六)(一九五八年三月二十六日).tex}
\input{4-018-1958.3-在成都会议上的插话(一九五八年三月).tex}
\input{4-019-1958.3.19-对《中国农村社会主义高潮》一书《按语》的批示(一九五八年三月十九日).tex}
\input{4-020-1958.3.22-对《上海化工学院两个右派分子的大字报》的批语(一九五八年三月二十二日).tex}
\input{4-021-1958.3.28-和《江峡》轮船员的谈话(一九五八年三月二十八日).tex}
\input{4-022-1958.3-视察四川省一个养猪场时的谈话(一九五八年三月).tex}
\input{4-023-1958.4.6-在汉口会议上的讲话(一九五八年四月六日).tex}
\input{4-024-1958.4.6-在汉口会议上的讲话(二).tex}
\input{4-025-1958.4.1-在汉口会议上的插话(一九五八年四月一日至六日).tex}
\section[视察抚顺煤矿时的指示(一九五八年四月二十九日)]{视察抚顺煤矿时的指示}
\datesubtitle{(一九五八年四月二十九日)}


对于煤的综合利用问题,要好好的研究,这是今后发展的一个重要方向。



\input{4-027-1958.4-对当前工作的十七项指示(传达记录)(一九五八年四月).tex}
\section[对宋××关于苏联专家问题报告的批示(一九五八年五月十六日)]{对宋××关于苏联专家问题报告的批示}
\datesubtitle{(一九五八年五月十六日)}


这是一个好文件,值得一读,请××××立即印发大会同志们,凡有苏联专家的地方,均应照此办理,不许有任何例外。苏联专家都是好同志,有理总是讲得通的,不讲理,或者讲的不高明,因而双方隔离不通,责任在我们方面。就共产主义队伍来说,四海之内皆兄弟,一定把苏联同志看作自己人,大会之后,根据总路线同他们多谈,政治挂帅,尊重苏联专家同志,刻苦虚心学习。但一定要破除迷信,打倒贾桂!贾桂(即奴才)是谁也看不起的。
<p align="right">毛泽东
一九五八年五月十六日启</p>



\input{4-029-1958.5.8-在八大二次会议上的讲话(摘要)(一)(一九五八年五月八日下午四时五十分).tex}
\input{4-030-1958.5.17-在八大二次会议上的讲话(二)(一九五八年五月十七日下午).tex}
\input{4-031-1958.5.2-在八大二次会议上的讲话(三)(一九五八年五月二十日下午).tex}
\input{4-032-1958.5.18-在八人二次会议代表团团长会议上的讲话(一九五八年五月十八日).tex}
\input{4-033-卑贱者最聪明,高贵者最愚蠢(一九五入年五月十八日).tex}
\input{4-034-1958.5.23-在八大二次会议上的讲话(四)(一九五八年五月二十三日下午).tex}
\section[接见阿联军事代表团时的谈话(摘录)(一九五八年五月)]{接见阿联军事代表团时的谈话(摘录)}
\datesubtitle{(一九五八年五月)}


中国人民得到了阿拉伯联合共和国的友谊,感到非常高兴。……全世界人民都支持你们,支持阿拉伯各国人民。我们是站在反殖民主义斗争的一条战线上,相互支持,相互关怀。



\input{4-036-1958.6-关于原子弹、氢弹的指示(一九五八年六月).tex}
\input{4-037-1958.6.17-关于《第二个五年计划指标》的指示(一九五八年六月十七日).tex}
\input{4-038-1958.6.21-在军委扩大会议上的讲话(一九五八年六月二十一日).tex}
\input{4-039-1958.6-在军委扩大会议小组长座谈会上的插话(根据记录整理为九条)(一九五八年六月廿三日).tex}
\input{4-040-1958.6.28-在军委扩大会议小组长座谈会上的讲话(一九五八年六月二十八日).tex}
\section[接见应举社社长时的谈话(摘录)(一九五八年六月)]{接见应举社社长时的谈话(摘录)}
\datesubtitle{(一九五八年六月)}


你们过去是一个穷社,经过几年的努力就改变了面貌,再过几年你们还会更好!

这是由于你们合作社全体社员的努力,才取得了这样大的成绩。全国的事情要办好,就要靠全国六亿人民的努力。

要戒骄戒躁,干部和群众要紧密地团结,要把红旗永远插在你们社里。让红旗越插越高。

民政工作就是作人民工作,不要怕麻烦。

<p align="right">(见一九五八年七月一日《人民日报》)</p>



\input{4-042-1958.7.8-对新华社、《人民日报》的指示谈肃反斗争宣传等问题(一九五八年七月八日××传达).tex}
\section[谴责殖民主义者侵略西亚(一九五八年七月二十八日)]{谴责殖民主义者侵略西亚}
\datesubtitle{(一九五八年七月二十八日)}


毛主席在印度新任驻华大使向他递交国书时致答词说:

“目前由于殖民主义者对西亚的侵略,至使国际和平受到严重威胁。”

毛主席还指出:我们“必将更加高举五项原则的旗帜,为维护亚洲和世界和平作出应有的努力。”



\input{4-044-1958.8.4-视察徐水时的谈话(摘录)(一九五八年八月四日).tex}
\section[视察山东时的谈话(摘录)(一九五八年八月九日)]{视察山东时的谈话(摘录)}
\datesubtitle{(一九五八年八月九日)}


毛主席特别强调布置各项工作必通过群众鸣放辩论,他说:计划、指示不经过群众辩论,主意是你们的;辩论后,群众自己是主人了,干劲自然更足。

领导必须多到下面去看,帮助基层干部总结经验,就地进行领导。

还是办人民公社好,它的好处是,可以把工、农,商、学、兵合在一起,便于领导。

你们的小米长得不好嘛,我看群众的干劲太少!

好,你这个人(指历城县北园乡北园农业社主任李书成。)不干就不干,一干就大干的。

你好,你(指山东省农科所付所长秦杰)学的学问能用上了。

是应该压迫你们一下,不压迫,你们就不会上梁山。

你们研究一下(棉花)为什么落桃,(棉花落桃)的问题,是否可以研究个办法,叫它少落或不落。

你们行还是农民行?

那很好,你们要继续努力,力争上游。

(在接见山东著名的劳动模范,农业社的干部们时说)你们干得很好,都鼓足了干劲。



\input{4-046-1958.8.13-视察天津时的谈话(汇集)(一九五八年八月十三日).tex}
\section[接见西哈努克时的讲话(摘录)(一九五八年八月十五日)]{接见西哈努克时的讲话(摘录)}
\datesubtitle{(一九五八年八月十五日)}


学生数目也要控制一下。小学没有关系。中学要控制一下,因为没有那么多事等着给他们做。……

不要搞普通的学校,要搞些技术学校,农业学校。



\input{4-048-1958.8-在《中央关于在农村建立人民公社问题的决议》中所加的一段话(一九五八年八月).tex}
\input{4-049-1958.8.17-在北戴河政治局扩大会议上的讲话(一)(一九五八年八月十七日).tex}
\input{4-050-1958.8.19-在北戴河政治局扩大会议上的讲话(二)(一九五八年八月十九日).tex}
\input{4-051-1958-在北戴河政治局扩大会议上的讲话(三)(一九五八年入月二十一日上午).tex}
\input{4-052-1958.8.21-在北戴河政治局扩大会议上的讲话(四)(一九五八年八月二十一日下午).tex}
\input{4-053-1958.8.30-在北戴河政治局扩大会议讲话(五)(一九五八年八月三十日上午).tex}
\input{4-054-1958.8-关于主要矛盾问题的讲话(一九五八年八月).tex}
\input{4-055-1958.9.2-给××,陈×,富春等同志的一封信.tex}
\input{4-056-1958.9.2-接见巴西记者穆里罗.马罗金.苏乌萨得杜特夫人的谈话(一九五八年九月二日).tex}
\input{4-057-1958.9.4-批评人民日报上的东西有很多大方向不对(一九五八年九月四日×××传达).tex}
\input{4-058-1958.9.5-最高国务会议上的讲话纪要(一)(一九五八年九月五日).tex}
\input{4-059-1958.9.8-在最高国务会议上的讲话(二)(一九五八年九月八日).tex}
\input{4-060-1958.9.9-在最高国务会议上的讲话(三)(一九五八年九月九日).tex}
\section[关于游泳的指示(一九五八年九月)]{关于游泳的指示}
\datesubtitle{(一九五八年九月)}


“你们不要老在游泳池里面练,游泳池小。游来游去就那么一点,要多到江河里去练习。”

“要走出游泳池。”

“在江河游泳,有逆流,可以锻炼意志和勇敢。”

(注:毛主席1958年9月12日在武汉和游泳运动员的谈话)

“游泳是一项很好的运动,应该提倡。”

(注:毛主席1958年9月在安徽和游泳运动员的谈话)



\section[视察武汉大学的指示(一九五八年九月十二日下午)]{视察武汉大学的指示(一九五八年九月十二日下午)}
\datesubtitle{(一九五八年九月十二日)}


青年人,就是要有志气。

……

学生自觉地要求实行半工半读,这是好事情,是学校大办工厂的必然趋势,对这种要求可以批准,并应给他们以积极的支持和鼓励。在教学改革中,应当注意发挥广大师生的积极性,多方面地集中群众的智慧。



\input{4-063-1958.9.13-视察武钢时的指示(一九五八年九月十三日).tex}
\input{4-064-1958.9-视察安徽时的指示(一九五八年九月).tex}
\input{4-065-1958.9.29-巡视大江南北回京后向新华社记者发表重要谈话(一九五八年九月二十九日).tex}
\input{4-066-1958.1.6-国防部长告台湾同胞书(一九五八年十月六日).tex}
\section[对卫生部党组《关于组织西医离职学习中医班总结报告》的批示(一九五八年十月十一日)]{对卫生部党组《关于组织西医离职学习中医班总结报告》的批示}
\datesubtitle{(一九五八年十月十一日)}


××同志:

此件很好。卫生部党组的建议在最后一段,即今后举办西医离职学习中医的学习班,由各省、市、自治区党委领导负责办理。我看如能在一九五八年每个省、市、自治区各办一个七十一一八十人的西医离职学习班,以两年为期,则在一九六零年冬或一九六一年春,我们就有大约二千名这样的中西结合的高级医生,其中可能出几个高明的理论家。此事请与徐××同志一商,替中央写个简短的指示,将卫生部的报告转发给地方党委,请他们加以研究遵照办理。指示中要指出这是一件大事,不可等闲视之。中国医药学是一个伟大的宝库,应当努力发掘,加以提高。指示和附件发出后,可在人民日报发表。



\input{4-068-1958.1.13-国防部长命令福建前线我军对金门炮击再停两星期(一九五八年十月十三日).tex}
\input{4-069-1958.1.25-中华人民共和国国防部长再告台湾同胞书(一九五八年十月二十五日).tex}
\input{4-070-1958.1.27-毛主席在参观中国科学院时和钱学森同志的谈话(一九五八年十月二十七日下午).tex}
\input{4-071-1958.1-听了华北、东北九省农业协作会议的汇报后的指示(一九五八年十月).tex}
\input{4-072-1958.1.31-在石家庄地区的谈话(一九五八年十月三十一日).tex}
\input{4-073-1958.11.1-在邯郸地区的谈话(一九五八年十一月一日).tex}
\input{4-074-1958.11.1-在新乡地区和五个县委书记谈话纪要(一九五八年十一月一日).tex}
\input{4-075-1958.11.2-在为八届六中全会作准备的郑州会议上的讲话第一次讲话(一九五八年十一月二日下午).tex}
\input{4-076-1958.11.6-在为八届六中全会作准备的郑州会议上的讲话第二次讲话(一九五八年十一月六日).tex}
\input{4-077-1958.11.7-在为八届六中全会作准备的郑州会议上的讲话第三次讲话(一九五八年十一月七日下午).tex}
\input{4-078-1958.11.9-在为八届六中全会作准备的郑州会议上的讲话第五次讲话(一九五八年十一月九日).tex}
\input{4-079-1958.11.9-关于读书的建议(一九五八年十一月九日于郑州).tex}
\input{4-080-1958.11.1-在为八届六中全会作准备的郑州会议上的讲话第五次讲话(一九五八年十一月十日上午).tex}
\input{4-081-1958.11.1-在为八届六中全会作准备的郑州会议上的讲话第六次讲话(一九五八年十一月十日下午).tex}
\section[对《人民日报》的两段指示(一九五八年十一月)]{对《人民日报》的两段指示}
\datesubtitle{(一九五八年十一月)}


×××说,主席在郑州谈话(一九五八年)时,对《人民日报》提了两点意见:

从九月份以来,宣传的火力比较集中。九月份以前,中心还不突出。《人民日报》总要冷静些。(有人又说.要当头脑冷静的促进派。)《人民日报》在三年困难时期,登了那么多鬼戏,宣扬有鬼无害论,至今没有进行批判,欠了账不还是不行的。迟早都要还。

《人民日报》一方面宣传反对现代修正主义,一方面又宣传有鬼无害论,自己把自己置于什么位置。



\input{4-083-1958-从我国社会主义建设看《苏联社会主义经济问题》一书(一九五八年).tex}
\input{4-084-1958.11.21-在武昌会议上的讲话第一次讲话(一九五八年十一月二十一日上午).tex}
\input{4-085-1958.11.23-第二次讲话(一九五八年十一月二十三日中午).tex}
\section[中共中央对卫生部党组关于组织西医离职学习中医班总结报告的批示(一九五八年十一月十八日)]{中共中央对卫生部党组关于组织西医离职学习中医班总结报告的批示}
\datesubtitle{(一九五八年十一月十八日)}


中国医药学是我国人民几千年来同疾病作斗争的经验总结。它包含着中国人民同疾病作斗争的丰富经验和理论知识。它是一个伟大的宝库,必须继续努力发掘,并加以提高。



\input{4-087-1958.11.23-记者对一切事物应保持冷静的头脑(一九五八年十一月二十三日×××传达).tex}
\input{4-088-1958.11.25-给周士钊的一封信(一九五八年十一月二十五日).tex}
\input{4-089-1958.11.25-第十次接见朝鲜代表团时的讲话(一九五八年十一月二十五日).tex}
\section[接见金日成时的讲话(一九五八年十一月于武汉)]{接见金日成时的讲话(一九五八年十一月于武汉)}
\datesubtitle{(一九五八年十一月)}


我们的党。我们的国家是巩固的,今天没有什么力量能够推翻我们这个国家,能够推翻我们这个党。我们好好的搞,是可以继续巩固的,但是我们还没有最后巩固,有可能我们来一个大失败,敌人全部的占领上海,占领武汉,占领北京,占领郑州这一条线,我们要退到西北西南山区,有这个可能。但是,我们保证能够经过几年的战斗.最后能够恢复我们中国,最后战败帝国主义。不能只看到现在发展顺利,我们随时要做好这个准备。



\input{4-091-1958.11.25-一个教训(一九五八年十一月二十五日).tex}
\input{4-092-1958.11-对两份关于国际问题报告的批示(一九五八年十一月二十五、二十七日).tex}
\input{4-093-1958.11.3-在武汉和各协作区主任的讲话(一)(一九五八年十一月三十日下午).tex}
\input{4-094-1958.12.1-论帝国主义和一切反动派都是纸老虎-在中共中央政治局武昌会议上的讲话(一九五八年十二月一日).tex}
\input{4-095-1958.12.9-在八届六中全会上的讲话(一九五八年十二月九日).tex}
\input{4-096-1958.12.1-对《张鲁传》评注(陈寿三国志魏志卷九,裴松之注)(一九五八年十二月十日于武昌).tex}
\input{4-097-1958.12.12-在武昌和各协作区主任的讲话(二)(一九五八年十二月十二日).tex}
\input{4-098-给自己诗词作的解释 .tex}
\section[对《清华大学物理教研组对待教师宁“左”勿右》一文的批示(一九五八年十二月二十二日)]{对《清华大学物理教研组对待教师宁“左”勿右》一文的批示}
\datesubtitle{(一九五八年十二月二十二日)}


×××同志:

建议将此件印发给全国一切大专学校,科学研究机关的党委、总支、支委阅读,并讨论一次,端正方向。争取一切可能争取的教授、讲师、助教、研究人员。为无产阶级的教育事业和文化科学事业服务。你看如何?文学艺永团体、报社杂志社和出版机关的党委、总支,也应发去。也应讨论一次,请酌定。



\input{4-100-1958.12-在修改《关于人民公社若干问题的决议》时所加的一段话(一九五八年十二月).tex}
\input{4-101-1958.12.23-接见参加全军政工会议的各军区负责同志时的谈话(一九五八年十二月二十三日).tex}
\section[对大跃进报导的指示(一九五八年末于武昌)]{对大跃进报导的指示(一九五八年末于武昌)}
\datesubtitle{(一九五八年)}


要实事求是,不要浮夸。要作冷静的促进派,要鼓实劲,不要鼓虚劲,冲天的干劲要与严格的科学的态度相结合。



\section[对地方党委领导新闻事业的指示(一九五八年)]{对地方党委领导新闻事业的指示}
\datesubtitle{(一九五八年)}


一九五八年。新华社国内分社和人民日报记者站合并,体制下放给地方党委。曾向主席汇报此事,主席说:

早就应该这样做,过去(分社)省也管不了,现在这个办法是适应国家的情况的。省委管比你们有办法。分社也就不孤单了。就有依靠了。



\input{4-104-1958-关于造船方针的指示(一九五八年×月).tex}
\input{4-105-1959.1.17-给福斯特同志的回信(一九五九年一月十七日).tex}
\input{4-106-1959.1.27-接见德意志民主共和国政府代表团的谈话(摘录)(一九五九年一月二十七日).tex}
\input{4-107-1959.2.2-在省市委书记会上的讲话(一九五九年二月二日).tex}
\input{4-108-1959.2.21-对新、洛、许、信四个地委座谈时的谈话(记录)(一九五九年二月二十一日).tex}
\input{4-109-1959.2.27-在郑州会议上的讲话(一)(一九五九年二月二十七日).tex}
\input{4-110-1959.2.27-在郑州会议上的讲话(二)(一九五九年二月二十七日).tex}
\input{4-111-1959.2.28-在郑州会议上的讲话(三)(一九五九年二月二十八日).tex}
\input{4-112-1959.3.1-在郑州会议上的讲话(四)(一九五九年三月一日).tex}
\input{4-113-1959.3.5-在郑州会议上的讲话(五)(一九五九年三月五日).tex}
\input{4-114-1959.3-在郑州会议上的讲话(六)(一九五九年三月).tex}
\input{4-115-1959.3.9-党内通讯(一)(一九五九年三月九日郑州).tex}
\input{4-116-1959.3.15-党内通讯(二)(一九五九年三月十五日武昌).tex}
\input{4-117-1959.3.17-党内通讯(三)(一九五九年三月十七日武昌).tex}
\input{4-118-1959.3.29-党内通讯(四)(一九五九年三月二十九日北京).tex}
\input{4-119-1959.3.3-对《陶××同志关于五级干部会议的报告》的批示(节录)一九五九年三月三十日.tex}
\input{4-120-1959.4.5-在上海会议上的讲话(一九五九年四月五日).tex}
\input{4-121-1959.4.15-在第十六次最高国务会议上的讲话纪要(一九五九年四月十五日).tex}
\input{4-122-1959.4.29-党内通讯(一九五九年四月二十九日).tex}
\input{4-123-1959.4.27-一九五九年四月二十七日在八届七中全会上的讲话(摘录)——工作方法九条(一九五九年四月).tex}
\input{4-124-1959.4-工作方法十六条(一九五九年四月).tex}
\section[接见智利政界人士的谈话纪录(摘录)(一九五九年五月十五日)]{接见智利政界人士的谈话纪录(摘录)}
\datesubtitle{(一九五九年五月十五日)}


教人是很好的事情,跟学生保持关系,是很愉快的,特别是当大学教授。要研究学问,就要当教授。

我有几篇哲学著作,就是因为教书需要我写,我才写的。因此当教授是好的。要备课,就要自己写讲稿,不用人家的课本最好。学生逼你上课,你总要贩卖一点东西。我现在还想当教授。现在正如你们所说的。我的知识增加了,可能当教授,或是讲师或是助教。



\input{4-126-1959.6-要政治家办报(一九五九年六月×××传达).tex}
\input{4-127-六月二十九日、七月二日谈话纪录(摘录) .tex}
\input{4-128-七月十日讲话纪录 .tex}
\input{4-129-1959.7.23-在庐山会议上的讲话(一九五九年七月二十三日).tex}
\input{4-130-1959.7-对于一封信的评论(一九五九年七月年廿六日).tex}
\section[对三个文件的批示(一九五九年七月廿九日)]{对三个文件的批示(一九五九年七月廿九日)}
\datesubtitle{(一九五九年七月)}


此三件印发各同志。印时请注意,将赫鲁晓夫的那篇(连同中央社的一则纽约消息)放在前面。三篇印在一起。请同志们研究一下,看苏联曾经垮台的公社和我们的人民公社是不是一个东西。看我们的人民公社究竟会不会垮台。如果要垮台的话,有那些是以使它们垮台的因素。如果不垮台的话,又是为什么。不为历史要求的东西,一定垮掉,人为的维持不垮是不可能的。合乎历史要求的东西、一定垮不了,人为的解散也是办不倒的。这是历史唯物主义的大道理。请同志们看一看马克思《政治经济学批判》的序言。近来攻击人民公社的人们就是抬出马克思这个科学原则,当作法宝祭起来打我们,你们难道不害怕这个法宝吗?



\section[给王稼祥的信(一九五九年八月一日)]{给王稼祥的信}
\datesubtitle{(一九五九年八月一日)}


此件请看一下,有些意思。我写了几句话,其意思是驳斥赫鲁晓夫的,将来我拟写文宣传人民公社的优越性。一个百花齐放,一个人民公社,一个大跃进,这三件事赫鲁晓夫是反对的,或者是怀疑的。我看他们是处于被动,我们非常主动。你看如何,这三件是要向全世界作战,包括党内大批反对派和怀疑派。



\input{4-133-1959.8.2-在庐山会议上的讲话(一九五九年八月二日).tex}
\section[给张闻天的信(一九五九年八月二日)]{给张闻天的信}
\datesubtitle{(一九五九年八月二日)}


闻天同志:

怎么搞的?你陷入那个军事俱乐部去了。真是物以类聚,人以群分。你这次安的是什么主意?那样四面八方勤劳辛苦,找出那些漆黑一团的材料,真是好宝贝!你是不是跑到东海龙王敖广那里取来的?不然,何其多也!然而一展览,尽是假的。讲完没两天你就心慌意乱,十五个吊桶打水,八上七下,被人们缠住脱不了身。自作自受,怨得谁人?我认为你是旧病复发,你的老而又老的疟疾原虫还未去掉,现在又发寒热症了。昔人吟疟疾词云:“冷来时冷的冰凌上卧,热来时热的蒸笼里坐,痛时节,痛的天灵破,颤时节,颤的牙关挫,只被你害杀人也么哥,只被你害杀人也么哥,真是寒来暑往人难过。”同志,是不是?如果是那就好了。你这个人很需要大病一场。《昭明文选》第三十四卷枚乘“七发”末云:“此亦天下要言妙道也,太子岂欲闻之乎?于是太子据几而起曰:涣乎若一听圣人辨士之言,涊然汗出,霍然病已。”你害的病与楚太子相似,如有兴趣可以读一读枚乘的“七发”,真是一篇妙文。你把马克思主义的要言妙道通通忘记了,于是乎跑进了军事俱乐部,真是文武合璧,相得益彰,现在有什么办法呢?愿借你同志之箸为同志筹之:两个字曰:“痛改”。承你看得起我,打几次电话,想到我处一谈,我愿意谈,近日有些忙,请待来日,先用此信,达我悃忧。
<p align="right">毛泽东</p>



\input{4-135-1959.8.5-八月二日对《湖南省平江县谈岭公社稻竹大队几十个食堂散伙又恢复的情况》一文的批语(一九五九年八月五日).tex}
\input{4-136-1959.8.6-对《王国藩社的生产情况一直很好》和《目前农村中“闲话”较多的是那些人》二文的批语(一九五九年八月六日).tex}
\input{4-137-1959.8.1-对《安徽省委书记处书记张恺帆下令解散无为县食堂报告》的批语(一九五九年八月十日).tex}
\section[对辽宁省执行中央反右倾指示报告的批语(一九五九年八月十二日)]{对辽宁省执行中央反右倾指示报告的批语}
\datesubtitle{(一九五九年八月十二日)}


印发各省市。各省、市、自治区的情况如何?辽宁那样的反右倾、鼓干劲的部署是否已经做了?效果如何?看来各地都有右倾情绪,右倾思想、右倾活动存在着,增长着。有各种程度不同的情况,有些地方存在着右倾机会主义分子,向党猖狂进攻的情绪,必须按照具体情况加以分析,把这歪风邪气打下去,辽宁作的很好,步骤也好,成绩显着,他们取得了主动权,迫使右倾机会主义分子处于被动。这个经验值得各地注意。



\input{4-139-1959.8.15-为《经验主义,还是马克思列宁主义》一书写的前言(一九五九年八月十五日).tex}
\input{4-140-1959.8.15-对《马克思主义者应该如何正确地对待革命的群众运动》一文的批语(一九五九年八月十五日).tex}
\input{4-141-1959.8.16-机关枪迫击炮的来历及其他(一九五九年八月十六日).tex}
\input{4-142-1959.8.16-右倾机会主义者挑起了斗争(一九五九年八月十六日).tex}
\input{4-143-1959.8.16-关于枚乘《七发》(一九五九年八月十六日).tex}
\input{4-144-附:枚乘《七发》今译 .tex}
\input{4-145-1959.8.17-一次讲话(一九五九年八月十七日).tex}
\section[对张闻天信的批示(一九五九年八月十八日)]{对张闻天信的批示}
\datesubtitle{(一九五九年八月十八日)}


印发各同志,印一百六十多份,发给每人一份,走了的,航送或邮送去。我以极大热情欢迎洛甫这封信。

<p align="right">毛泽东八月十八日</p>



\section[关于游泳的谈话(一九五九年八月)]{关于游泳的谈话}
\datesubtitle{(一九五九年八月)}


“游泳是同大自然作斗争的一种运动,你们应该到大江大海去锻炼。”

毛主席轻快自如地领头游着,并风趣地问一位同志:“你说是人怕水,还是水怕人?”

那位同志回答说:“人怕水?”

毛主席笑着说:“我看是水怕人。水怕人压迫他。”

(1959年8月毛主席去九江和江西省歌舞团的同志游泳时的谈话)



\input{4-148-1959.9.1-给诗刊的第二封信(一九五九年九月一日).tex}
\section[视察人民大会堂时的指示(一九五九年九月五日)]{视察人民大会堂时的指示}
\datesubtitle{(一九五九年九月五日)}


一九五九年九月五日,我们敬爱的伟大领袖毛主席亲自视察了人民大会堂工程,他视察了礼堂的一二层和上海厅之后说:“人民大会堂的建成确实是成绩伟大,你们这么大的功劳,是不是立块碑吧!那将多大呀!照相吧!也站不下呀!”

毛主席又亲切教导说:“要向老红军学习,不为名,不为利,不计报酬,不怕牺牲。红军打仗也没有礼拜天,没有休息,没有加班费,还是学习老红军吧,要和老红军一样”。

毛主席看了宴会厅,询问工程情况,坚定地说:“大跃进就是好,有人说大跃进不好,十三陵水库,人民大会堂就是大跃进的产物。没有大跃进就没有大会堂。让那些右派来看看,究竟是不是大跃进!”

后来在一次会议上,主席又高度赞扬了建筑工人高度的共产主义劳动精神说:“最近,我们看天安门礼堂,只有十个月,过去许多人说不信,请个苏联专家说不信,到了今年六月,苏联专家说有可能,到了九月,他们大都佩服了,说中国确有大跃进。一万二千人,全国各地方调来的,全国各省的力量,技术的力量,人的力量。完全不休息礼拜天,每天三班制,也不搞计件工资,许多人本来是八小时的,结果他做了十二小时不下工,多的四小时他不要钱呢?他不要。有一些工程没完成他不下来,有的两天两夜不睡觉坚持在那里,不是八小时,也不是十二小时而是四十八小时,就在工地上不下来。是不是物质刺激呢?增加几元钱呢?一小时一元钱嘛,他不要,这些人不要。无非是平均工资五十元,就那么一点,但是他们为作一个共同的事业而奋斗。一万二千职工,十个月搞成一大片,这里面不仅有按劳取酬,而且有列宁所讲的伟大创造共产主义礼拜天,有不计报酬在内。”



\input{4-150-1959.9.9-对彭德怀九月九日信的批示(一九五九年九月九日).tex}
\input{4-151-1959.9.11-在中共中央军委扩大会议上和外事会议上的讲话(一九五九年九月十一日).tex}
\section[关于肃反工作的一个批语(一九五九年九月十八日)]{关于肃反工作的一个批语}
\datesubtitle{(一九五九年九月十八日)}


看法妥当,让他们活动,注意观察,大有可为。他们是在如来佛手掌中,跳不出去的,你们应当当作一件大事去办,积极而又艺术地去做观察和侦察的工作。



\input{4-153-1959.1.11-关于发展养猪事业的一封信(一九五九年十月十一日).tex}
\section[对《我们一个社要养猪两万头》一文批语(一九五九年十一月十九日)]{对《我们一个社要养猪两万头》一文批语}
\datesubtitle{(一九五九年十一月十九日)}


请各省市区负责同志注意:如果你们同意的话,就把这篇文章印发一切农业合作社以供参考,并且仿照办理,要知道阳谷县是打虎英雄,武松的故乡,可是这一带没有喂猪的习惯,这个合作社改变了这种习惯,开始喂猪,第一年失败,第二年成功,第三年发展,第四年大发展,平均每人约有两头,共计二万头。这个合作社可以这样做,为什么别的合作社不可以这样做呢?



\section[对新华社的指示(一九五九年十二月)]{对新华社的指示}
\datesubtitle{(一九五九年十二月)}


新华社这些年做了一些工作。但是,在这个问题上,简直没有做什么(按:指发展国外分社)。驻外记者派得太少,没有自己的消息;有也太少。为什么不派?没有干部?中国这么大,抽不出人?是不是中宣部过去没帮助。

应该大发展,各个国家都派,把地球管起来!让全世界都能听到我们的广播。



\input{4-156-在上海会议上的讲话(摘要)(一九六○年一月).tex}
\section[关于科学奖金和学衔的指示(一九六○年一月)]{关于科学奖金和学衔的指示(一九六○年一月)}


斯大林奖金我们没有就不要搞,追逐个人名利地位的事不要搞。我们打了那么多年仗还不是把蒋介石那个特级上将打倒了。勋章、博士那些东西不要搞了。



\input{4-158-使官僚主义走向它的反面——对一个文件的批语(一九六○年春).tex}
\input{4-159-对广东省委《关于当前人民公社工作中几个问题的指示》的批语(一九六○年三月五日).tex}
\input{4-160-关于卫生工作的指示(一九六○年三月十八日).tex}
\input{4-161-对《鞍山市委关于工业战线上的技术革新和技术革命运动开展情况的报告》的批语(一九六○年三月二十二日).tex}
\input{4-162-关于反华问题(一九六○年三月二十二日).tex}
\input{4-163-关于山东六级干部大会情况的批示(一九六○年三月二十三日).tex}
\section[对聂荣臻同志《关于技术革命运动的报告》的批示(一九六○年三月二十五日)]{对聂荣臻同志《关于技术革命运动的报告》的批示(一九六○年三月二十五日)}


我国工业交通战线,农林牧副渔战线,财政贸易交通战线,文教卫生战线和国防战线的技术革命和文化革命的全民运动,正在猛烈发展,新人新事层出不穷,务请你们细心观察随时总结,予以推广。



\input{4-165-在天津会议上的讲话(摘录)(一九六○年三月二十八日).tex}
\input{4-166-接见非洲外宾时的谈话(一九六○年五月七日).tex}
\input{4-167-接见拉丁美洲外宾时的谈话(一九六○年五月八日).tex}
\input{4-168-接见伊拉克、伊朗和塞浦路斯外宾时的谈话(一九六○年五月九日).tex}
\input{4-169-同蒙哥马利的谈话(一九六○年五月二十七日).tex}
\input{4-170-在上海会议上的讲话(一九六○年六月十四日).tex}
\input{4-171-在上海会议上对四个文件的批语(一九六○年六月十五日).tex}
\section[在上海会议上的讲话(一九六○年六月十八日)]{在上海会议上的讲话(一九六○年六月十八日)}


××是一个革命的国家……最大多数的人是我们的朋友。或者可以成为我们的朋友,敌人只是少数,这一点不但应当成为我们对外工作的指导思想,而且应当成为我们长期对外工作的思想。全世界的胜利都是我们的。



\section[十年总结(一九六○年六月十八日)]{十年总结(一九六○年六月十八日)}


前八年照抄外国的经验。但从一九五六年提出十大关系起,开始找到自己的一条适合中国的路线。一九五七年反右整风斗争,是社会主义革命过程中反映了客观规律,而前者则是开始反映中国客观经济规律。一九五八年五月党代表大会制定了一个较为完整的总路线,并且提出了打破迷信,敢想、敢说、敢作的思想,这就开始了一九五八年的大跃进。去年八月发现人民公社是可行的。赫然挂在河南新乡县七里营的墙上的是这样几个字:“七里营人民公社”。我到襄城县、长葛县看了大规模的生产合作社。河南省委史向生同志,中央《红旗》编辑部李友九同志,同遂平县委、嵖岈山党委会同在一起,起草了一个嵖岈山人民公社章程。这个章程基本上是正确的。八月在北戴河中央起草了一个人民公社决议,九月发表。几个月内公社的架子就搭起来了,但是乱子出得不少,与秋冬大办钢铁同时并举,乱子就更多了。于是乎有十一月的郑州会议,提出了一系列的问题,主要谈到价值法则、等价交换、自给生产、交换生产。又规定了劳逸结合,睡眠、休息、工作,一定要实行生产、生活两样抓。十二月武昌会议,作出了人民公社的长篇决议,基本正确,但只解决集体、国营两种所有制的界线问题,社会主义与共产主义的界线问题,一共解决两个外部的界线问题,但还不认识公社内部的三级所有制问题。一九五八年八月北戴河会议提出了××吨钢在一九五九年一年完成的问题。一九五八年十二月武昌会议降至××吨钢。一九五九年一月北京会议是为了想再减一批而召开的。我和××同志对此都感到不安,但会议仍有很大的压力,不肯改。我也提不出一个恰当的指标来。一九五九年四月上海会议规定了一个××指标,仍然不合实际。我在会上作了批评,这个批评之所以作。是在会议开会之前两日,还没有一个成文的盘子交出来,不但各省不晓得,连我也不晓得,不和我商量,独断专行。我生气了,提出了批评。我说我要挂帅,这是大家都记得的。下月(五月)北京中央会议规定指标为××吨,这才反映了客观实际的可能性。五、六、七月出现了一个小小的马鞍形。七、八两月庐山基本上取得了主动,但在农业方面仍然被动,直至于今,管农业的同志,管商业的同志在一个时间内,思想方法有一些不对头,忘记了实事求是的原则,有一些片面思想(形而上学思想)。一九五九年夏季庐山会议,右倾机会主义猖狂进攻。他们教育了我们,使我们基本上清醒了。我们举行反击获得胜利。一九六○年上海会议,规定后三年指标,我感到仍然存在一个极大的危险,就是对于留有余地,对于藏一手,对于实际可能性还要打一个大大的折扣,当事人还不懂得。一九五六年××同志的第二个五年计划,大部分指标,如钢等,替我们留了三年余地,多么好啊!农业方面则犯了错误,指标高了,以至不可能完成,要下决心改,在今年七月的党代表大会上一定要改过来。从此就完全主动了。同志们,主动权是一个极端重要的事情。主动权就是“高屋建瓴”“势如破竹”,这件事来自实事求是,来自客观情况对于人们头脑的真实反映,即人们对于客观外界的辩证法的认识过程,中间经过许多错误的认识,逐步改正这些错误,以归于正确。现在就全党同志来说,他们的思想并不都是正确的,有许多人并不懂得马列主义的立场、观点和方法。我们有责任帮助他们懂得,特别是县、社、队的同志。

看来,错误不可能不犯。如列宁所说,不犯错误的人从来没有,郑重的党在于重视犯错误,找出犯错误的原因,分析可能犯错误的主观和客观的原因,公开改正。我党的总路线是正确的,实际工作也是基本上做得好的。有一部分错误也是难于避免的。哪里有不犯错误一次就完成了真理的所谓圣人呢?真理的认识不是一次完成的,而是逐步完成的。我们是辩证唯物论的认识论者,不是形而上学的认识论者。自由是必然的认识。由必然王国到自由王国的飞跃是在一个长期的认识过程中逐步完成的。对于我国的社会主义革命和建设,我们已经有十年的经验了,已经懂得了不少的东西了,但是我们对于社会主义建设经验还不足,在我们面前,还有一个很大的未被认识的必然王国。我们还不深刻地认识它。我们要在今后实践中,继续调查它、研究它,从而找出它固有的规律,以便利用这些规律为社会主义事业服务。对中国如此,对整个世界也应该如此。

我试图做出一个十年经验的总结。上述这些话,只是一个轮廓,而且是粗浅的,许多问题没有写进去,因为是两个钟头内写出的,以便在今天下午讲一下。



\input{4-174-接见日本文学代表团时的谈话(一九六○年六月二十一日).tex}
\input{4-175-苏联《政治经济学教科书》阅读笔记(社会主义部分、第三版)第一部分(从第二十章到二十三章).tex}
\input{4-176-苏联《政治经济学教科书》阅读笔记第二部分(从第二十四章到第二十九章).tex}
\input{4-177-苏联《政治经济学教科书》阅读笔记第三部分(从第三十章到第三十四章).tex}
\input{4-178-苏联《政治经济学教科书》阅读笔记第四部分(从第三十五章到结束语).tex}
\input{4-179-苏联《政治经济学教科书》阅读笔记(补遗).tex}
\input{4-180-在北戴河会议上的讲话(节录)(一九六○年七月十八日).tex}
\input{4-181-和狄克逊、夏基的谈话(一九六○年九月二十五日).tex}
\input{4-182-记关于农村劳动力的问题对山西省委关于农村劳动力问题的报告的批示(一九六○年十月二十七日).tex}
\input{4-183-彻底纠正五风(一九六○年十一月十五日).tex}
\input{4-184-中央关于转发“甘肃省委关于贯彻中央紧急指示信的第四次报告”的批示(一九六○年十一月二十八日).tex}
\input{4-185-接见厄瓜多尔文化代表团和古巴妇女代表团时的谈话(摘要)(一九六○年十二月二十日).tex}
\input{4-186-1958.1-批评《人民日报》不应“反冒进”(传达纪要)(一九五八年一月).tex}
\section[与陈昌奉同志的谈话(一九五八年八月九日)]{与陈昌奉同志的谈话}
\datesubtitle{(一九五八年八月九日)}


山东地方很好,根据地很大,新解放了许多县城,很需要干部。到那里要尊重地方,尊重地方干部,尊重新单位的负责人,要和当地干部搞好团结。

学习也好,倒不是为当官,而是为人民多做些事。当然,学习中间也有很多困难,只要你有为人民服务的决心,什么困难都能克服。

指导员也是革命工作嘛!非当上什么官才行吗?

想当官的人到共产党这里来就是走错门啦!我们这里还有什么官不官啊!都是一样。我们是为人民服务的,到我们这里发财当然不用说了。想升官也达不到目的。



\input{4-188-1958.12.4-在八届六中全会上的讲话(一九五八年十二月四日下午).tex}
\input{4-189-1958.12.6-关于当前宣传工作中应注意的几个问题的指示(在武昌与胡××、吴××谈当前宣传工作中应注意的几个问题传达记录)(一九五八年十二月六日).tex}
\input{4-190-1959.3-在上海会议上的讲话(摘录,大意)(一九五九年三月).tex}
\input{4-191-1959-在上海会议上的讲话和插话(一九五九年三,四月).tex}
\input{4-192-1959.5.15-论中印关系问题(一九五九年五月十五日).tex}
\input{4-193-1959.8.11-在八届八中全会上的讲话(一九五九年八月十一日).tex}
\input{4-194-1959.9.14-在第二届全国人民代表大会常务委员会第九次会议上提出的中国共产党中央委员会的建议(一九五九年九月十四日).tex}
\input{4-195-关于坚决地认真地清理劳动力加强农业生产第一线的紧急指示(一九六○年八月二十日).tex}
\input{4-196-关于官僚主义严重存在的问题(对一个文件的批语)(一九六○年春).tex}


\end{document}