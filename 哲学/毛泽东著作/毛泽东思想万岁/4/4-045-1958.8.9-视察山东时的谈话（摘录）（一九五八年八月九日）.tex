\section[视察山东时的谈话(摘录)(一九五八年八月九日)]{视察山东时的谈话(摘录)}
\datesubtitle{(一九五八年八月九日)}


毛主席特别强调布置各项工作必通过群众鸣放辩论,他说:计划、指示不经过群众辩论,主意是你们的;辩论后,群众自己是主人了,干劲自然更足。

领导必须多到下面去看,帮助基层干部总结经验,就地进行领导。

还是办人民公社好,它的好处是,可以把工、农,商、学、兵合在一起,便于领导。

你们的小米长得不好嘛,我看群众的干劲太少!

好,你这个人(指历城县北园乡北园农业社主任李书成。)不干就不干,一干就大干的。

你好,你(指山东省农科所付所长秦杰)学的学问能用上了。

是应该压迫你们一下,不压迫,你们就不会上梁山。

你们研究一下(棉花)为什么落桃,(棉花落桃)的问题,是否可以研究个办法,叫它少落或不落。

你们行还是农民行?

那很好,你们要继续努力,力争上游。

(在接见山东著名的劳动模范,农业社的干部们时说)你们干得很好,都鼓足了干劲。


