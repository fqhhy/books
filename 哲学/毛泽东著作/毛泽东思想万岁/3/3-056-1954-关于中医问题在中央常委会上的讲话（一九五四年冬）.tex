\section[关于中医问题在中央常委会上的讲话(一九五四年冬)]{关于中医问题在中央常委会上的讲话(一九五四年冬)}
\datesubtitle{(一九五四年)}


我认为中国对世界上的大贡献,中医是其中的一项。中医在手工业、农业的生产基础上产生的,对金、木、水、火、土的理论可以加予批判,宝贵的经验必须加以保护和发扬。批判必须懂得,什么叫科学,正确的,系统的知识叫科学。西医是否科学,也带有唯心的,像机械唯物也需要加以改造。中国应有一个医,不应该长久的有两医,西医名称是不妥当的,应有唯物辩证的一个医。有的人看不起中医,看不起中医是错误的,也有人把中医强调得太夸大了,也是不对的。中医医院要有重点的试办,中医进修问题,对其进修基础课是对的,应加入经验交流课程,讲中医药理恐有困难。


