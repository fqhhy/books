\section[扫除文盲,发展农村文化教育事业(一九五六年)]{扫除文盲,发展农村文化教育事业}
\datesubtitle{(一九五六年)}


从一九五六年开始,按照各地情况,分别在十二年内,基本上扫除青年和壮年中的文盲,普及小学教育。要求做到一般的社有小学和业余文化学校,一般的乡有农业中学,以便进一步提高农村基层干部和农民的文化水平。农村办学应当采取多种形式,除了国家办学以外,必须大力提倡群众集体办学。提倡勤工俭学。

随着农业生产的发展,合作社应当根据可能的条件,按照勤俭建国,勤俭办社,勤俭持家的原则,逐步改进和开展文化娱乐工作和体育活动。

<p align="right">(摘自《1956-1967年全国农业发展纲要(第二次修改草案)》第三十一条)</p>


