\section[在中央工作会议上的讲话(一九六六年十月二十五日)]{在中央工作会议上的讲话}
\datesubtitle{(一九六六年十月二十五日)}


讲几句话,两件事。

十七年来,有一件事我看做得不好,就是搞一、二线。原来的意思是考虑到国家的安全,鉴于苏联斯大林的教训,搞了一线二线,我处在二线,别的同志在一线。现在看来不那么好,结果很分散,一进城就不能集中了,相当多的独立王国,所以十一中全会作了改变,这是一件事。我处在二线日常工作不主持,许多事让别人去搞,培养别人的威信,以便我见上帝的时候,国家不会出现那么大的震动,大家赞成这个意见,后来处在一线的同志,有些事情处理得不那么好。有些应当我抓的事情,我没有抓,所以,我也有责任,不能完全怪他们。为什么说我也有责任呢?

第一,常委分一、二线,搞书记处,是我提议的,大家同意了。再嘛是过于信任别人了。这件事引起警惕,还是在制定二十三条那个时候。北京就是没有办法,中央也没有办法。去年九、十月提出中央出了修正主义,地方怎么办?我就感到我的意见在北京不能实行。为什么批判吴晗不在北京发起,而在上海发起呢?因为北京没有人办。现在北京问题解决了。

第二件事,文化大革命闯了一个大祸,就是批发了北大聂元梓一张大字报,给清华附中写了一封信,还有我自己写了一张《炮打司令部》的大字报。这几件事,时间很短,六、七、八、九、十,五个月不到,难怪同志们还不那么理解。时间很短,来势很猛,我也没有料到。北大大字报一广播,全国都闹起来了,红卫兵信还没有发出,全国红卫兵都动起来了,一冲就把你们冲了个不亦乐乎。我这个人闯了这么个大祸。所以你们有怨言,也是难怪的。\marginpar{\footnotesize 274}

上次开会我是没有信心的,说过决定通过了不一定能执行,果然很多同志还是不那么理解。现在经过两个月了,有了经验,好一点了。这次会议两个阶段,头一个阶段,大家发言都不那么正常,后一个阶段经中央同志讲话,交流经验,就比较顺了,思想就通了一些。运动只搞五个月,可能要搞两个五个月,也许还要多一点。

民主革命搞了二十八年(1921—1924年)。开始搞民主革命,谁也不知道怎么个革法,斗争怎么斗争法,以后才摸出一些经验。路也是一步一步从实践中走出来的,总结经验,搞了二十八年嘛。社会主义革命也搞了十七年,文化革命只有五个月嘛,所以就不能要求同志们都就那么理解。去年批判吴晗的文章,许多同志不去看,不那么管。以前批判武训传、红楼梦研究,是个别抓,抓不起来,不全盘抓不行,这个责任在我。个别抓,头痛医头,脚痛医脚,是不能解决问题的。这次文化大革命,前几个月,一、二、三、四月用那么多文章,中央又发了通知,可是并没有引起多大注意,还是大字报、红卫兵一冲,引起注意,不注意不行了。革命革到自己头上来了,赶快总结经验,做政治思想工作,为什么两个月之后又开这个会?就是总结经验,做好政治思想工作。你们回去以后有大量的政治思想工作要作。中央局、省委、地委、县委召开十几天会,把问题讲清楚,也不要以为所有都能讲清楚。有人说,“原则通了,碰到具体问题处理不好。”原来我想不通,原则问题搞通了,具体问题还不好处理?现在看来还是有点道理,恐怕还是政治思想工作没有作好。上次开会回去,有些地方没有来得及很好开会,十个书记有七、八个接待红卫兵,一冲就冲乱了,学生们生了气,自己还不知道,也没有准备回答问题,还以为几十分钟讲一讲,表示欢迎就可以了。人家一肚子气,几个问题一问不能回答就被动了。这个被动是可以改变的,可以变被动为主动的。所以我对这次会议信心增强了。不知你们怎么样?如果回去还是老章程,维护现状,让一派红卫兵对立,拉另一派红卫兵保驾,就搞不好。我看会改变,情况会好转。当然不能过多地要求中央局、省、地、县广大干部全部都那么豁然贯通。不一定,总有那么一些人不通,有少数人是要对立的,但是我相信多数讲得通的。

上面讲两件事情:

第一件事讲历史,一件事讲历史,十七年一线二线,不统一,别人有责任,我也有责任。

第二件事,五个月文化大革命,火是我点起来的,时间很仓促。与廿八年民主革命和十七年社会主义革命比起来,时间是很短的,只有五个月。不到半年,不那么通,有抵触情绪,是可以理解的。为什么不通!你们过去只搞工业、农业、交通,就是没有搞文化大革命,你们外交部也一样,军委也一样。你们没有想到的事情来了,来了就来了,我看冲一下有好处,多少年没有想,一冲就想了。无非是犯错误,什么路线错误,改了就算了,谁要打倒你们!我也是不想打倒你们,我看红卫兵也不要打倒你们。有两个红卫兵对李雪峰讲:“没有想到我们老前辈为什么怕红卫兵?”还有修权四个小孩分成四派,有的同学到他家里来,有时一来好几十个,有好处,我看跟小孩接触很有好处。大接触一百五十万几个钟头就结束了,也是一种方式,各有各的作用。

这次会议发的简报不少,我几乎全部看了。你们过不了关,我也不好过,你们着急,我也着急,不能怪同志们,因为时间太短。有的同志说不是有心犯错误,是糊里糊涂犯了错误。可以原谅也不能完全怪刘少奇同志和邓小平同志,他们有责任,中央也有责任,中央也没有管好,时间太短,新的问题没有精神准备,政治思想工作没有做好,我看十七天会议以后会好一些。

还有哪个讲?今天就完了,散会。\marginpar{\footnotesize 275}

