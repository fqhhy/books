\documentclass[b5paper,oneside,12pt]{ctexbook}
\usepackage[hmargin=0.3in,vmargin=0.5in]{geometry} 
\usepackage[]{multirow}
\usepackage{calc} % 可以在命令中计算长度

\pagestyle{plain} %整书页眉页脚设置
\setlength{\marginparsep}{2pt} % 边注设置
\setlength{\marginparwidth}{20pt}

% 章节样式
\ctexset{chapter/numbering=false}
\ctexset{
    section={numbering=false, afterskip = 0ex},
    subsection={format=\large\heiti\centering,numbering=false,beforeskip=1ex,afterskip = 1.75ex}
}


\usepackage[perpage,hang]{footmisc} %脚注
%% 开启 footmisc 的 hang 选项
\setlength{\footnotemargin}{1.25em}     % 整个脚注文本的左边距,加此边距,来显示脚注序号。
\setlength{\skip\footins}{1\baselineskip} % 脚注线 和 脚注内容 间距 
% \setlength{\footnotesep}{\skip\footins} % 两个脚注文本之间的间距
\renewcommand{\hangfootparskip}{0pt}
\renewcommand{\hangfootparindent}{2em}

\usepackage{etoolbox}
% 一些LATEX内部命令含有@字符,如\@addtoreset,如需使用这些内部命令,就需要借助于另两个命令\makeatletter和\makeatother.
% \makeatletter
% 此补丁去掉脚注内容前面序号的上标样式,不与 footnotebackref 包兼容,与hyperref 兼容
% \patchcmd{\@makefntext}
    % {\@makefnmark}
    % {\hbox{\normalfont\@thefnmark}}
    % {}{}

% 此补丁,可用于仅使用 footmisc 时,去除上标样式,可能要改动
% \patchcmd{\@makefntext}
% {\ifFN@hangfoot\bgroup}
% {\ifFN@hangfoot\bgroup\def\@makefnmark{\normalfont\@thefnmark}}
% {}{}
% \makeatother

% 给脚注编号前后添加 〔〕,hspace 为了调整加完〔〕之后的间距
\renewcommand\thefootnote{{\hspace{-0.55em}〔\arabic{footnote}〕\hspace{-0.68em}}} 

\usepackage[]{hyperref} % 放在 footmisc 后面

\usepackage[numberlinked]{footnotebackref} % 与 hyperref 同时使用需要注意
\makeatletter
% 此补丁 与 footnotebackref 包配合,去掉脚注内容前面序号的上标样式 
\patchcmd{\@makefntext}{\textsuperscript}{}{}{}
\makeatother


%%%%%% 自定义新样式
%自定义日期副标题格式,为了保险,最好使用两层大括号
\newcommand\datesubtitle[1]{{\centering\large #1\par\vspace{1ex}}}  
% 引用样式:使用 latex 原始的 list 环境
\newenvironment{yinyong}{%
    \begin{list}{}{\parsep\parskip
        \setlength\topsep{0pt}
        \setlength\itemindent{2em}%
        \setlength\parindent{2em}
        \setlength\listparindent{2em}
        \setlength{\leftmargin}{2em}
        \setlength{\rightmargin}{2em}
        \kaishu
    }
    \item[]
}{
  \end{list}
}

% 靠右对齐,右边距2字
\newcommand{\kaoyouerziju}[1]{{\raggedleft #1 \hspace{2em} \par}}
% 楷体,右边距5字
\newcommand{\kaitiqianming}[1]{{\raggedleft\large\kaishu\ziju{1} #1 \hspace{5em} \par}}

% 一人带职位
\newcommand{\yirendaizhiwei}[2]{
    {\setlength{\tabcolsep}{0em}
    {\raggedleft\begin{tabular} {cc}%
        #1 & \quad{} #2 \hspace{4em} \\ 
        \end{tabular} \\[1pt]}}
}

% 右下定宽
\newcommand{\youxiadingkuan}[1]{
    \begin{list}{}{
        \setlength{\topsep}{0pt}        % 列表与正文的垂直距离
        \setlength{\partopsep}{0pt}     % 
        \setlength{\parsep}{\parskip}   % 一个 item 内有多段,段落间距
        \setlength{\itemsep}{0pt}       % 两个 item 之间,减去 \parsep 的距离
        \setlength{\itemindent}{0pt}%
        \setlength\parindent{0pt}
        \setlength\listparindent{0pt}
        \setlength{\leftmargin}{0.4\linewidth}
        \setlength{\rightmargin}{2em}
    }
    \item[] #1
    \end{list}
}

% 对话,文章为几人对话的样式
\newenvironment{duihua}{
	\begin{list}{}{
        \setlength{\topsep}{0pt}        % 列表与正文的垂直距离
        \setlength{\partopsep}{0pt}     % 
        \setlength{\parsep}{\parskip}   % 一个 item 内有多段,段落间距
        \setlength{\itemsep}{\lineskip}       % 两个 item 之间,减去 \parsep 的距离
        \setlength{\labelsep}{0pt}%
        \setlength{\labelwidth}{3em}%
        \setlength{\itemindent}{0pt}%
        \setlength\listparindent{\parindent}
        \setlength{\leftmargin}{3em}
        \setlength{\rightmargin}{0pt}
        }
}{\end{list}}

\title{毛泽东思想万岁\\1960.1—1968.5}
\author{毛泽东}
\date{}

\begin{document}

\frontmatter
\maketitle
\tableofcontents

\mainmatter
% \chapter{}
\input{5-001-1961.1.13-在中央工作会议上的讲话(一九六一年一月十三日).tex}
\input{5-002-1961.1.18-在八届九中全会上的讲话(一九六一年一月十八日).tex}
\section[《反对本本主义》说明(一九六一年三月十一日)]{《反对本本主义》说明\\{\large(原名:“关于调查研究”)}}
\datesubtitle{(一九六一年三月十一日)\footnote{据《毛泽东年谱》,3月10日—13日 在广州小岛招待所主持召开三南会议,主要讨论人民公社体制和工作条例问题。同时,刘少奇、周恩来、陈云、邓小平于三月十一日至十三日在北京主持召开有中共中央西北局、东北局、华北局及所属各省、市、自治区负责人参加的工作会议(即三北会议),讨论的问题与三南会议相同。后来,北京方面向毛泽东建议,两边合起来开会,得到毛泽东同意。十四日,参加三北会议的同志到达广州。3月11日 同胡乔木、田家英谈《调查工作》一文修改问题。同日 为印发《调查工作》一文给三南会议写如下批语:“这是一篇老文章,是为了反对当时红军中的教条主义思想而写的。那时没有用‘教条主义’这个名称,我们叫它做‘本本主义’。写作时间大约在一九三〇年春季,已经三十年不见了。一九六一年一月,忽然从中央革命博物馆里找到,而中央革命博物馆是从福建龙岩地委找到的。看来还有些用处,印若干份供同志们参考。”并批注:“送林彪同志阅,一九三〇年的,从闽西找出来的。阅后退毛。”毛泽东在印发这篇文章时,对正文作了一些文字修改,将标题改为《关于调查工作》。}}

这是一篇老文章,是为了反对当时红军中的教条主义思想而写的。那时没有用“教条主义”这个名称,我们叫它做“本本主义”。\marginpar{6}写作时间大约在一九三〇年春季,已经三十年不见了。一九六一年一月,忽然从中央革命博物馆里找到。而中央革命博物馆是从福建龙岩地找到的。看来还有些用处。印若干分供同志们参考。

\kaitiqianming{毛泽东}
\kaoyouerziju{一九六一年三月十一日}


\section[《中央关于认真进行调查研究工作问题给各中央局、省、市、自治区党委一封信》(摘录)(一九六一年三月二十七日)]{《中央关于认真进行调查研究工作问题给各中央局、省、市、自治区党委一封信》(摘录)}
\datesubtitle{(一九六一年三月二十七日)}

毛主席提倡哲学要走出课堂,走出书斋。毛主席讲:真理在谁手里,我们就跟谁走,挑大粪的人有真理,我们就跟挑大粪的人走。

\input{5-005-1961.3-在广州会议上的讲话(节录)(一九六一年三月).tex}
\input{5-006-1961.4.28-在接见亚洲、非洲外宾时的谈话(一九六一年四月二十八日).tex}
\input{5-007-1961.4.19-在接见古巴文化代表团时的谈话(一九六一年四月十九日).tex}
\section[调查成灾一例(一九六一年五月三十日)]{调查成灾一例\\{\large——对“关于‘调查研究’的调查”的批示}}
\datesubtitle{(一九六一年五月三十日)}

如果还是如同去长辛店铁道机车车辆制造厂做调查的那些人们实行官僚主义的老爷式的使人厌恶得透顶的那种调查方法,党委有权教育他们。死官僚不听话的,党委有权把他们轰走。\marginpar{8}\footnote{5月28日 阅田家英报送的戚本禹五月十二日写的材料《关于“调查研究”的调查》和田家英报送这个材料的信。田家英的信中说:秘书室工作人员戚本禹,去年六月下放到长辛店机车车辆工厂劳动。最近他寄了一份材料给我,反映一些机关、学校人员到工厂作调查的情况。这个材料提出了一些在大兴调查研究之风中间值得注意的问题。戚本禹的材料说,他们利用业余时间摸了一下各级领导机关到长辛店机车车辆厂做调查研究工作的情况,认为在二十几个调查组的工作里,比较普遍地存在着“十多十少”的问题。毛泽东为戚本禹的材料拟了一个题目《调查成灾的一例》。批示:“此件印发工作会议各同志。同时印发中央及国家机关各部门各党组。派调查组下去,无论城乡,无论人多人少,都应先有训练,讲明政策、态度和方法,不使调查达不到目的,引起基层同志反感,使调查这样一件好事,反而成了灾难。”30日,毛泽东对这个材料再次批示:“此件,请中央及国家机关各部门各党组,各中央局,各省、市、区党委,一直发到县、社两级党委,城市工厂、矿山、交通运输基层党委,财贸基层党委,文教基层党委,军队团级党委,予以讨论,引起他们注意,帮助下去调查的人们,增强十少,避免十多。如果还是如同下去长辛店铁道机车车辆制造工厂做调查的那些人们,实行官僚主义的老爷式的使人厌恶得透顶的那种调查法,党委有权教育他们。死官僚不听话的,党委有权把他们轰走。同时,请将这个文件,作为训练调查组的教材之一。”}\footnote{见链接《毛泽东年谱(1949—1976)》http://dangshi.people.com.cn/n/2013/0527/c85037-21626561-7.html}

\input{5-009-1961.6.12-在北京会议上的讲话(一九六一年六月十二日).tex}
\input{5-010-1961.7.30-给江西共产主义劳动大学的一封信(一九六一年七月三十日).tex}
\input{5-011-1961.9.6-对《各地贯彻执行六十条的情况和问题》的批语(一九六一年九月六日).tex}
\input{5-012-1961.9.29-给政治局常委及有关同志的信(一九六一年九月二十九日).tex}
\input{5-013-1961.10.7-在接见日本朋友时的谈话(一九六一年十月七日).tex}
\section[关于科学研究十四条的指示(一九六二年一月)]{关于科学研究十四条的指示}
\datesubtitle{(一九六二年一月)}

“……科学研究工作十四条,这些条例草案已经在实行或者试行,以后要修改,有些还可能大改”,“应当好好地总结经验,制定一整套的方针政策和办法,使他们在正确的轨道上前进”。“在总路线指导下,制定一整套的方针、政策和办法,必须通过从群众中来的方法,通过做系统的、周密的调查研究的方法,对群众中的成功经验和失败经验,作历史的考察,才能找出客观事物所固有的而不是人们主观臆造的规律,才能制定适合情况的各种条例。这事很重要,请同志们注意到这点”。

\section[给郭沫若回信中的几句话(一九六二年一月)]{给郭沫若回信中的几句话}
\datesubtitle{(一九六二年一月)}

一九六一年,郭沫若看了《孙悟空三打白骨精》后,写了一首诗\footnote{郭沫若诗:《七律·孙悟空三打白骨精》人妖颠倒是非淆,对敌慈悲对友刁。咒念金箍闻万遍,精逃白骨累三遭。千刀当剐唐僧肉,一拔何亏大圣毛。教育及时堪赞赏,猪犹智慧胜愚曹。},毛主席十一月十七日写了一首七律《和郭沫若同志》\footnote{一从大地起风雷,便有精生白骨堆。僧是愚氓犹可训,妖为鬼蜮必成灾。金猴奋起千钧棒,玉宇澄清万里埃。今日欢呼孙大圣,只缘妖雾又重来。},站得更高,看得更远。以后一九六二年一月六日郭沫若又作了一首和主席的诗,诗曰:

赖有晴空霹雳雷,不教白骨聚成堆。九天四海澄迷雾,八十一番弭大灾。僧受折磨知悔恨,猪期振奋报涓埃。金睛火眼无容赦,哪怕妖精亿度来。

该和诗由康生同志转交主席,主席回信郭沫若说“和诗好,不是‘千刀当剐唐僧肉’。对中间派采取了统一战线政策,这就好了。”

\input{5-016-1962.1.30-在扩大的中央工作会议上的讲话(一九六二年一月三十日).tex}
\input{5-017-1962.5.3-接见几内亚政府经济代表团和妇女代表团的谈话(一九六二年五月三日).tex}
\input{5-018-1962.8.6-在北戴河中央工作会议上的讲话(一九六二年八月六日).tex}
\input{5-019-1962.8.9-在北戴河中央工作会议中心小组会上的讲话(一九六二年八月九日).tex}
\section[对中共中央组织部的批评(一九六二年八月十二日)]{对中共中央组织部的批评}
\datesubtitle{(一九六二年八月十二日)}


中共中央组织部从来不向中央作报告,以至中央同志对组织部同志的活动一无所知,全部封锁,成了一个独立王国。


\input{5-021-1962.9.24-在八届十中全会上的讲话(一九六二年九月二十四日上午怀仁堂).tex}
\input{5-022-1962.7.24-在八届十中全会上对工业支援农业的指示(一九六二年七月二十四日).tex}
\section[关于电台的指示(一九六二年十月一日)]{关于电台的指示}
\datesubtitle{(一九六二年十月一日)}


中近东许多国家发生政变。搞政变的人开始就要夺取电台,向全国和全世界说话,原政府的声音人们就听不到了。我们的电台怎么样?是否掌握在可靠的人手里?要从部队调一个强的干部去。


\section[批评新华社(一九六二年)]{批评新华社}
\datesubtitle{(一九六二年)}


《内部参考》登那么多包产到户的材料是错误的,今后不要再登。办内部参考要有方向。\marginpar{\footnotesize 38}


\section[听取中印边境自卫反击战汇报时的指示(一九六三年二月)]{听取中印边境自卫反击战汇报时的指示}
\datesubtitle{(一九六三年二月)}


看来我们的军队还是要政治工作,抓四个第一,抓三大民主,加强薄弱环节,搞好党的建设。


\section[同苏修大使契尔沃年科的谈话(一九六三年二月二十三日)]{同苏修大使契尔沃年科的谈话}
\datesubtitle{(一九六三年二月二十三日)}


(苏修大使契尔沃年科求见主席,主席称不适。因再三要求,主席着睡衣接见。)
\begin{duihua}
\item[\textbf{契:}] 听说你们要发表文章,不必要的,感到很沉重。

\item[\textbf{主席:}] 不必要,你们为什么发表那么多文章?没有什么沉重的,不过互相论战,不过是唇枪舌箭而已。

\item[\textbf{契:}] 请你到莫斯科去谈一谈,可以吗?

\item[\textbf{主席:}] 我已经老了,不中用了。老而不死,破套鞋去不成了。
\end{duihua}


\input{5-027-1963.5.4-接见阿尔巴尼亚劳动青年联盟代表团、新闻工作者代表团、工会代表团和档案工作者代表团的谈话(一九六三年五月四日).tex}
\input{5-028-1963.5.8-对四个文件的批示(一九六三年五月八日).tex}
\input{5-029-1963.5.8-对东北和河南两件报告的批示(一九六三年五月八日).tex}
\input{5-030-1963.5-在关于四清运动中央工作会议上的讲话(一九六三年五月).tex}
\input{5-031-1963.5-关于《山西省昔阳县干部参加劳动已形成社会风尚》一文的批语(一九六三年五月).tex}
\input{5-032-1963.5.9-对《浙江省七个关于干部参加劳动的好材料》的批示(一九六三年五月九日).tex}
\input{5-033-1963.5-关于农村社会主义教育等问题的指示(一九六三年五月).tex}
\input{5-034-1963.5-在杭州会议上的谈话(一九六三年五月).tex}
\section[批示×××在农村蹲点至少五个月(一九六三年六月三日)]{批示×××在农村蹲点至少五个月}
\datesubtitle{(一九六三年六月三日)}
{\noindent\kaishu\centering (写在中央工作会议简报李雪峰同志发言上面)\par}

此件送×××一阅,阅后退我。

你应当下决心在今冬明春这段期间,在北京农村地区或天津郊区蹲点,至少五个月。家里工作可以间接或抽时间回来处理。从新华社和人民日报抽出一批人相当地干部合组一个工作队,包一个最坏的人民公社,一直把工作做完,以后并成为你们经常联系的一个点。还要在一个冬春,参加城市五反。千万不要放弃参加这次伟大革命的机会。

\kaitiqianming{毛泽东}
\kaoyouerziju{六月三日}
\marginpar{\footnotesize 54}


\input{5-036-1963.7.26-六月三日接见古巴文化、工会、青年等代表团的谈话(一九六三年七月二十六日).tex}
\input{5-037-1963.8.1-八连颂(一九六三年八月一日).tex}
\input{5-038-1963.8.8-呼吁世界人民联合起来反对美国帝国主义的种族歧视、支持美国黑人反对种族歧视的斗争的声明(一九六三年八月八日).tex}
\input{5-039-1963.8.29-反对美国——吴庭艳集团侵略和屠杀越南南方人民的声明(一九六三年八月二十九日).tex}
\section[电唁杜波依斯博士逝世(一九六三年八月二十九日)]{电唁杜波依斯博士逝世}
\datesubtitle{(一九六三年八月二十九日)}


杜波依斯夫人:

我沉痛地获悉杜波依斯博士逝世的消息,谨向你表示深切的哀悼。

杜波依斯博士是我们时代的一伟人。他为黑人和全人类的解放进行英勇的斗争的事迹、在学术上的卓越成就,和他对中国人民的真挚友谊,将永远留在中国人民的记忆里。

\kaitiqianming{毛泽东}
\kaoyouerziju{一九六三年八月二十九日}
{\raggedleft(《人民日报》1963.6.30.)\par}


\section[在中央工作会议上对文学艺术的指示(一九六三年九月)]{在中央工作会议上对文学艺术的指示}
\datesubtitle{(一九六三年九月)}


戏剧要推陈出新,不应推陈出陈,光唱帝王将相,才子佳人和他们的丫头保镖之类。
\marginpar{\footnotesize 61}

\section[祝贺霍查同志五十五岁生日的电报(一九六三年十月十五日)]{祝贺霍查同志五十五岁生日的电报}
\datesubtitle{(一九六三年十月十五日)}

地拉那
阿尔巴尼亚劳动党中央委员会第一书记
亲爱的思维尔·霍查同志:

在你五十五岁生日的时候,我代表中国共产党和中同人民,并且以我个人的名义,向你,阿尔巴尼亚劳动党的创始者和领导者,阿尔巴尼亚人民敬爱的领袖、中国人民亲密的朋友,致以衷心的、兄弟般的祝贺。

你把自己的全部精力献给阿尔巴尼亚人民反对法西斯的解放斗争和社会主义革命、社会主义建设的事业。以你为首的久经考验的阿尔巴尼亚劳动党正确地领导着英雄的阿尔巴尼亚人民,高举反对帝国主义的旗帜,为反对现代修正主义、反对现代教条主义,捍卫马克思主义和维护国际共产主义运动的团结,作出了卓越的贡献。中国共产党和中国人民,对你和以你为首的阿尔巴尼亚劳动党,对阿尔巴尼亚人民,表示崇高的敬意。

祝阿尔巴尼亚人民在社会主义建设事业之中取得更加辉煌的成就。祝中间两党和两国人民的兄弟友谊万古长青。祝你,亲爱的霍查同志,健康,长寿!

\yirendaizhiwei{中国共产党中央委员会主席}{毛泽东}

\kaoyouerziju{一九六三年十月十五日}

\section[接见阿尔巴尼亚总检察长等的谈话(一九六三年十一月十五日)]{接见阿尔巴尼亚总检察长等的谈话}
\datesubtitle{(一九六三年十一月十五日)}

\begin{duihua}

\item[\textbf{主席:}] 很欢迎同志们。你们来了几天了?

\item[\textbf{阿拉尼特·切拉(以下简称切拉):}] 十天了。

\item[\textbf{主席:}] 走北路来的,还是走南路来的?

\item[\textbf{切拉:}] 最近是从朝鲜来的,我们住朝鲜住了一个月,在朝鲜是休假。到朝鲜是从北路走的。

\item[\textbf{主席:}] 他们让你们过?

\item[\textbf{切拉:}] 让我们经过了,但对我们冷遇。

\item[\textbf{主席:}] 冷遇啊!请抽烟。(外宾说,不会抽),朝鲜的同志们很好,他们的工作做得很好。

\item[\textbf{切拉:}] 我们也是这样认为的。

\item[\textbf{主席:}] 这几个月在全世界,反对修正主义斗争更加发展了。你们坚决地站稳了立场,并且取得了胜利。你们的国家是被他们包围的。\marginpar{\footnotesize 62}你们对全世界真正的马克思列宁主义者是一个很大的鼓舞。

回去时,问候你们的领导同志们好,问候霍查同志、谢胡同志,还有其他同志。

\item[\textbf{切拉:}] 一定转达。

\item[\textbf{主席:}] 请喝点茶。除了问候霍查同志、谢胡同志,还有卡博同志、阿利雅、巴卢库等其他同志,也替我转达问候他们。

\item[\textbf{切拉:}] 一定转达。

\item[\textbf{主席:}] 你们今年的收成怎么样?

\item[\textbf{切拉:}] 我们今年的收成情况是:去年冬天雨下得多了,造成今年夏收不好;但今年春耕春种搞的好,所以今年秋收是好的。可以说今年的年成是个好的年成。

\item[\textbf{索弗克利·巴巴华西里}(以下简称巴巴华西里):] 今年的气候对我们的秋耕秋种是有利的。

\item[\textbf{主席:}] 很好。今年我们有点灾,一般说来是增产的。如果没有南边的旱灾和北边的水灾,那今年是个大丰收。去年比前年增产一千万吨。今年有好多社会主义国家的农业不好。

\item[\textbf{切拉:}] 我们来的时候,经过布达佩斯。听说那里的人民意见很大,有抱怨情绪。他们今年的收成不好,政府向美国买粮食。我们去的时候,还没有告诉人民,现在也许告诉了。

\item[\textbf{主席:}] 你们是否最近就要回国?

\item[\textbf{切拉:}] 现在预定二十六日离开中国。

\item[\textbf{主席:}] 今天是十五日,还要到外边去。

\item[\textbf{张××:}] 还要到上海、杭州、广州、昆明,从那里离开中国回去。

\item[\textbf{主席:}] 好,到那些地方去看看。

\item[\textbf{切拉:}] 我们将会看到很多东西。你们的同志对我们的帮助很大。

\item[\textbf{主席:}] 交换意见嘛。

\item[\textbf{切拉:}] 是帮助了我们。

\item[\textbf{主席:}] 互相交换经验。你们阿尔巴尼亚同志到中同来,中国同志都是很欢迎灼。

\item[\textbf{切拉:}] 我们具体地看到了。虽然在阿尔巴尼亚早已知道你们会欢迎我们的,到这里我们亲眼看到了。

\item[\textbf{主席:}] 你们是第一次来中国?

\item[\textbf{切拉:}] 是第一次。

\item[\textbf{主席:}] 你们两位都是做政法工作的吗?

\item[\textbf{切拉:}] 不,我是司法工作者,他(指巴巴华西里)是在党中央当视察员。

\item[\textbf{主席:}] 没有去东北看看?

\item[\textbf{切拉:}] 时间有限。我们在朝鲜停了一个月,余下的时间就不多了。现在想利用这些时间到中国南方看看。要到中国各个地方都走一趟,这是一件难事。

\item[\textbf{主席:}] 我刚才从南方回来。南方的秋收还没有完全结束,现在大概差不多了,广东可能还没有收完。你们这次到不到广东去?

\item[\textbf{黄火星:}] 要去广州。

\item[\textbf{主席:}] 对付反革命分子,对付贪污浪费分子,单是用行政的办法,法律的办法是不行的,要依靠群众的力量。检察院、法院和公安部门,同党的工作,同群众的工作配合起来,这样比较好一些。比如讲,铺张浪费、贪污分子,一般说靠行政是整不好的,他们就是怕群众。叫做上下夹攻,他们就无路可走了。\marginpar{\footnotesize 63}

要隔几年就整顿一次。即使不是一年一次,几年就要整一次。比如,一个机关,几十人、几百人的机关,过几年就会发生一些问题。我们国家仍然存在着相当严重的阶级斗争。我们过去十年没有抓这个问题了。从去年起,我们准备用几年的时间,把阶级斗争的问题和其他有关的问题抓一下,不然,就很不好搞。有旧的资产阶级残余存在,又产生新的资产阶级分子,就是做投机生意的,贪污的等等。这些人就是修正主义的社会基础,如果现在不整,再过十几年,中国会出修正主义。当然,他们的人数比较是少数,大概百分之几的样子。

我们主要不靠捉人、杀人,主要靠批评教育。但不是说一个也不捉,一个也不杀。

对罪大恶极的,罪恶很大,人民群众要求把他们捉起来,就非捉起来不可;有破坏行为,如杀人放火,破坏工厂、破坏桥梁等少数分子。就是那些普通的破坏分子,他们反对社会主义,比如讲,放谣言啊等等,不是严重的破坏分子都不捉,依靠群众来监督他们,在劳动中去改造。看来,这个方法可能是一个比较好的方法。我们的经验供你们参考,各国的情况不同,你们根据你们的实际情况办事,相信你们会做得更好。

司法工作是不容易做的。检察院,法院和公安部门都是专政的工具。

\item[\textbf{切拉:}] 毛泽东同志,我们同你们的同志淡了些问题,他们还把我们带到北京监狱去看了,他们向我们介绍了很多情况。我们感到你们教育人的工作做得很出色。

\item[\textbf{主席:}] 不是每个地方都做得好的。

\item[\textbf{切拉:}] 也可能你们的工作还有缺点,但基础是正确的。

主席;就是用教育的方法改造人。

\item[\textbf{巴巴华西里:}] 对于你们用的这种方法,感到受益不少。

\item[\textbf{主席:}] 第一条,我们要相信群众;第二条,就是这些反革命分子是劳动力。如果把他们捉起来,杀掉,他们的家庭和生产队就丧失了这些劳动力。第二条,对于他们的子女不好做工作,他们的子女要恨我们。所以,用教育的方法来改造,就可以避免了。我们相信依靠群众是可以把他们教育改造好的,他们又是一些劳动力,可以参加社会生产。这样又可以做好他们的子女和家属的工作,使他们不恨我们。

但是并不是每一个地方的工作都做得好。有那么一些同志性急,喜欢用简单的方法解决问题。动不动就把人抓起来或者要求把他杀掉。我们这些同志是把矛盾上交,从下面交到上面来。把矛盾上交的方法并不是一个奸的方法,上面不好处理,还不如放在群众中间,一面教育,一面让他们劳动好。至于有少数分子,你们不是看了北京监狱吗?那是要抓起来的,但也是采取教育的方法进行改造。

他们看了哪个监狱?

\item[\textbf{黄火星:}] 北京市的监狱。

\item[\textbf{主席:}] 那些人有工作做吗?

\item[\textbf{黄火星:}] 那里有塑料厂、鞋厂、袜厂等等。

\item[\textbf{主席:}] 他们学了技术,放出去以后好劳动。

\item[\textbf{切拉:}] 我们认为这样做是很对的。我们国家的劳改营里也有些劳动,但没有你们开展得这样广泛。我们监狱的工作是薄弱的,虽然,在我们监狱中关的人很少,是那些非常危险的分子。尽管这样,对这种人还是采取教育的方法。

\item[\textbf{主席:}] 对!我们第一要相信人是可以改造过来的,在一定的条件下,\marginpar{\footnotesize 64}在无产阶级专政的条件下,一般说是可以把人改造过来的。只有个别人改造不过来。那也不要紧,刑期满了放回去,有破坏活动就再捉回来。有的放出去一次,他照样破坏;放二次,他再破坏;放三次,他再要破坏。是有这样的人,那我们只好把他长期养下去,把他关在监狱的工厂工作。或者把他们家属也搬来,有些刑满了不愿意回去的就把家属也接来。

\item[\textbf{张××:}] 刑满了可以把家属搬来,安置就业。

\item[\textbf{主席:}] 对,就是安置就业。有些人是自己不愿意回去的,因为回到当地名誉不好,他在这里已经有很多熟人了,这样就可以把他的家属也搬来,等于迁居了。这样的也不少。

\item[\textbf{黄火星:}] 北京市那个监狱,也有就业的,有四百多人。

\item[\textbf{主席:}] 我们把一个皇帝也改造得差不多了。

\item[\textbf{切拉:}] 我们听说过,他叫溥仪。

\item[\textbf{主席:}] 我在这里见过他。他现在有五十几岁了,他现在有职业了,听说还重新结了婚。

\item[\textbf{切拉:}] 听说他还写了本书,叫《我的前半生》。

\item[\textbf{主席:}] 现在这本书还没有公开发行。我们觉得他这木书写得不怎么好。他把自己说得太坏了,好像一切责任都是他的。其实,应当说这是一种社会制度下的一种情况。在那样的旧的社会制度下产生这样一个皇帝,那是合乎情理的。不过对这个人,我们也还要看。

谈到这里好不好。现在让我们照相吧!
\end{duihua}

\input{5-044-1963.11.16-给林彪、荣臻等同志的信(一九六三年十一月十六日).tex}
\input{5-045-1963.11.26-接见古巴诗人、作家和艺术家联合会文学部主任比达·罗德里格斯夫妇的谈话(一九六三年十一月二十六日).tex}
\input{5-046-1963.12.12-对柯庆施同志有关上海曲艺革命化改革总结报告的批示(一九六三年十二月十二日).tex}
\input{5-047-1963.12.13-关于加强相互学习,克服故步自封骄傲自满的指示(一九六三年十二月十三日).tex}
\input{5-048-1963.12.13-谦虚——戒骄(一九六三年十二月十三日).tex}
\input{5-049-1963.12.14-给林彪同志的信(一九六三年十二月十四日).tex}
\input{5-050-1963-论反对官僚主义(一九六三年).tex}
\input{5-051-1963.12-在聂荣臻同志汇报时的谈话(一九六三年十二月).tex}
\section[在中南海元旦联欢会上的讲话(一九六四年一月一日)]{在中南海元旦联欢会上的讲话}
\datesubtitle{(一九六四年一月一日)}

马列主义原来是外国的,和中国革命结合了,我们就创造性地发展了它,也就成为我们自己的了。

\input{5-053-1964.1.8-同×××谈人民日报要学习解放军(一九六四年一月八日).tex}
\input{5-054-1964.1.9-慰问恩克鲁玛总统的信(一九六四年一月九日).tex}
\input{5-055-1964.1.12-就巴拿马人民反对美帝国主义的爱国斗争对《人民日报》记者发表的谈话(一九六四年一月十二日).tex}
\input{5-056-1964.1-谈报纸革命化问题(一九六四年一月).tex}
\input{5-057-1964.1.27-就最近日本人民反对美帝国主义的爱国正义斗争发表谈话(一九六四年一月二十七日).tex}
\input{5-058-1964.1.28-接见阿尔及利亚民族解放阵线代表和法律工作者代表团的谈话(一九六四年一月二十八日).tex}
\input{5-059-1964.1-几段插话(一九六四年一月).tex}
\section[对《人民日报加强学术文章的报告》的批示(一九六四年二月三日)]{对《人民日报加强学术文章的报告》的批示}
\datesubtitle{(一九六四年二月三日)}


\noindent ××、××同志:

《人民日报》历来不重视思想理论工作,哲学社会科学文章很少,把这个理论阵地送给《光明日报》、《文汇报》和《新建设》月刊。这种情况必须改变过来才好。现在他们有了改的主意了,请书记处讨论一下,并给他们解决干部问题为盼。


\section[对《中央关于传达石油工业部关于大庆石油会战情况通知》的批示(一九六四年二月五日)]{对《中央关于传达石油工业部关于大庆石油会战情况通知》的批示}
\datesubtitle{(一九六四年二月五日)}


大庆油田的经验虽然有其特殊性,但是具有普遍意义,他们贯彻执行了党的社会主义建设总路线,坚持政治挂帅,坚持群众路线,系统地学习和运用解放军的政治工作经验,把政治思想、革命干劲和科学管理紧密结合起来,把工作做活了。这是一个多、快、好、省的典型。它的一些主要经验,不仅在工业部门中适用,在交通、财贸、文教各部门,在党、政、军、群众团体的各级机关中也都适用或者可做参考。\marginpar{\footnotesize 88}


\input{5-062-1964.2.9-接见新西兰共产党总书记威尔科克斯夫妇时的谈话(一九六四年二月九日).tex}
\section[关于出版三十本马、恩、列、斯著作的指示(一九六四年二月十三日)]{关于出版三十本马、恩、列、斯著作的指示}
\datesubtitle{(一九六四年二月十三日)}


\noindent ×××同志:

一、此件看过,很好,可以立即发下去。

二、三十本书,大字线装,分册(一部大书分十册、八册,小书不分册,中书仍要分册),请你督促迅速办下去。希望今年办成,可以吗?你想一下告我为盼。每部印一万、两万分,好吗?我急于想要看这种大字书。


\input{5-064-1964.2.13-春节谈话记要(一九六四年二月十三日).tex}
\section[送给李讷的四句话(时间不详)]{送给李讷的四句话}
\datesubtitle{(时间不详)}

1. 天将降大任于斯人也,必先苦其心志,劳其筋骨,饿其体肤,空乏其身,行拂乱其所为,所以动心忍性,增益其所不能。

2. 彻底的唯物主义者是无所畏惧的。

3. 道路是曲折的,前途是光明的。

4. 在命运的前头痛击下头破血流但仍不回头。


\input{5-066-1964.2-与毛远新同志的谈话纪要(一)(一九六四年二月).tex}
\section[关于胡藏芸案件的指示(一九六四年二月)]{关于胡藏芸案件的指示}
\datesubtitle{(一九六四年二月)}


最近两个下放干部来我这里,谈到北京市公安局五处化工厂思想政治工作做得不活,领导者水平不高,据说有一个犯人经过教育以后,坦白了全部问题,结果加重了刑期,这样做就有顾虑了。不坦白反而可以早出去,坦白了却加重了刑期,此事如果属实,就奇怪了。坦白应该从宽。他不骗你了,应该从宽嘛!是不是有这样的事,谢富治同志或徐××,可以去这个厂子了解了解。这样的工厂很重要,应有一个知识水平较高的人去领导。

\kaoyouerziju{(1964年2月,对胡藏芸案件的指示,汪东兴同志传达)}


\section[关于学校课程和讲授、考试方法问题的批示(在北京铁路二中调查材料的批示)(一九六四年三月十日)]{关于学校课程和讲授、考试方法问题的批示}
\datesubtitle{(在北京铁路二中调查材料的批示)\\(一九六四年三月十日)}


此件应发给中央宣传部各正副部长,中央教育部各正副部长、司局长每人一份,北京市委、市人委负责人及管教育的同志每人一份,固中央三份。并请他们加以调查研究。现在学校课程太多,对学生压力太大。讲授又不甚得法。考试方法以学生为敌人,举行突然袭击。这三项都是不利于培养青年们在德智体诸方面生动活泼地主动地得到发展的。


\input{5-069-1964.3.22-人民日报要搞理论工作(一九六四年三月二十二日).tex}
\section[对新华社的指示(一九六四年春)]{对新华社的指示(一九六四年春)}
\datesubtitle{(一九六四年)}


你们新华社只有几千人,太少了,恐怕是世界通迅社中最小的吧!可以办得更大一些。


\section[给华罗庚的信(一九六四年三月十八日)]{给华罗庚的信}
\datesubtitle{(一九六四年三月十八日)}


华罗庚先生:

诗和信已经收读,壮志凌云,可喜可贺。肃此。敬颂教祺!

\kaitiqianming{毛泽东}
\kaoyouerziju{一九六四年三月十八日}


\section[给高×的信(一九六四年三月十八日)]{给高×的信}
\datesubtitle{(一九六四年三月十八日)}


高×先生:

寄书寄词,还有二信,均己收到,极为感谢,高文两册,我很爱读,肃此,敬颂吉安。

\kaitiqianming{毛泽东}
\kaoyouerziju{一九六四年三月十八日}
\marginpar{\footnotesize 98}

\input{5-073-1964.3.28-在邯郸四清工作座谈会上的讲话(一九六四年三月二十八日).tex}
\input{5-074-1964.4.3-在一次汇报时的插话(一九六四年三月).tex}
\input{5-075-1964.4.15-接见阿尔及利亚文化代表团时的谈话(一九六四年四月十五日).tex}
\section[对人民日报理论工作的批评(一九六四年四月)]{对人民日报理论工作的批评}
\datesubtitle{(一九六四年四月)}


《人民日报》的理论工作是应付我的,谁管理论版?(答:陈浚)让王若水管好了,他的文章多吗?


\section[关于胡藏芸案件的指示和批示(一九六四年四月)]{关于胡藏芸案件的指示和批示}
\datesubtitle{(一九六四年四月)}


你看,确有此事吧!有些人只爱物,不爱人,只重生产,不重改造。把犯人当成劳役,只有压服不行。其实抓紧思想政治工作,以思想工作第一,作好这一面,不仅不会妨碍生产,相反还会促进生产。

\kaoyouerziju{ (1964年4月,同汪东兴同志的谈话)}

人是可以改造的,就是政策和方法要正确才行。

\youxiadingkuan{ (1964年4月20日,对公安部党组关于调查处理胡藏芸案件的情况报告的批示)}
\marginpar{\footnotesize 109}

\input{5-078-1964.4.28-在谢富治同志汇报劳改工作时的指示(一九六四年四月二十八日).tex}
\section[对共青团九大的指示(一九六四年)]{对共青团九大的指示}
\datesubtitle{(一九六四年)}


有为青年多得很,青年一代要打败老一代,我们的未来就是他们的,不要为名望、知识所惧怕,青年人要敢想、敢说、敢做、要从各种狭隘的限制中解放出来。


\input{5-080-1964.5.11-在计委领导小组汇报第三个五年计划时的一些插话(一九六四年五月十一日).tex}
\input{5-081-1964.5.15-接见阿尔巴尼亚妇女代表团和电影工作者的谈话(一九六四年五月十五日).tex}
\input{5-082-1964.5-在四个副总理汇报时的插话(一九六四年五月).tex}
\input{5-083-1964.6.6-关于第三个五年计划在中央工作会议上的讲话(一九六四年六月六日).tex}
\input{5-084-1964.6.8-和×××的谈话(一九六四年六月八日).tex}
\input{5-085-1964.6.16-关于军事工作落实与培养革命接班人的讲话(一九六四年六月十六日于十三陵).tex}
\input{5-086-1964.6-对人民日报文艺宣传的批评(一九六四年六月中旬).tex}
\input{5-087-1964.6.18-接见桑给巴尔专家米·姆·阿里夫妇的谈话(一九六四年六月十八日).tex}
\input{5-088-1964.6.23-接见智利新闻工作者代表团的谈话(一九六四年六月二十三日).tex}
\input{5-089-1964.6.24-接见外宾关于保健的一段谈话(一九六四年六月二十四日).tex}
\input{5-090-1964.6.24-和王海蓉同志的谈话(一九六四年六月二十四日).tex}
\section[对全国文联和所属各协会整风的批示(一九六四年六月二十七日)]{对全国文联和所属各协会整风的批示}
\datesubtitle{(一九六四年六月二十七日)}


这些协会和他们所掌握的刊物的大多数(据说有少数几个好的),十五年来,基本上(不是一切人)不执行党的政策,做官当老爷,不去接近工农兵,不去反映社会主义革命和建设。最近几年,竟然跌到了修正主义的边缘。如不认真改造,势必在将来的某一天,要变成像匈牙利裴多菲俱乐部那样的团体。


\section[畅游十三陵水库时对青年的谈话(一九六四年六月)]{畅游十三陵水库时对青年的谈话}
\datesubtitle{(一九六四年六月)}


毛主席问大家:“在七级大风里,你们游过吗?”\marginpar{\footnotesize 136}

“一人高的浪,你们游过吗?”

“游泳是同大自然作斗争的一种运动,你们应该到大江大海去锻炼。”

“这很好,部队要学会游泳。”

“会休息,才会持久。”

“时代不同了,男女都一样,男同志能办到的事情,女同志也能办得到。”

“将来打仗还是要靠两条腿。”

“有许多地方通不过去,看起来只有两只脚有用。”


\input{5-093-1964-在小型会议上的讲话(一九六四年).tex}
\section[在中央工作会议小型会议上的指示(一九六四年)]{在中央工作会议小型会议上的指示}
\datesubtitle{(一九六四年)}


向来讲话不要鼓掌,要允许人们打瞌睡,养神。我过去读书就看小说,老师一查,就把课本放在小说上。这也许是我的毛病,也许是先生不对,不如看小说。后来就发明可以打瞌睡。你不要讲我这个人没有创造,还是有一些。这样一来,就整了那些不是交谈式而是训话式的人。有了讲义就不要再讲了。


\section[关于部队开展游泳活动的指示(一九六四年七月二日)]{关于部队开展游泳活动的指示}
\datesubtitle{(一九六四年七月二日)}


部队要学游泳,所有部队都要学会。学游泳有个规律,摸到了规律就容易学会。整营、整团要学会全付武装泅渡。每团先搞一个营,每师先搞一个团,然后做到全会。\marginpar{\footnotesize 137}


\section[对一渡河支部提拔新生力量报导的指示(一九六四年七月四日)]{对一渡河支部提拔新生力量报导的指示}
\datesubtitle{(一九六四年七月四日)}


主席看了七月四日人民日报二版关于一渡河党支部培养青年干部的报导,当日作如下指示:

×××同志,怀柔县一渡河支部提拔新生力量的做法,各省可能都有,要广泛采访,转载,在几年之内做到每县每社每个工厂、学校、机关都有报导。但要是真实的,典型的。固步自封的反面材料,也要登一点。这个问题,报社和通讯社应当讨论一下。并与各省、市、区联系,要他们也一样做。
\kaitiqianming{毛泽东}
\kaoyouerziju{七月四日}


\input{5-097-1964.7.10-接见日本社会党人士佐佐木更三、黑田寿男、细迫兼光等的谈话(一九六四年七月十日).tex}
\input{5-098-1964.7.14-无产阶级专政的历史教训(一九六四年七月十四日).tex}
\input{5-099-1964.7-对汪东兴同志报告的批示(一九六四年七月).tex}
\section[观看北京京剧一团演出的革命现代京剧《沙家浜》的指示(一九六四年七月)]{观看北京京剧一团演出的革命现代京剧《沙家浜》的指示}
\datesubtitle{(一九六四年七月)}


要突出武装斗争,强调武装的革命消灭武装的反革命,戏的结尾要正面打进去。加强军民关系的戏,加强正面人物的音乐形象。


\section[关于部队学习游泳的批示(一九六四年八月六日)]{关于部队学习游泳的批示}
\datesubtitle{(一九六四年八月六日)}


此件看了,很好。是否在一切有条件的地方,部队的大多数人都可以试验学游泳?军委是否已发出了指示?

条件不好,主要是:(一)有吸血虫及其他毒害的河流、池塘;(二)有大漩涡的河流地段;(三)有鲨鱼的海中。此外,部队中总有一部分不适宜于游水的,不要强令人人都下水。


\section[对卫生部党组关于高级干部保健工作报告的批示(一九六四年八月)]{对卫生部党组关于高级干部保健工作报告的批示}
\datesubtitle{(一九六四年八月)}


“保健局应当取消。”

“北京医院医生多,病人少,是一个老爷医院,应当开放。”

\section[对《中央宣传部关于公开放映和批判<北国之春>、<早春二月>的请示报告》的批示(一九六四年八月)]{对《中央宣传部关于公开放映和批判<北国之春>、<早春二月>的请示报告》的批示}
\datesubtitle{(一九六四年八月)}


可能不只这两部影片,还有别的都需要批判。使修正主义材料公布于众。\marginpar{\footnotesize 147} 


\input{5-104-1964.8.18-关于哲学问题的讲话(一九六四年八月十八日).tex}
\input{5-105-1964.8.20-同李雪峰等同志的谈话(一九六四年八月二十日).tex}
\input{5-106-1964.8.22-接见出席第十届禁止原子弹氢弹世界大会后访华外宾的谈话(一九六四年八月二十二日).tex}
\input{5-107-1964.8.24-关于坂田文章的谈话(一九六四年八月二十四日).tex}
\input{5-108-1964.8-同毛远新同志的第二次谈话(一九六四年八月).tex}
\input{5-109-1964.8.25-接见非洲和拉丁美洲青年学生代表团时的谈话(一九六四年八月二十五日).tex}
\input{5-110-1964.8.29-接见尼泊尔教育代表团时的谈话(一九六四年八月二十九日).tex}
\input{5-111-1964.8-关于团结方法的讲话(一九六四年八月).tex}
\input{5-112-1964.9.4-在×××反修报告会上的插话(一九六四年九月四日).tex}
\input{5-113-1964.9.4-接见老挝爱国战钱党文工团团长、副团长和主要成员时的谈话(一九六四年九月四日武汉).tex}
\input{5-114-1964.9.10-接见法国技术展览会负责人及法国驻华大使的谈话(摘录)(一九六四年九月十日).tex}
\section[与计划领导小组的谈话(一九六四年九月十二日)]{与计划领导小组的谈话}
\datesubtitle{(一九六四年九月十二日)}


工作方法,这是思想问题。来的计划是一本书。问题是思想从哪里来的,始终没有搞清楚。计划是从哪里来的?我看是俄国的书本,再加上自己脑筋的“想当然”。我看要上谋中央、下谋群众,还要谋前后左右。


\input{5-116-1964.9.20-接见阿尔及利亚政府经济代表团时的讲话(一九六四年九月二十日).tex}
\section[对《中央音乐学院的意见》的批示(一九六四年九月二十七日)]{对《中央音乐学院的意见》的批示}
\datesubtitle{(一九六四年九月二十七日)}


××同志:

此件请一阅。信是写得好的,问题是应该解决的,应采取征求群众意见的方法,在教师、学生中先行讨论,收集意见。古为今用,洋为中用。

此信表示一派人的意见,可能有许多人不赞成。
\kaitiqianming{毛泽东}
\kaoyouerziju{九月二十七日}


\section[观看革命现代芭蕾舞剧《红色娘子军》后的指示(一九六四年十月八日)]{观看革命现代芭蕾舞剧《红色娘子军》后的指示}
\datesubtitle{(一九六四年十月八日)}


《红色娘子军》方向是对的,革命是成功的,艺术上也是好的。


\input{5-119-1964.10-接见西南非妇女代表团的谈话(摘录)(一九六四年十月).tex}
\input{5-120-1964.10.16-会见古巴党政代表团的谈话纪录(摘录)(一九六四年十月十六日).tex}
\input{5-121-1964.12-关于总结经验的指示(一九六四年十二月).tex}
\section[对《关于×××、×××、×××问题》的批示(一九六四年十二月十二日)]{对《关于×××、×××、×××问题》的批示}
\datesubtitle{(一九六四年十二月十二日)}


此件已阅,写的可以,是好的。但有骨头,无血肉,感到枯燥无味,则是缺点。望你们在今后几个月内,搞出一个有骨有血有皮有毛的东西出来。要有逻辑有论证,否则仍然是形而上学的东西。十几年来,形而上学盛行,唯物辩证法很少人理,现在是改变的时候了。\marginpar{\footnotesize 187}


\input{5-123-1964.12.15-在中央工作会议上的插话(一九六四年十二月十五日下午).tex}
\input{5-124-1964.12.19-对高扬文同志的蹲点报告的批语(一九六四年十二月十九日).tex}
\input{5-125-1964.12.20-在中央工作座谈会上关于四清问题的讲话(一九六四年十二月二十日).tex}
\section[人民日报现在有看头了(一九六四年十二月二十日)]{人民日报现在有看头了}
\datesubtitle{(一九六四年十二月二十日)}


现在人民日报有看头了。理论上加强了,也有一些有意思的东西。今天(二十日)二版关于设计讨论的四篇小文章\footnote{十一月二十日《关于“用革命精神改造设计工作”的讨沦》第六期登出的四篇短文章,是《带着党政策下现场》、《走出个人主义的小圈子》、《放下施工指导的架子,虚心向工人学习》、《不能把工厂看成静止不变的东西》。编后的二百五十字,题目是《根本问题在哪里?》}全看了,编者按也写得好,大白菜\footnote{大白菜报导,指一九六四年十二月五日一版《卖菜札记》和评《领导还是被领导?》}也上了头条,很好。要继续努力。解放军报、中国青年报有些短的,生动活泼的、思想性强的内容,要学习。\marginpar{\footnotesize 198}


\input{5-127-1964.12.27-在中央工作会议上陈伯达同志发言时的插话(一九六四年十二月二十七日).tex}
\input{5-128-1964.12.28-在中央工作会议上的讲话(一九六四年十二月二十八日).tex}
\section[同江西省委同志谈话记录(摘录)(一九六四年)]{同江西省委同志谈话记录(摘录)}
\datesubtitle{(一九六四年)}


湖南还提出农村要建立儿童团,少先队,青年联合会,因为现在农村有不少高小程度的小知识分子。

(少先队)为什么要按学校组织呢?还是按行政村组织好,还可以把地富儿童组织起来,使他们受教育。
\marginpar{\footnotesize 201}


\input{5-130-1964.12-中国的大跃进(一九六四年十二月).tex}
\section[社会主义社会阶级斗争并没有熄灭(一九六四年)]{社会主义社会阶级斗争并没有熄灭}
\datesubtitle{(一九六四年)}


社会主义社会是一个很长的历史阶段。在社会主义社会里,在实现了工业国有化和农业集体化,完成了生产资料所有制的社会主义改造以后,阶级矛盾仍然存在,阶级斗争并没有熄灭。在这个历史阶段中,必须在经济战线上、政治战线上和思想文化战线上进行彻底的社会主义革命。同时,只要世界上还有帝国主义、资本主义、各国反动派和现代修正主义存在,资本主义的阴风总会不时地吹到社会主义国家里来。因此,在社会主义国家中,社会主义同资本主义之间谁胜谁负的斗争,需要一个很长的时间,才能最后解决。

\kaoyouerziju{ 转引自周总理在人大三届一次会议上的政府工作报告,《人民日报》一九六五年一月一日}\marginpar{\footnotesize 202}


\input{5-132-1964.12.25-同参加“亚非文学交流座谈会”的亚非作家的谈话(摘录)(一九六四年十二月二十五日).tex}
\input{5-133-关于划阶级问题的指示(时间不详).tex}
\input{5-134-关于依靠贫下中农的问题(时间不详).tex}
\input{5-135-1964-《前十条》和《六十条》为什么能调动人的力量?(一九六四年).tex}
\input{5-136-蹲点问题(一九六四).tex}
\input{5-137-论实事求是 .tex}
\input{5-138-小资产阶级的通病 .tex}
\section[《关于学习解放军加强政治工作的指示》的批示(一九六四年)]{《关于学习解放军加强政治工作的指示》的批示}
\datesubtitle{(一九六四年)}


现在全国学习解放军、学大庆,学校也要学习解放军。解放军好,是政治思想好,也要向全国城市、农业、工业、商业、教育的先进单位学习。

国家工业各部门,现在有人从上至下,即从部到厂矿都学习解放军。设政治部,政治处和政治指导员,实行四个第一和三八作风——看来不这样做是不行的,是不能提起整个工业部门(还有商业部门,还有农业部门)成百万成千万的干部和工人的革命精神的。\marginpar{\footnotesize 208}


\input{5-140-1965.1.3-关于四清运动的一次讲话(一九六五年一月三日).tex}
\input{5-141-1965.1.9-和美国记者斯诺的谈话(一九六五年一月九日).tex}
\section[对徐××同志《关于如何打乒乓球》一文的批示(一九六五年一月十二日)]{对徐××同志《关于如何打乒乓球》一文的批示}
\datesubtitle{(一九六五年一月十二日)}


徐××同志的讲话和××同志的批语,印发中央同志们一阅,并希望你们回去后,再加印发,以广宣传。同志们,这是小将们向我们这一大批老将们挑战了,难道我们不应该向他们学习一点什么东西吗?讲话全文充满了辩证唯物论,处处反对唯心主义和任何一种形而上学。多年以来没有看到这样好的作品,他讲的是打球,我们要从他那里学习的是理论、政治、经济、文化、军事。如果我们不向小将们学习,我们就要完蛋了。


\input{5-143-1965.1.14-中央《关于学习周恩来同志政府工作报告的通知》(摘引)(一九六五年一月十四日).tex}
\input{5-144-1965.1.27-接见苏班德里约的谈话(一九六五年一月二十七日).tex}
\input{5-145-1965.1.29-对《陈××同志蹲点报告》的批示(一九六五年一月二十九日).tex}
\input{5-146-1965.1-听取谷牧、余秋里汇报计划工作时的指示(一九六五年一月).tex}
\input{5-147-1965.2.11-听取××汇报工业交通会议情况时的指示(一九六五年二月十一日).tex}
\section[听取××同志汇报会上的指示(一九六五年二月二十一日上午)]{听取××同志汇报会上的指示}
\datesubtitle{(一九六五年二月二十一日)}


我们现在有些东西保密得太过分了。技术上很落后,你给人家人家不要,这个问题也要一分为二。一、有的非保密不可,如……要保密;二、有些加工技术,有什么密可保!?根本不要保密。


\input{5-149-1965.2.22-接见海军干部工作会议、《解放军报》编辑记者会议和第三批战士演出队时的指示.tex}
\section[正面教员与反面教员(一九六五年二月)]{正面教员与反面教员}
\datesubtitle{(一九六五年二月)}

革命的政党,革命的人民,总是要反复地经受正反两个方面的教育,经过比较和对照,才能够锻炼得成熟起来,才有赢得胜利的保证。我们的中国共产党人,有正而教员,这就是马克思、恩格斯、列宁、斯大林。也有反面教员,这就是蒋介石、日本帝国主义者、美帝国主义者和我们党内犯“左”倾或右倾机会主义路线错误的人。如果只有正面教员而没有反面教员,中国革命是不会取得胜利的。轻视反面教员的作用,就不是一个彻底的辩证唯物主义者。

\kaoyouerziju{ (摘自《赫鲁晓夫言论集》第三集的出版说明)}

\input{5-151-1965.3-接见巴勒斯坦解放组织代表团时的谈话(一九六五年三月).tex}
\input{5-152-1965.4.21-接见中南局第九次全体会议同志时的指示(一九六五年四月二十一日).tex}
\section[关于劳改工作的指示(一九六五年四月)]{关于劳改工作的指示}
\datesubtitle{(一九六五年四月)}


劳改犯办了很多事情,要用,在一定条件下,可以为我们所用。我们没有几个贪污分子不行,这样大的国家没有几个反革命分子不行。

他们可有才能!没有才能,反革命干什么?在一定条件下他能做很多事,凡有功的,可以摘掉帽子,有的还可以奖励。二十三条为什么规定这么一条:工作队的成员不一定要十分干净,有坏人不要怕,有些人政治上不好,但很有才干,改造嘛!改造不了也不要紧,只要把他们引到正路,就很有用,我们不会贪污,不懂贪污,也不懂技术,他们懂得这些,只要我们知道他们,政治挂帅,可以让他们办很多事,可以发挥他们的作用。就是要调动他们,他们中间可有才能,他发明真空电炉,是高级产品,我们不行,他行。

\kaoyouerziju{ (1965年4月,关于王灿文发明真空冶炼电炉的指示)}

改造要抓紧,不要在经济上做文章,不要想在劳改犯人身上搞多少钱,要抓改造,让他们寄点钱回家。

第一是思想改造,第二是生产,劳改工作干部不能太弱,要训练。训练两个星期就行了。

\kaoyouerziju{ (1965年4月,在十四次公安会议期间对劳改工作的指示)}


\input{5-154-1965.6.26-对卫生工作的指示(一九六五年六月二十六日).tex}
\section[对北京师范学院调查材料的批示(一九六五年七月三日)]{对北京师范学院调查材料的批示}
\datesubtitle{(一九六五年七月三日)}


学生负担过重,影响健康,学了也无用。建议从一切活动总量中砍掉三分之一。邀请学校师生代表,讨论几次,决定执行。如何请酌。


\section[对军队文工团到农村、工厂去锻炼的指示(一九六五年七月十五日)]{对军队文工团到农村、工厂去锻炼的指示}
\datesubtitle{(一九六五年七月十五日)}


军队文工团只知道军队的事不够,还应知道工人、农民的事。因此也应同地方文工团一样,要到工厂、农村去,特别要到农村去锻炼,去参加四清。


\section[关于模特儿问题的批示(一九六五年七月十八日)]{关于模特儿问题的批示}
\datesubtitle{(一九六五年七月十八日)}


男女老少裸体模特儿是绘画和雕塑必须的基本功,不要不行。封建思想加以禁止是不妥的。即使有些坏事出现,也不要紧,为了艺术科学,不惜少有牺牲。\marginpar{\footnotesize 231}

齐白石、陈半丁之流,就花木而论,还不如清末某些画家。

中国画家,就我见过的,只有一个徐悲鸿留了人体素描,其余如齐白石、陈半丁之流,没有一个人能画人物的。徐悲鸿学过西洋画法,此外还有一个刘海粟。


\section[给章士钊的信(一九六五年七月十八日)]{给章士钊的信}
\datesubtitle{(一九六五年七月十八日)}


\noindent 行严先生:

各信及指要下都已收到,已经读过一遍,还想读一遍。七部也还想再读一遍,另有友人也想读,大问题是唯物史观问题,即主要是阶级斗争问题。

但此事不能求之于世界观已经固定之老先生们,故不必改动,嗣后历史学者可能批评你这一点,请你要有精神准备,不怕人家批评。又高先生评郭文已看过,他的论点是地下不可能掘出真、行、草墓石。草书不会书碑,可以断言,至于真、行是否曾经书碑,尚待地下发掘证实,但争论是应该有的。我当劝说郭老、康生、伯达诸同志赞成高二适一文公诸于世。柳文上部盼急寄来。敬礼

康吉!

\kaitiqianming{毛泽东}
\kaoyouerziju{一九六五年七月十八日}



\input{5-159-1965.7.19-对医务人员的谈话(一九六五年七月十九日).tex}
\input{5-160-1965.7.20-在全国工交系统四清工作座谈会上的讲话(一九六五年七月二十日).tex}
\section[给华罗庚的信(一九六五年七月二十一日)]{给华罗庚的信}
\datesubtitle{(一九六五年七月二十一日)}


\noindent 华罗庚同志:

来信及平话,早在外地收到,你现在奋发有为,不为个人而为人民服务,十分欢迎。听说你在西南视察并讲学,有所收获,极为庆幸。专此奉复,敬颂教安。
\kaitiqianming{毛泽东}
\kaoyouerziju{一九六五年七月二十一日}


\section[接见李宗仁先生及其夫人的谈话(一九六五年七月二十七日)]{接见李宗仁先生及其夫人的谈话}
\datesubtitle{(一九六五年七月二十七日)}


毛主席同李宗仁先生及其夫人热烈握手。

毛主席说:你们回来,很好,欢迎你们。\marginpar{\footnotesize 233}

李宗仁先生说:我们回来后都为祖国的强大感到高兴。

毛主席说:祖国是比过去强大了一些,但还不很强大,我们至少还得再建设二三十年才能真正强大起来。

李宗仁先生对毛主席说:在海外的许多人士都怀念祖国,他们渴望回到祖国来。

毛主席说:跑到海外的,凡是愿意回来,我们都欢迎。他们回来我们都以礼相待。毛主席建议李宗仁先生到全国各地去看看。

今天同时被接见的还有陈思远先生。

接见后,毛泽东主席和江青同志设宴招待李宗仁先生和夫人郭德洁女士,以及陈思云先生。


\input{5-163-1965.8.2-在钱××、张×汇报卫生工作时的插话指示(一九六五年八月二日).tex}
\input{5-164-1965.8.3-接见法国事务部长马尔罗时的谈话(一九六五年八月三日).tex}
\input{5-165-1965.8.8-接见几内亚教育代表团、总检察长时的谈话(一九六五年八月八日).tex}
\section[关于民族工作的一次指示(一九六五年九月)]{关于民族工作的一次指示}
\datesubtitle{(一九六五年九月)}


要彻底解决民族问题,完全孤立民族反动派,没有大批从少数民族出身的共产主义干部,是不可能的。


\input{5-167-1965.9.14-同黑非洲留法学生联合代表团的谈话(摘录)(一九六五年九月十四日).tex}
\input{5-168-1965.9.18-接见阿尔巴尼亚内务部代表团时谈劳改工作(一九六五年九月十八日).tex}
\input{5-169-1965.10-关于大学文科改革的指示(摘录)(一九六五年十月).tex}
\section[路过济南在火车上听取汇报时的指示(一九六五年十一月)]{路过济南在火车上听取汇报时的指示}
\datesubtitle{(一九六五年十一月)}


(一九六五年十一月主席去华东路过济南,在火车上,有关领导向主席汇报工作)主席指示:你们有没有钢?(答,准备搞××吨,将来再发展到×××万吨)哦,搞×××吨,那好。

我对这一条比较积极,我是支持地方的。各省总要有个×万吨的钢铁厂,能制造机器,制造武器。我不怕你们造反,我自己也是造反的,造了多少次反,袁世凯当皇帝逼出了个蔡锷造反。如果中央出了军阀,出了修正主义,你们就可以造反。但是你们不能随便造反,不能造马列主义的反,否则你们就会吃亏,会成为修正主义。一个省也造不起反来。一个省搞点钢!搞×万吨左右的钢铁厂,一翻×万吨,再一翻××万吨,再一翻××万吨。但是要有条件。没有条件就不能炼钢。\marginpar{\footnotesize 243}


\section[在听取×××同志汇报时的插话(摘录)(一九六五年十一月)]{在听取×××同志汇报时的插话(摘录)}
\datesubtitle{(一九六五年十一月)}


(在汇报中谈到大学生一毕业就分配到机关,没有受过锻炼的时候)

就是这批人出现修正主义。

现在的教育要改革,一个小孩子要学习十六、七年,小学六年,中学六年,大学五年……

……要办抗大式的学校,……

……社教就是大学,……

现在学生马、牛、羊、鸡、犬、豕都不分,怎么不出修正主义?

初中学生一学期劳动一周太少了,小学也可以搞点轻微劳动,看起来还是搞半耕半读好。

课程负担重,为什么上那么多东西?我看要砍三分之一到二分之一。


\section[批判罗瑞卿(一九六五年十二月二日)]{批判罗瑞卿}
\datesubtitle{(一九六五年十二月二日)}


罗的思想同我们有距离,林彪同志带了几十年的兵,难道还不懂得什么是军事,什么是政治?军事训练几个月的兵就可以打仗,过去打的都是政治仗。要恢复林彪突出政治的原则。罗把林彪同志实际当作敌人对待,罗当总长以来,从未单独向我请示报告过工作,罗不尊重各元帅,他又犯了彭德怀的错误;罗在高、饶问题上实际上陷进去了,罗个人独断,罗是野心家。凡是搞阴谋的人,他总是拉几个人在一起。


\input{5-173-1965.12.2-反对折衷主义(一九六五年十二月二日).tex}
\input{5-174-1965.12.21-在杭州会议上的讲话(一九六五年十二月二十一日).tex}
\input{5-175-1965.12.21-在杭州与陈伯达、艾思奇等同志的谈话(一九六五年十二月二十一日).tex}
\section[对林彪同志一封来信的批语(一九六五年十二月二日)]{对林彪同志一封来信的批语}
\datesubtitle{(一九六五年十二月二日)}


那些不相信突出政治,对突出政治表示阳奉阴违,而自己另外散布一套折衷主义(即机会主义的)的人们,大家应当有所警惕。\marginpar{\footnotesize 249}


\input{5-177-1965.12.14-对机要保密工作的指示(一九六五年十二月十四日).tex}
\input{5-178-1965-关于礼宾工作指示(一九六五年).tex}
\input{5-179-1966.2.8-关于《海瑞罢官》(一九六六年二月八日).tex}
\input{5-180-1966.2.18-同毛远新同志的谈话(三)(一九六六年二月十八日).tex}
\input{5-181-1966.2-关于农业机械化问题与备战备荒为人民的指示信(一九六六年二月十九曰).tex}
\section[在修改《林彪同志委托江青同志召开的部队文艺工作座谈会纪要》时所加的话(一九六六年三月)]{在修改《林彪同志委托江青同志召开的部队文艺工作座谈会纪要》时所加的话}
\datesubtitle{(一九六六年三月)}


搞掉这条黑线之后,还会有将来的黑线,还得再斗争。所以,这是一场艰巨、复杂、长期的斗争,要经过几十年甚至几百年的努力。这是关系到我国革命前途的大事,也是关系到世界革命前途的大事。

过去十几年的教训是:我们抓迟了。只抓过一些个别的问题,没有全盘的系统的抓起来,而只要我们不抓,很多阵地也就只好听任黑线去占领,这是一条严重的教训。

文艺上反对外国修正主义的斗争,不能只捉丘赫拉依之类小人物,要捉大的,捉肖洛霍夫,要敢于碰他。他是修正主义文艺的鼻祖。

须知其他阶级的代表人物也是有他们的党性原则的,并且很顽强。


\section[看了“人工喉”、“断手再植”、“止血粉’等文章后对医务工作者的指示(一九六六年三月十二日)]{看了“人工喉”、“断手再植”、“止血粉’等文章后对医务工作者的指示}
\datesubtitle{(一九六六年三月十二日)}


应该加强医务人员的马列主义学习,并用以指导业务工作。既然军事上证明了所谓弱者可以打败强者,没有念过书或念过很少书的可以打败黄埔毕业生、陆军大学毕业生,医务界为什么是例外?医学校也要加强马列主义课程,好多毕业生就是不懂马列主义。血吸虫病的检查与治疗应该免费。过去医务人员就是不接近群众,不相信群众。消灭钉螺的办法还不是群众创造出来的?所以,我写的那首诗内说:“华佗无奈小虫何”。今后要在医务界大力系统的宣传马列主义。医务人员都要下去。\marginpar{\footnotesize 254}


\input{5-184-1966.3.20-在政治局扩大会议上的讲话(一九六六年三月二十曰华东).tex}
\section[在中央政治局常委扩大会议上的讲话(摘录)(一九六六年三月十七日至二十日)]{在中央政治局常委扩大会议上的讲话(摘录)}
\datesubtitle{(一九六六年三月十七日至二十日)}


我们在解放以后,对知识分子实行包下来的政策,有利也有弊。现在学术界和教育界是资产阶级知识分子掌握实权。社会主义革命越深入,他们就越抵抗,就越暴露出他们的反党反社会主义的面目。吴晗和翦伯赞等人是共产党员,也反共,实际上是国民党。\marginpar{\footnotesize 257}现在许多地方对于这个问题认识还很差,学术批判还没有开展起来。各地都要注意学校、报纸、刊物、出版社掌握在什么人手里,要对资产阶级的学术权威进行切实的批判。我们要培养自己的年青的学术权威。不要怕青年人犯“王法”,不要扣压他们的稿件。中宣部不要成为农村工作部。(注:中央农村工作部一九六二年被解散。)

《前线》也是吴晗、廖沬沙、邓拓的,是反党反社会主义的。

文、史、哲、法、经要搞文化大革命,要坚决批判,到底有多少马克思主义?


\input{5-186-1966.3.28-与康生等同志谈话纪要(一九六六年三月二十八日至三十日).tex}
\section[在政治局扩大会议期间的讲话(一九六六年四月十六日至二十五日)]{在政治局扩大会议期间的讲话}
\datesubtitle{(一九六六年四月十六日至二十五日)}


中国会不会出修正主义当权派的问题,一是出,一是不出;一是早出,一是晚出。还是早出好;搞好了可能不出,在中国出修正主义是困难的。

书记处也是分化的,彭、陆、杨、谭、罗等都当过书记处书记,是不断分化,合乎辩证法。有人怕得要死。不分化是主观愿望。中央有,各省也会有。


\section[在中央政治局常委扩大会议上的讲话(一九六六年四月二十二日)]{在中央政治局常委扩大会议上的讲话}
\datesubtitle{(一九六六年四月二十二日)}


我不相信,在文化革命中的问题只是吴晗问题,后面还要一串串“三家村”。文化革命是触及人们灵魂的革命,是意识形态的斗争,触及的很广泛,涉及面很宽。朝里有人,比如中央宣传都、中央文化部都发生这方面的问题,朝里都有人。各大区、各省市都有。

在党中央各部门,包括大区、名省市,朝里是否那么干净?我不相信。\marginpar{\footnotesize 258}


\section[批判彭真(对康生同志讲话)(一九六六年四月二十八日至二十九日)]{批判彭真}
\datesubtitle{(对康生同志讲话)\\(一九六六年四月二十八日至二十九日)}


北京一根针也插不进去,一滴水也滴不进去。彭真要按他的世界观改造党,事物是向他的反面发展的,他自己为自己准备了垮台的条件。这是必然的事,是从偶然中暴露出来的,一步一步深入的。历史教训并不是人人都引以为戒的。这是阶级斗争的规律,是不以人们的意志为转移的。凡是在中央有人搞鬼,我就号召地方起来攻他们,叫孙悟空大闹天宫,并要搞那些保“玉皇大帝”的人。彭真是混到党内的渺小人物,没有什么了不起,一个指头就通倒他。“西风落叶下长安”,告诉同志们不要无穷地忧虑。“灰尘不扫不走,阶级敌人不斗不倒。”

赞成鲁迅的意见,书不可不读,不可多读。不读人家会欺骗你。

现象是看得见的,本质是隐蔽的。本质也会通过现象表现出来。彭真的本质隐藏了三十年。

要不要告诉阿尔巴尼亚同志?没有什么不可告人的。


\input{5-190-1966.5.7-给林彪同志的信(对军委总后勤部“关于进一步搞好部队农副业生产的报告”的批示)(一九六六年五月七日).tex}
\input{5-191-1966.5.16-为中共中央“五·一六”通知所加的几段话(一九六六年五月十六日).tex}
\section[关于发表全国第一张马列主义大字报的批示(一九六六年六月一日)]{关于发表全国第一张马列主义大字报的批示}
\datesubtitle{(一九六六年六月一日)}


此文可由新华社发表,在全国各地报刊发表,十分必要。对北京大学这个反动堡垒。从此可以打破。

速办!

北大的这张大字报是马列主义的大字报,必须立刻广播,立即见报。

\kaoyouerziju{ (摘自康生同志九月八日的讲话)}


\section[在接见日本、古巴、巴西和阿根廷朋友时的谈话(一九六六年七月十日武汉)]{在接见日本、古巴、巴西和阿根廷朋友时的谈话}
\datesubtitle{(一九六六年七月十日 武汉)}


帝国主义最怕的是亚洲、非洲、拉丁美洲人民觉悟,怕世界各国人民的觉悟,我们要团结起来,把美帝国主义从亚洲、非洲、拉丁美洲赶回他的老家去。\marginpar{\footnotesize 261}


\section[关于北大“六·一八”事件(一九六六年七月)]{关于北大“六·一八”事件}
\datesubtitle{(一九六六年七月)}


六·一八”事件不是反革命事件,而是革命事件。


\section[畅游长江时对青年的指示(一九六六年七月十七日上午)]{畅游长江时对青年的指示}
\datesubtitle{(一九六六年七月十七日 上午)}


“人民万岁。”

“长江又宽,又深,是游泳的好地方。”

“长江水深流急,可以锻炼身体,可以锻炼意志。”

“三个人当中有一个人会游泳吗?(答有)这很好!”


\input{5-196-1966.7.21-重要讲话(一九六六年七月二十一日).tex}
\input{5-197-1966.7.22-在会见大区书记和中央文革小组成员的讲话(一九六六年七月二十二日).tex}
\input{5-198-1966.7-对中央首长的讲话(一九六六年七月).tex}
\section[对共青团中央的批评(一九六六年七月)]{对共青团中央的批评}
\datesubtitle{(一九六六年七月)}


有人讲团中央“三胡”糊里糊涂。明明白白站在资产阶级方面,什么糊里糊涂?团中央不但不支持学生群众运动,反而镇压学生群众运动,应严格处理。


\section[关于打人问题(一九六六年八月一日)]{关于打人问题}
\datesubtitle{(一九六六年八月一日)}


党的政策不主张打人。但打人也要进行阶级分析,好人打坏人活该;坏人打好人,好人光荣;好人打好人误会。今后再不许打人。要摆事实讲道理。\marginpar{\footnotesize 265}


\input{5-201-1966.8.1-给清华附中红卫兵小将的一封信(一九六六年八月一日).tex}
\input{5-202-1966.8.4-在中央常委扩大会议上的插话(一九六六年八月四日下午).tex}
\input{5-203-1966.8.5-炮打司令部——我的一张大字报(一九六六年八月五日).tex}
\section[在修改《欢呼北大的一张大字报》一文时加的一段话(一九六六年八月五日)]{在修改《欢呼北大的一张大字报》一文时加的一段话}
\datesubtitle{(一九六六年八月五日)}


危害革命的错误领导,不应当无条件接受,而应当坚决抵制。在这次文化大革命中,广大革命师生及革命干部对于错误的领导,就广泛地进行过抵制。


\section[在中共中央群众接待站的讲话(一九六六年八月十日)]{在中共中央群众接待站的讲话}
\datesubtitle{(一九六六年八月十日)}


你们要关心国家大事,要把无产阶级文化大革命进行到底!


\section[对杨寅田大字报的批示(一九六六年八月)]{对杨寅田大字报的批示}
\datesubtitle{(一九六六年八月)}


印发全会(指十一中全会)各同志。将杨寅田大字报题目用特号字,全文用老五号刊出。\marginpar{\footnotesize 267}


\input{5-207-1966.8.12-在八届十一中全会闭幕式上的讲话(一九六六年八月十二日).tex}
\section[在第一次接见红卫兵大会上与林彪同志的谈话(一九六六年八月十八日)]{在第一次接见红卫兵大会上与林彪同志的谈话}
\datesubtitle{(一九六六年八月十八日)}


这个运动规模很大,确实把群众发动起来了,对全国人民思想革命化,有很大意义。


\input{5-209-1966.8.18-在天安门城楼上与北大附中×××谈话纪要(一九六六年八月十八日).tex}
\input{5-210-1966.8.23-在中央工作会议上的讲话(一九六六年八月二十三日).tex}
\section[关于北航《红旗战斗队》在国防科委坚持斗争问题的指示(一九六六年八月)]{关于北航《红旗战斗队》在国防科委坚持斗争问题的指示}
\datesubtitle{(一九六六年八月)}


不要怕,不要让学生席地而坐,搭起棚子来,让学生闹上三个月。


\section[关于工作组(一九六六年八月)]{关于工作组}
\datesubtitle{(一九六六年八月)}


全国的工作组几乎百分之九十以上犯了普遍性的方向和路线错误。

\kaoyouerziju{ (摘自周总理八月二十二日讲话)}\marginpar{\footnotesize 269}


\section[对四位外国专家的大字报的批示(一九六六年八月二十九日)]{对四位外国专家的大字报的批示}
\datesubtitle{(一九六六年八月二十九日)}


我同意这张大字报,外国革命专家及其孩子要同中国人完全一样,不许两样。请你们考虑一下,凡自愿的,一律同样做。如何,请酌定。


\section[对《关于长沙、青岛、西安等地情况报告》的批示(一九六六年九月七日)]{对《关于长沙、青岛、西安等地情况报告》的批示}
\datesubtitle{(一九六六年九月七日)}


\noindent 林彪、恩来、××、康生、伯达、×××、江青等同志:

此件请看一看,青岛、西安、长沙等地的情况是一样的,都是组织工农反学生,这样下去是不能解决问题的。似宜由中央发一指示,不准各地这样,然后写一篇社论,劝工农不要干预学生运动。北京就没有调动工农整学生,除人民大学曾调六百农民入城保郭影秋之外,其他都没有,以北京的经验告地方照办。

我看谭×××和这位副市长的意见是正确的。

\kaitiqianming{毛泽东}
\kaoyouerziju{一九六七年九月七日}



\input{5-215-1966.9.9-对署名“奥地利《红旗》派的同志”来信的批示(一九六六年九月九日).tex}
\section[关于中央文革(一九六六年九月)]{关于中央文革}
\datesubtitle{(一九六六年九月)}


中央有很多部没做多少好事,文革小组却做了不少好事,名声很大。这个部得改一改。

\kaoyouerziju{ (摘自周总理九月十九日讲话)}


\section[就左派队伍问题与张春桥姚文元同志的谈话纪要(一九六六年十月一日)]{就左派队伍问题与张春桥姚文元同志的谈话纪要}
\datesubtitle{(一九六六年十月一日)}


主席问:你看左派有多少?

答:上海好一点,左派队伍大。

主席问:多大?答:工人有四十万、五十万、六十万。

主席说:左派不会多,大概占10$\%$。

答:后期可能多一点。

主席说:有20$\%$就不得了啦。

(注:以上谈话是六六年国庆节晚上在天安门上看焰火休息时进行的,张春桥同志一九六七年六月十六日在上海革委会扩大会议上传达此次谈话情况时还补充说:这是指世界观的改造,不是指一般的表现,是指世界观里有较多的马列主义、毛泽东思想。要完全的马列主义、毛泽东思想就更难了。)\marginpar{\footnotesize 271}


\section[接见在京部队时的指示(一九六六年十月十三日)]{接见在京部队时的指示}
\datesubtitle{(一九六六年十月十三日)}


下次接见,采取阅兵式的办法。不管多少人,解放军要统统包下来,经过训练,把解放军的光荣传统、三八作风、三大纪律、八项注意,带到全国各地去。三大纪律八项注意歌,人人要会唱,我也要会唱。


\input{5-219-1966.10-关于组织外地来京革命师生进行政治军事训练的指示(一九六六年十月).tex}
\section[对陈伯达同志《两个月来运动的总结》的指示(一九六六年十月二十四日)]{对陈伯达同志《两个月来运动的总结》的指示}
\datesubtitle{(一九六六年十月二十四日)}


直送伯达同志,改稿看过,很好。抓革命、促生产这两句话是否在什么地方加进去,请考虑。要大量印成小册子,每个支部、每个红卫兵小队起码有一份。


\section[在中央政治局汇报会议上的讲话(一九六六年十月二十四日)]{在中央政治局汇报会议上的讲话}
\datesubtitle{(一九六六年十月二十四日)}


毛主席说:“有什么可怕呢?你们看了李雪峰的简报没有?他的两个孩子跑出去,回来后教育李雪峰说:‘我们这里的老首长,为什么那么害怕红卫兵呢?我们又没打你们’。大家就是不检讨。伍修权家有四个孩子,分为四派,有很多同学到他家里去,有时十几个人。接触多了就没有什么可怕的了,觉得他们很可爱。自己要教育人,教育者要先受教育。你们不通,不敢见红卫兵,不和学生说真话,做官当老爷,先不敢见面,后不敢讲话,革了几十年的命,越来越蠢了,刘少奇给江渭清\footnote{原文为××,据静火版毛选改。}的信,批评了江渭清,说他蠢,他自己就聪明了吗?”\marginpar{\footnotesize 272}

毛主席问刘澜涛:“你回去打算怎么办?”刘回答:“回去看看再说”。主席说:“你说话总是那么吞吞吐吐。”

毛主席问总理会议情况,总理说:“会议开得差不多了,明天再开半天,具体问题回去按大原则解决。”

主席问李井泉:“廖志高(四川省委第一书记),怎么样?”李答:“开始不大通,会后一般较好。\footnote{静火版毛选此处:“开始不大通,全会后一段时间比较好了,从历史上
看,他还是一贯正确的。”}”主席说:“什么一贯正确,你自己就溜了,吓得魂不附体,跑到军区去住。回去要振作精神,好好搞一搞。把刘邓的大字报贴到大街上去不好。要允许人家犯错误,要允许人家革命,允许改嘛!让红卫兵看看《阿Q正传》。”

主席说:“这次会开得比较好一些,上次会是灌而不进,没有经验。这次会有了两个月的经验。一共不到五个月的经验,民主革命搞了二十八年,犯了多少错误,死了多少人!社会主义革命搞了十七年,文化革命只搞了五个月,最少得五年才能得出经验。一张大字报,一个红卫兵,一个大串连,谁也没料到,连我也没料到,弄得各省市呜呼哀哉。学生也犯了一些错误,主要是我们这些老爷们犯了错误”。

主席问李先念:“你们今天会开的怎么样?”李答:“财经学院说他们要开声讨会,我要检讨,他们不让我说话。”主席讲:“你明天还去检讨,不然人家说你溜了。”李说:“明天我要出国。”主席讲:“你先告诉他们一下。过去是‘三娘教子’,现在是‘子教三娘’。我看你有点精神不足”。

主席说:“他们不听你们检讨,你们就偏检讨,他们声讨,你们就承认错误。乱子是中央闹起来的,责任在中央,地方也有责任。我的责任是分一二线。为什么分一二线呢?一是身体不好,二是苏联的教训。马林可科不成熟,斯大林死前没有当权,每一次会议都敬酒,吹吹捧捧。我想在我没死之前,树立他们的威信,没有想到反面。

(陶铸同志插话:“大权旁落”)

主席说:“这是我故意大权旁落,现在倒闹独立王国,许多事情不与我商量,如土地会议、天津讲话、山西合作社、否定调查研究、大捧王光美。本来应经中央讨论,作个决议就好了。邓小平从来不找我,从一九五九年到现在,什么事情都不找我。五九年八月庐山会议我是不满意的,尽是他们说了算,弄得我是没有办法的。六二年,忽然四个副总理,李富春、谭震林、李先念、薄一波到南京找我,后又到天津,我马上答应,四个又去了,可邓小平就不来。武昌会议我不满,高指标弄得我毫无办法。到北京开会,你们开六天,我要开一天还不行。完不成任务不要紧,不要如丧考妣。遵义会议后,党内比较集中,三八年六中全会后,项英、彭德怀搞独立王国。(新四军皖南事变、彭德怀的百团大战)那些事情都不打招呼。七大后中央没有几个人,胡宗南进攻延安,中央分两路,我同恩来,任弼时在陕北,刘少奇、朱德在华北,还比较集中。进城后就分散了,各搞一摊,特别分一线二线就更分散了。一九五三年财经会议后,就打过招呼,要大家相互通气,向中央通气,向地方通气。刘邓二人是搞公开的,不是秘密的,与彭真不同。过去陈独秀、张国焘、王明、罗章龙、李立三都是搞公开,这不要紧。高岗、饶漱石、彭德怀都是搞两面手法。彭德怀与他们勾结上了,我不知道。彭真、罗瑞卿、杨尚昆、陆定一是搞秘密的,搞秘密的没有好下场,好结果。犯路线错误的要改,陈、王、李\footnote{陈独秀、王明、李立三}没改(周总理插话:李立三思想没改),不管什么小集团,不管什么门头,都要关紧关严,只要改过来,意见一致。团结就好。要准许刘邓革命,允许改。你们说我和稀泥,我就是和稀泥的人。七大时陈奇涵说:不能把犯王明路线的人选为中央委员,王明和其他几个人都选上了中央委员啦。现在只走了一个王明,其他几个人还在嘛!洛甫不好,王稼祥我有好感,东崮一战他是赞成的,宁都会议洛甫要开除我,周、朱他们不同意,\marginpar{\footnotesize 273}遵义会议他起了好作用,那个时候没有他们不行。洛甫是顽固的,少奇同志是反对他们的,聂荣臻也是反对他们的。对刘少奇不能一笔抹杀。你们有错误就改嘛!改了就行。回去振作精神,大胆放手工作。这次会议是我建议开的。时间这么短,不知是否通,可能比上次好。我没料到一张大字报,一个红卫兵,一个大串连,就闹起来了这么大的事。学生有些出身不太好的,难道我们出身都好吗?不要招降纳叛,我的右派朋友很多,周谷城、张治中,一个人不接近几个右派那怎么行呢?哪有那么干净?接近他们就是调查研究,了解他们的动态。那天在天安门上,我特意把李宗仁拉在一起,这个人不安置比安置好,无职无权好。民主党派要不要?一个党行不行?学校党组织不能恢复太早。一九五七年以后发展的党员很多,翦伯赞、吴晗、李达都是共产党员,都那么好吗?民主党派就那么坏?我看民主党派比彭、罗、陆、杨好。民主党派还要,政协也还要。同红卫兵讲清楚,中国的民主革命,是孙中山搞起来的,那时没有共产党,是孙中山领导搞起来的,反康梁、反帝制。今年是孙中山诞生一百周年,怎么纪念哪?和红卫兵商量一下,还要开纪念会。我的分一线二线走向了反面(康生同志讲:八大政治报告是有阶级斗争熄灭论),报告我们看了,是大会通过的,不能单叫他们两个负责。”

工厂、农村还是分期分批回去打通省、市同学的思想,把会议开好,上海找个安静的地方开会,学生就让他们闹去。我们开了十七天会,有好处。像林彪同志讲的,要向他们做好政治思想工作。斯大林在一九三六年讲阶级斗争熄灭了,一九三九年又搞肃反,这还不是阶级斗争。你们回去要振作精神做好工作,谁会打倒你们?


\section[在中央工作会议上的讲话(一九六六年十月二十五日)]{在中央工作会议上的讲话}
\datesubtitle{(一九六六年十月二十五日)}


讲几句话,两件事。

十七年来,有一件事我看做得不好,就是搞一、二线。原来的意思是考虑到国家的安全,鉴于苏联斯大林的教训,搞了一线二线,我处在二线,别的同志在一线。现在看来不那么好,结果很分散,一进城就不能集中了,相当多的独立王国,所以十一中全会作了改变,这是一件事。我处在二线日常工作不主持,许多事让别人去搞,培养别人的威信,以便我见上帝的时候,国家不会出现那么大的震动,大家赞成这个意见,后来处在一线的同志,有些事情处理得不那么好。有些应当我抓的事情,我没有抓,所以,我也有责任,不能完全怪他们。为什么说我也有责任呢?

第一,常委分一、二线,搞书记处,是我提议的,大家同意了。再嘛是过于信任别人了。这件事引起警惕,还是在制定二十三条那个时候。北京就是没有办法,中央也没有办法。去年九、十月提出中央出了修正主义,地方怎么办?我就感到我的意见在北京不能实行。为什么批判吴晗不在北京发起,而在上海发起呢?因为北京没有人办。现在北京问题解决了。

第二件事,文化大革命闯了一个大祸,就是批发了北大聂元梓一张大字报,给清华附中写了一封信,还有我自己写了一张《炮打司令部》的大字报。这几件事,时间很短,六、七、八、九、十,五个月不到,难怪同志们还不那么理解。时间很短,来势很猛,我也没有料到。北大大字报一广播,全国都闹起来了,红卫兵信还没有发出,全国红卫兵都动起来了,一冲就把你们冲了个不亦乐乎。我这个人闯了这么个大祸。所以你们有怨言,也是难怪的。\marginpar{\footnotesize 274}

上次开会我是没有信心的,说过决定通过了不一定能执行,果然很多同志还是不那么理解。现在经过两个月了,有了经验,好一点了。这次会议两个阶段,头一个阶段,大家发言都不那么正常,后一个阶段经中央同志讲话,交流经验,就比较顺了,思想就通了一些。运动只搞五个月,可能要搞两个五个月,也许还要多一点。

民主革命搞了二十八年(1921—1924年)。开始搞民主革命,谁也不知道怎么个革法,斗争怎么斗争法,以后才摸出一些经验。路也是一步一步从实践中走出来的,总结经验,搞了二十八年嘛。社会主义革命也搞了十七年,文化革命只有五个月嘛,所以就不能要求同志们都就那么理解。去年批判吴晗的文章,许多同志不去看,不那么管。以前批判武训传、红楼梦研究,是个别抓,抓不起来,不全盘抓不行,这个责任在我。个别抓,头痛医头,脚痛医脚,是不能解决问题的。这次文化大革命,前几个月,一、二、三、四月用那么多文章,中央又发了通知,可是并没有引起多大注意,还是大字报、红卫兵一冲,引起注意,不注意不行了。革命革到自己头上来了,赶快总结经验,做政治思想工作,为什么两个月之后又开这个会?就是总结经验,做好政治思想工作。你们回去以后有大量的政治思想工作要作。中央局、省委、地委、县委召开十几天会,把问题讲清楚,也不要以为所有都能讲清楚。有人说,“原则通了,碰到具体问题处理不好。”原来我想不通,原则问题搞通了,具体问题还不好处理?现在看来还是有点道理,恐怕还是政治思想工作没有作好。上次开会回去,有些地方没有来得及很好开会,十个书记有七、八个接待红卫兵,一冲就冲乱了,学生们生了气,自己还不知道,也没有准备回答问题,还以为几十分钟讲一讲,表示欢迎就可以了。人家一肚子气,几个问题一问不能回答就被动了。这个被动是可以改变的,可以变被动为主动的。所以我对这次会议信心增强了。不知你们怎么样?如果回去还是老章程,维护现状,让一派红卫兵对立,拉另一派红卫兵保驾,就搞不好。我看会改变,情况会好转。当然不能过多地要求中央局、省、地、县广大干部全部都那么豁然贯通。不一定,总有那么一些人不通,有少数人是要对立的,但是我相信多数讲得通的。

上面讲两件事情:

第一件事讲历史,一件事讲历史,十七年一线二线,不统一,别人有责任,我也有责任。

第二件事,五个月文化大革命,火是我点起来的,时间很仓促。与廿八年民主革命和十七年社会主义革命比起来,时间是很短的,只有五个月。不到半年,不那么通,有抵触情绪,是可以理解的。为什么不通!你们过去只搞工业、农业、交通,就是没有搞文化大革命,你们外交部也一样,军委也一样。你们没有想到的事情来了,来了就来了,我看冲一下有好处,多少年没有想,一冲就想了。无非是犯错误,什么路线错误,改了就算了,谁要打倒你们!我也是不想打倒你们,我看红卫兵也不要打倒你们。有两个红卫兵对李雪峰讲:“没有想到我们老前辈为什么怕红卫兵?”还有修权四个小孩分成四派,有的同学到他家里来,有时一来好几十个,有好处,我看跟小孩接触很有好处。大接触一百五十万几个钟头就结束了,也是一种方式,各有各的作用。

这次会议发的简报不少,我几乎全部看了。你们过不了关,我也不好过,你们着急,我也着急,不能怪同志们,因为时间太短。有的同志说不是有心犯错误,是糊里糊涂犯了错误。可以原谅也不能完全怪刘少奇同志和邓小平同志,他们有责任,中央也有责任,中央也没有管好,时间太短,新的问题没有精神准备,政治思想工作没有做好,我看十七天会议以后会好一些。

还有哪个讲?今天就完了,散会。\marginpar{\footnotesize 275}


\section[在中央工作会议期间与各大区同志的谈话(一九六六年十月)]{在中央工作会议期间与各大区同志的谈话}
\datesubtitle{(一九六六年十月)}


大家在工作上犯了资产阶级反动路线的错误,主要责任是制定资产阶级反动路线的人,执行的人有各种情况,要区别对待。


\input{5-224-1966.10.25-致阿尔巴尼亚劳动党第五次代表大会的贺电(一九六六年十月二十五日).tex}
\input{5-225-1966.10.25-在中央政治局工作汇报会议上的讲话(一九六六年十月).tex}
\section[在聂荣臻同志去指挥发射导弹时的指示(一九六六年十月二十七日)]{在聂荣臻同志去指挥发射导弹时的指示}
\datesubtitle{(一九六六年十月二十七日)}


你是常打胜仗的,这次可能打败仗,要准备两手。

\kaoyouerziju{ (摘自周总理一九六六年十月廿八日在中央工作会议上的讲话)}\marginpar{\footnotesize 277}


\section[在第七次接见红卫兵大会上与中央负责同志的谈话(一九六六年十一月十日)]{在第七次接见红卫兵大会上与中央负责同志的谈话}
\datesubtitle{(一九六六年十一月十日)}


你们要政治挂帅,到群众里面去,和群众在一起,把无产阶级文化大革命搞得更好。


\section[要支持群众的革命串连(一九六六年十一月)]{要支持群众的革命串连}
\datesubtitle{(一九六六年十一月)}


这是很重要的事,应该大搞,没有了不起的问题,要支持群众的革命串连,要搞就大搞,不会没地方住的。

\kaoyouerziju{ (摘自××一九六六年十一月十八日给我被苏修无理勒令回国的留学生的报告)}


\input{5-229-1966.11.28-祝贺阿尔巴尼亚解放二十二周年的电报(一九六六年十一月二十八日).tex}
\section[在文化大革命中学会大民主(一九六六年十二月)]{在文化大革命中学会大民主}
\datesubtitle{(一九六六年十二月)}


在游泳中学会游泳,在斗争中学会斗争,我们在这次文化大革命中要学会大民主。

\kaoyouerziju{ (摘自周恩来同志一九六六年十二月十九日讲话)}


\section[关于复员转业军人参加文化大革命问题的指示(一九六六年十二月)]{关于复员转业军人参加文化大革命问题的指示}
\datesubtitle{(一九六六年十二月)}


(一)一切复员转业军人,不准成立红卫军和其他名称的单独组织,只能参加所在单位文化革命组织。

(二)不准冲进解放军机关及所属部队,也不许到部队串连和散发传单。

(三)所有转、复军人,必须保持和发扬解放军的光荣传统,并协助解放军加强战备,保卫无产阶级文化大革命。


\input{5-232-1966.12.31-关于军政训练的指示(一九六六年十二月三十一日).tex}
\section[在中央常委扩大会上的四点指示(一九六六年十二月)]{在中央常委扩大会上的四点指示}
\datesubtitle{(一九六六年十二月)}


一、大家要挺身而出,同群众见面,接受群众的批评,并进行自我批评,引火烧身。\marginpar{\footnotesize 279}

二、大家要挺身而出,向群众解释政策。戴高帽子、抹黑脸的,脱帽洗脸,立即上班工作。

三、从长远利益出发,团结多数。牛鬼蛇神就是地、富、反、坏、右少数。有些人就是犯了严重错误,还得挽救他,使之改过自新。不然,怎么能团结95%以上呢?

四、说服干部,使他们懂得,不要人人过关,都搞得灰溜溜的,两个挺身而出,不要怕字当头。敢字当头,最大的问题也能解决。怕字当头,价钱越来越高。


\section[讨论“工矿十条”时的讲话(一九六六年十二月六日)]{讨论“工矿十条”时的讲话}
\datesubtitle{(一九六六年十二月六日)}


先有事实,然后有概念。没有事实,怎么能形成概念?没有实际,那能有理论?有时理论和实际是并行的,有时理论先行,但是实际总归是第一位。工人不先把革命闹起来,那儿来的几条规定?


\section[赞扬参加接待红卫兵和革命师生的解放军指战员的工作(一九六六年十二月)]{赞扬参加接待红卫兵和革命师生的解放军指战员的工作}
\datesubtitle{(一九六六年十二月)}


你们这次在无产阶级文化大革命中,工作做得很好。

\kaoyouerziju{ (周总理一九六六年十二月十九日讲话传达)}


\input{5-236-1966.12.27-给周总理的亲笔信(一九六六年十二月二十七日).tex}
\section[对《林彪同志给浙江省军区的指示》的批示(一九六六年十二月二十九日)]{对《林彪同志给浙江省军区的指示》的批示}
\datesubtitle{(一九六六年十二月二十九日)}


此件应发到全军营以上各级机关去。

\kaitiqianming{毛泽东}
\kaoyouerziju{一九六六年十二月二十九日}

附:林彪同志给浙江省军区的指示

要把学生的工作当作群众工作来做,这是送上门来的群众工作,不但不应当由于这个问题引起军队和革命学生的对抗,而且应当借这个机会,大力加强军队与革命学生的团结。

处理这个问题的原则要重申以下三条:

一、领导同志要挺身而出,同群众见面,既不能躲,也不能压,越躲越压越糟糕。

二、对学生提出的正确批评,要诚恳接受,完全接受,自己做错了的,要坦率地进行自我批评。他们的合理要求,凡能做到的要完全做到。对他们不正确的意见和不合理的要求,要进行解释和教育。

三、从头到尾要贯彻对学生热情、友好、耐心的态度,在耐心的问题上,军队要做出榜样,听到反面的话,绝不能粗暴、发脾气。

{\footnotesize(注:这是一九六六年十二月二十九日浙江军区杜平用电话向林彪同志报告了省军区与浙江红色造反联络站谈判情况后,当天下午五点半钟林彪同志作出的重要指示。)}


\section[和林彪同志的一段谈话(摘录) ]{和林彪同志的一段谈话(摘录) }


林彪:现在全国都在深入学习毛主席著作。

毛主席:我不愿照抄照传,要突破,不要迷信,要有新的论点,新的创造。

林彪:要以毛泽东思想作种子。

毛主席:好。

林彪:不能满足于经济建设,要搞精神建设。\marginpar{\footnotesize 281}


\section[关于一九六七年文化大革命的指示(一九六七年一月一日)]{关于一九六七年文化大革命的指示(一九六七年一月一日)}
\datesubtitle{(一九六七年)}


1.今年搞文化大革命的指导思想是《红旗》和《人民日报》元旦社论,展开全面的阶级斗争。

2.要抓四个重点。北京、上海、天津、东北。责任是在造反派身上,要团结多数,造反派队伍要超过一倍以上。

3.上海很有希望,许多学生、工人、机关干部起来了,这是当前文化大革命的形势。

4.红卫兵要向解放军学习,一定要朴素。

毛主席在元旦祝酒时说:“祝你们明年过社会主义关!”

\kaoyouerziju{ (张春桥同志传达)}


\input{5-240-1967.1.8-关于陶铸问题的指示(一九六七年一月八日).tex}
\section[对中央文革小组的讲话(一九六七年一月九日)]{对中央文革小组的讲话}
\datesubtitle{(一九六七年一月九日)}


《文汇报》现在左派夺了权,四号造了反,《解放日报》六号也造了反,这个方向是好的。《文汇报》夺权后,三期报都看了,选登了红卫兵的文章,有些好文章可以选登。《文汇报》五日《告全市人民书》,《人民日报》可转载,电台可广播。内部造反很好!过几天可以综合报导,这是一个阶级推翻一个阶级,这是一场大革命。许多报纸,依我说封了好,\marginpar{\footnotesize 282}但报还是要出的,问题是由什么人出。《文汇报》、《解放日报》造反好。这两张报一出来,一定会影响华东、全国各省、市。

搞一场革命,定要先造舆论。“六一”《人民日报》夺了权,中央派了工作组,发表了《横扫一切牛鬼蛇种》的社论。我不同意另起炉灶,但要夺权,唐平铸换了吴冷西,开始群众不相信,因为《人民日报》过去骗人,又未发表声明。两个报纸夺权是全国性的问题,要支持他们造反。

我们报纸要转载红卫兵文章,他们写得很好,我们的文章死得很。中宣部可以不要,让那些人住那里吃饭,许多事宣传部、文化部都管不了,你(陈伯达)我管不了,红卫兵一来就管住了。

上海革命力量起来,全国就有希望,它不能不影响华东,以及全国各省市,《告全市人民书》是少有的好文章,讲的是上海市,问题是全国性的。

现在搞革命有些人要这要那,我们搞革命,自一九二〇年起,先搞青年团,后搞共产党,那有经费、印刷厂、自行车?我们搞报纸同工人很熟,一边聊天一边改稿子。我们要与各种人,左、中、右都发生联系。一个单位统统搞得那样干净,我向来不赞成。(有人反映吴冷西很舒服,胖了。)太让吴冷西他们舒服了,不主张让他们都罢官,留在岗位上让群众监督。

我们开始搞革命,接触的是机会主义,不是马列主义。青年时《共产党宣言》也未看过。要抓革命,促生产,不能脱离生产搞革命,保守派不搞生产。这是一场阶级斗争。你们不要相信“死了张屠夫,就吃混毛猪”。以为是没有他们不行,不要相信那一套!


\section[与外宾谈如何看大字报(一九六七年一月)]{与外宾谈如何看大字报}
\datesubtitle{(一九六七年一月)}


看大字报要一分为二,大多数是革命的大字报,有的是不革命的大字报,有的是坏大字报;有的是符合事实的大字报,有的是不符合事实的大字报。


\section[关于接管的指示(一九六七年一月)]{关于接管的指示}
\datesubtitle{(一九六七年一月)}


接管是不可避免的。

我们这个政府,过去是上面派去少数干部和下面大多数留用人员组成了政府,不是工人、农民起来闹革命夺得了政府,这就很容易产生封建主义、修正主义的东西。

(谢副总理说:我们老一点的同志,对这个运动不理解,从开始就不理解,到现在还不理解,转不过弯来。)

转不过弯来靠边站,但给饭吃。

(谢副总理说:昨天向主席谈到,联合行动委员会有许多高干子弟。)这是阶级斗争。

\kaoyouerziju{(摘自谢副总理一九六七年一月十七日对公安干部的讲话)}\marginpar{\footnotesize 283}


\section[谈机关文化大革命的重要性(一九六七年一月九日)]{谈机关文化大革命的重要性}
\datesubtitle{(一九六七年一月九日)}


我们机关的文化大革命是非常重要的。如果只有学生运动、工人运动、农民运动,没有机关干部积极投入运动是不行的,好多重要问题靠机关干部亲自揭露。揭发是不可避免的。我们这个政府过去是由上面派去的少数干部和下面的绝大多数留用人员组成,不是工人、农民起来闹革命夺得了政府,这就很容易产生封建主义、修正主义的东西。


\section[关于军队支持左派的指示(一九六七年一月二十一日)]{关于军队支持左派的指示}
\datesubtitle{(一九六七年一月二十一日)}


\noindent 林彪同志:

应派军队支持左派广大群众。请酌处。

\kaitiqianming{毛泽东}
\kaoyouerziju{一九六七年元月二十一日}

以后,凡有真正革命派要求军队支持援助,都应该这样做。所谓不介入是假的,早已介入了。此事似应重新发布命令,以前命令作废。请酌。又及。


\input{5-246-1967.1.23-谈无产阶级文化大革命新阶段(一九六七年一月二十三日).tex}
\section[在军委扩大会议上的讲话(一九六七年一月二十七日)]{在军委扩大会议上的讲话}
\datesubtitle{(一九六七年一月二十七日)}


一、军队对文化大革命的态度,在运动开始时是不介入的,但实际上已介入了(如材料送到军队上去保管,有的干部去军队)。在现在的形势下,两条路线的斗争非常尖锐的情况下,不能不介人,介入就必须支持左派。

二、老干部的多数到现在对文化大革命还不了解,多数靠吃老本,过去有功劳要很好地在这次运动中锻炼自己,改造自己。要立新功,要立新劳。(这时主席引用了《战国策》的《触詟说赵太后》),要坚决站在左派方面,不能和稀泥,坚决支持左派,之后在左派的接管和监督下,搞好工作。

三、关于夺权。报纸上说夺走资本主义道路当权派和坚持资产阶级反动路线顽固分子的权,不是这样的不能夺?现在看来不能仔细分,应该夺来再说,不能形而上学,否则受限制,夺末后是什么性质的当权派,在运动后期再判断,夺权后报国务院同意。

四、夺权前的老干部和新夺权的干部要共同搞好业务,保守国家机密。

\kaoyouerziju{ (一九六七年元月二十七日周总理传达摘要)}


\input{5-248-1967.1.27-关于军队文化大革命的指示(一九六七年一月二十七日).tex}
\section[关于外国朋友参加文化大革的指示(一九六七年一月二十八日)]{关于外国朋友参加文化大革的指示}
\datesubtitle{(一九六七年一月二十八日)}


外国朋友真正革命的可以参加无产阶级文化大革命运动。\marginpar{\footnotesize 285}


\input{5-250-1967.1-与周总理谈夺权问题.tex}
\section[关于夺权问题(一九六七年一月)]{关于夺权问题}
\datesubtitle{(一九六七年一月)}


如果权落在右派手里,权本来就在右派手里,夺过来。如果再被别人夺过去。仍然在右派手,没有什么了不起,还可以再夺。\marginpar{\footnotesize 286}


\section[为《红旗》杂志一九六七年第三期社论《论无产阶级革命派的夺权斗争》所加的一段话(一九六七年一月)]{为《红旗》杂志一九六七年第三期社论《论无产阶级革命派的夺权斗争》所加的一段话(一九六七年一月)}
\datesubtitle{(一九六七年)}


只要不是反党反社会主义分子而又坚持不改和累教不改的,就要允许他们改过,鼓励他们将功赎罪。

各级干部都应经受无产阶级文化大革命的考验,都应为无产阶级文化大革命建立新的功劳,不能躺在过去的成绩上自以为了不起,看轻新起来的革命小将,对自己只看过去的功劳,而看不见今天的革命大方向,对新的革命小将则又只看到他们的的缺点和错误,而看不见他们革命大方向是正确的,这样看法是完全错误的,必须改过来。


\section[对广播系统夺权的指示(一九六七年一月二十三日)]{对广播系统夺权的指示}
\datesubtitle{(一九六七年一月二十三日)}


中央人民广播电台的革命同志夺了权,很好。听说现在内部又要分裂,内部争吵。还有广播学院革命派掌了权,又分化。要劝他们团结,以大局为重,要搞大团结主义,不搞小团体主义。管他反对不反对自己,反对自己反对错了的人,也要善于和他们团结。和反对自己的人不能合作,我就不赞成。内部有分歧,应按人民内部矛盾来处理,有不同意见可以商量解决。


\section[谈文明斗争(一九六七年二月三日)]{谈文明斗争}
\datesubtitle{(一九六七年二月三日)}


斗争要文明些,我们是无产阶级专政,要高姿态,要高风格。北京街头上标语水平不高,到处都打倒、砸烂狗头,那有那么多的狗头,都是人头。这样搞群众很难理解。搞喷气式飞机照相片,登报贴在大街上被外国记者搞走了。现在要将斗争水平提高,现在水平太低。

八月初,也没这凶嘛,斗倒斗臭要在政治上斗臭,要对后代进行教育。不然他们将来掌权了,也这样干,这就太简单化了。他们认为这样斗臭了。还有把别人生活上的问题摆出来也叫斗臭了,我看不合适。主要是政治上斗臭。\marginpar{\footnotesize 287}


\input{5-255-1967.2.3-和卡博巴庐库同志的谈话(一九六七年二月三日).tex}
\section[关于西安问题的批示(一九六七年二月十四日)]{关于西安问题的批示}
\datesubtitle{(一九六七年二月十四日)}


\noindent 送林彪、恩来同志:

排斥交大派,支持极“左”派的主张值得研究。应继续作调查研究工作,不必急于表态。破坏工厂,极“左”派是有嫌疑的,而交大派不破坏工厂的。请酌。

此件恩来看后送林彪同志。

\kaitiqianming{毛泽东}
\kaoyouerziju{二月十四日}



\input{5-257-1967.2.12-接见张春桥姚文元同志对上海文化大革命的指示(一九六七年二月十二日——十八日).tex}
\section[关于宗派主义问题的指示(一九六七年二月)]{关于宗派主义问题的指示}
\datesubtitle{(一九六七年二月)}


凡是闹宗派主义、小团体主义的,最后都是搞不成的。

\kaoyouerziju(陈伯达同志传达)\marginpar{\footnotesize 292}


\section[给中国人民解放军北京卫戍区二月十八日写的两个报告的批示(一九六七年二月十九日)]{给中国人民解放军北京卫戍区二月十八日写的两个报告的批示}
\datesubtitle{(一九六七年二月十九日)}


中国人民解放军北京卫戍区二月十八日写了两个报告,一个是《关于五所高等院校短期军政训练试点的报告》,一个是《关于两个中学军政训练试点工作总结报告》,毛主席的批示如下:

林彪同志:

一、此两件应即转发全国;

二、大学、中学和小学高年级每年训练一次,每次二十天,上课以后,在军训的二十天中,军训时间每天不要超过四小时,同时学校原课程每天相应减少四小时;

三、党政军民机关除老年外,中年、青年都要实行军训,每年二十天。

以上请酌办。

\kaitiqianming{毛泽东}
\kaoyouerziju{一九六七年二月十九日}

\section[要埋头工作,善于思考(一九六七年二月)]{要埋头工作,善于思考}
\datesubtitle{(一九六七年二月)}


五四运动风流人物,有影响的人物,五四运动的右翼是胡适,后来他成了美帝的走狗,五四运动的陈独秀也成了反革命。当时的李大钊,写的文章也不多,但他埋头工作,后来成为革命的左派。还有鲁迅,当时他重视社会调查,独立思考,后来成为伟大的马克思主义者。我们要从历史当中吸取教训,不做昙花一现的人物,要埋头工作,善于思考,密切联系群众。中国历次的革命及我们亲身经历的革命,真正有希望的人是能想问题的人,不出风头的人,现在大吵大闹的人,一定要成为历史上昙花一现的人物。


\input{5-261-1967.2.20-为中央《给全国农村人民公社贫下中农和各级干部的信》加的一段话(一九六七年二月二十日).tex}
\section[对北京两所中学军训材料的批示(一九六七年二月)]{对北京两所中学军训材料的批示}
\datesubtitle{(一九六七年二月)}


\noindent 林彪同志:

请派人去调查一下,这两校军政训练的经验,是否属实,核实后,可以写一千字左右的总结,发到全面参考。

又,大专院校也要做一个总结,发到全国。请酌。


\section[对《郑州日报》问题的批示(一九六七年二月)]{对《郑州日报》问题的批示}
\datesubtitle{(一九六七年二月)}


准备查封军管,时机宜迟几天,让骂解放军的多骂几天,然后左、右两派各派数人来谈。

\kaoyouerziju{(四月六日周总理传达)}


\section[关于夺权的提法的指示(一九六七年二月二十七日)]{关于夺权的提法的指示}
\datesubtitle{(一九六七年二月二十七日)}


不同意“大联合、大夺权”的口号。难道说没有一个单位是无产阶级当权?建议把大夺权的“大”字去掉。“大联合夺权”。

今后斗争矛头应指向走资本主义道路当权派,不提坚持资产阶级反动路线的顽固分子。

(注:这两个口号均是被王力篡改了的,形“左”而实为极右,引起了许多不良的影响,所以主席发现后就坚决指示把它们改过来。)

\par

“天下者,我们的天下;国家者,我们的国家;社会者,我们的社会;我们不说,谁说?我们不干,谁干?……”这句话是对当年形势提的,今后不要再提了。

\kaoyouerziju{ (摘引周恩来同志1967年3月1日接见西安革命派代表时的讲话。)}


\section[在《论革命的“三结合”》一文中所写的两段话(一九六七年第五期《红旗》杂志社论)]{在《论革命的“三结合”》一文中所写的两段话}
\datesubtitle{(一九六七年第五期《红旗》杂志社论)}


在需要夺权的那些地方和单位,必须实行革命的“三结合”的方针,建立一个革命的\marginpar{\footnotesize 294}、有代表性的、有无产阶级权威的临时权力机构。这个权力机构的名称,叫革命委员会好。

从上至下,凡要夺权的单位,都要有军队代表或民兵代表参加,组成“三结合”,不论工厂、农村、财贸、文教(大、中、小学)、党政机关及民众团体都要这样做。县以上都派军队代表,公社以下都派民兵代表,这是非常之好的。军队代表不足,可以暂缺,将来再派。


\section[关于军队要协同地方管工业的指示(一九六七年三月三日)]{关于军队要协同地方管工业的指示}
\datesubtitle{(一九六七年三月三日)}


此件可印发军级会议各同志。军队不但要协同地方管农业,对工业也要管。沈阳军区派遣大批人员进厂做宣传和做调查的办法是很多的。××军在无锡、××军在重庆、××军在伊春、苇河等处也有好的经验。总之,军队不能坐视工业生产下降而置之不理。


\input{5-267-1967.3-两条路线斗争的基本问题(一九六七年三月).tex}
\section[对铁道兵党委一个报告(渡口驻军支左经验)的批示(一九六七年三月七日)]{对铁道兵党委一个报告(渡口驻军支左经验)的批示}
\datesubtitle{(一九六七年三月七日)}


\noindent 林彪、恩来、文革小组:

此件似可转发全国全军,参照执行。请酌处。

\kaitiqianming{毛泽东}
\kaoyouerziju{三月七日}


\section[对《天津延安中学以教学班为基础实现全校大联合和整顿巩固发展红卫兵的体会》一文件的批语(一九六七年三月七日)]{对《天津延安中学以教学班为基础实现全校大联合和整顿巩固发展红卫兵的体会》一文件的批语}
\datesubtitle{(一九六七年三月七日)}


\noindent 林彪、恩来、文革小组各同志:\marginpar{\footnotesize 295}

此件似可转发全国,参照执行。军队应分期分批对大学、中学和小学高年级实行军训,并且参予关于开学、整顿组织、建立三结合领导机构和实行斗批改的工作。先作试点,取得经验,逐步推广。还要说服学生,实行马克思所说的只有解放全人类,才能最后解放无产阶级自己的教导。在军训时,不要排斥犯错误的教师和干部,除老年和生病的以外,要让这些人参加,以利改造。所有这些要认真去做,问题并不难解决。

\kaitiqianming{毛泽东}
\kaoyouerziju{三月七日}



\section[在胜利中不要冲昏头脑(一九六七年三月)]{在胜利中不要冲昏头脑}
\datesubtitle{(一九六七年三月)}


全国粉碎彭、罗、陆、杨反党集团,粉碎刘、邓资产阶级反动路线,要搞尚未揭开的党内走资本主义道路当权派和顽固坚持资产阶级反动路线的人。有什么理由放松警惕?在胜利中不要冲昏头脑,在胜利中要戒骄戒躁。


\input{5-271-1967-关于《毛泽东选集》注释等问题的指示(一九六七年三、八月).tex}
\section[对中共中央印发薄一波、刘澜涛、杨献珍等人自首叛变材料的批示(一九六七年三月十六日)]{对中共中央印发薄一波、刘澜涛、杨献珍等人自首叛变材料的批示}
\datesubtitle{(一九六七年三月十六日)}


党、政、军、民、工厂、农村、商业内部都混入少数反革命分子、右派分子、变节分子。\marginpar{\footnotesize 296}此次运动中,这些人大部分自己跳出来,是大好事。应由革命群众查明,彻底批判,然后分别轻重,酌情处理。


\input{5-273-1967.3.20-对林彪同志三月二十日报告的指示(一九六七年三月二十日).tex}
\section[对齐齐哈尔铁路局机务段报告的批示(一九六七年三月二十日)]{对齐齐哈尔铁路局机务段报告的批示}
\datesubtitle{(一九六七年三月二十日)}


一切秩序混乱的铁路局都应实行军事接管,以便尽快恢复正常秩序。一切秩序好的铁路局,也应该派出军事代表,吸取好的经验,以利推广。此外,汽车、轮船、港口装卸也都要接管起来。只管工业,不管交通是不够的。此事请你们研究。

\kaoyouerziju{(摘自周总理一九六七年三月二十一日接见铁路、交通、邮电部革委会核心组成员、革命组织和革命群众代表及各部委成员的讲活)}


\section[对报纸工作的指示(一九六七年三月)]{对报纸工作的指示}
\datesubtitle{(一九六七年三月)}


文化大革命以来在报纸上有三个不够,调查的不够,揭露的不够,批判的不够。\marginpar{\footnotesize 297}


\section[关于党组织的指示(一九六七年三月)]{关于党组织的指示}
\datesubtitle{(一九六七年三月)}


我们总要有一个党,现在这样的状况是一种暂时现象,没有一个党是不行的。各个革命造反派的组织怎么能代替党?革命委员会也不能代替党。

\kaoyouerziju{(摘自张春桥同志1967年3月26日在上海整风会上的讲话)}


\section[对《红旗》杂志调查员调查报告《“打击一大片、保护一小撮”是资产阶级反动路线一个组成部分》一文的批示(一九六七年三月二十九日)]{对《红旗》杂志调查员调查报告《“打击一大片、保护一小撮”是资产阶级反动路线一个组成部分》一文的批示}
\datesubtitle{(一九六七年三月二十九日)}


此件很好,可以公开发表,并予广播。还应调查一、二个学校,一、二个机关的情况。请先印发参加碰头会的同志以及其他同志看一看。


\input{5-278-1967.3-关于大批判的指示(一九六七年三月).tex}
\section[关于宜宾问题的批示(一九六七年四月)]{关于宜宾问题的批示}
\datesubtitle{(一九六七年四月)}


许多外地学生,几次冲入中南海,一些军事院校冲进国防部,中央和军委并没有指责他们,更没有让他们认罪,悔过,或者写检讨,讲清楚,让他们回去就行了,而各地把冲军事机关一事,却看得太重了。\marginpar{\footnotesize 298}


\input{5-280-1967.4-对《爱国主义还是卖国主义?》一文所加的一段话(一九六七年四月).tex}
\input{5-281-1967.4-关于部队支左的指示(一九六七年四月).tex}
\input{5-282-1967.4-关于四川文化大革命的指示(一九六七年四月).tex}
\input{5-283-1967.3-关于批判黑《修养》(一九六七年三月、四月).tex}
\input{5-284-1967.4-不能放弃对党内最大走资本主义道路当权派的斗争(一九六七年四月).tex}
\input{5-285-1967.4-关于《触詟说赵太后》(一九六七年四月).tex}
\section[与×同志的一段对话(一九六七年四月)]{与×同志的一段对话}
\datesubtitle{(一九六七年四月)}


主席问:你们说是北京的大学大联合了,成立了红代会,很好呀!那三个司令部呢?

答:没有了,都取消了。

主席:啊,为什么要取消呀?我才不相信呢,已经取消了?

\kaoyouerziju{ (摘自周恩来同志1967年4月18日在广州群众组织座谈会的讲话)}


\section[关于军管问题(一九六七年四月)]{关于军管问题}
\datesubtitle{(一九六七年四月)}


军管太少了。

刘邓派工作组是压迫革命、反对群众,我们军管是支持革命的。

\kaoyouerziju{ (摘自谢富治副总理四月十八日讲话)}


\section[关于革命派的大联合(一九六七年四月)]{关于革命派的大联合}
\datesubtitle{(一九六七年四月)}


目前全国斗争形势很好,成绩很大,经验很多,全国都在前进中。革命派在优势情况下,可按系统、按部门、按单位实行大联合。但要注意保守派把造反派吃掉,不要用解散革命造反派的办法实行大联合。


\section[接见谢富治同志时的谈话(一九六七年四月十九日下午)]{接见谢富治同志时的谈话(一九六七年四月十九日下午)}
\datesubtitle{(一九六七年四月十九日)}


我祝贺你,祝贺这次大会的成功。请代向全北京市革命造反派祝贺。

致敬电是全世界无产者联合起来的大宣言,就不要再搞宣言了。

青年人要参加你们的工作,使前辈人不脱离群众,使青年人得到锻炼。青年人不能脱产,这样会造成脱离群众的,要半官半民。

北京的形势还有反复,无政府主义就是机会主义的惩罚。要不怕犯错误,各种反动的观点和反动的群众组织是极少数的。就是反动组织也要做工作,但是还得斗争。


\section[无政府主义是对机会主义的惩罚(一九六七年四月)]{无政府主义是对机会主义的惩罚}
\datesubtitle{(一九六七年四月)}


过去的八条,现在的十条结合起来是对的。左派起来了,对立面也起来了,这也不要紧,有点反复有好处。无政府主义是对机会主义的惩罚,要走向反面。

要把所有的联动放出来。\marginpar{\footnotesize 302}


\input{5-291-1967.4.23-关于陕西驻军虚心听取群众意见改进工作报告的批示和批注(一九六七年四月二十三日).tex}
\input{5-292-1967.4.23-关于四川问题的指示(一九六七年四月二十三日).tex}
\input{5-293-1967-在中央常委、中央文革小组和政治局会议上的讲话(摘录) (一九六七年四、五月).tex}
\section[ “五一”节对阿尔巴尼亚贵宾的谈话(一九六七年五月一日)]{ “五一”节对阿尔巴尼亚贵宾的谈话}
\datesubtitle{(一九六七年五月一日)}


主席对阿尔巴尼亚贵宾说:“我们还是有困难的。中国是有希望的,世界是有希望的。当前主要任务就是大批判,大斗争,尽快实现三结合。(有人谈到像“九评”那样批判《修养》。)主席说,不要写长文章,两千字就够了,不要超过三千。


\input{5-295-1967.5.1- “五一”节对中央首长的谈话(一九六七年五月一日).tex}
\input{5-296-1967.5.1- “五一”节和张×的谈话(一九六七年五月一日).tex}
\section[对棉粮工作的指示(一九六七年五月一日)]{对棉粮工作的指示}
\datesubtitle{(一九六七年五月一日)}


必须把粮食抓紧,把棉花抓紧,把布匹抓紧,粮油征购工作不仅是一个经济任务,它首先是一个政治任务。一定要十分抓紧,每年一定要把收割、保管吃用(收、管、吃)抓的很紧,而且抓的及时。

不注意储备,铺张浪费,吃光用光,这些都是错误的,都是粮食工作中的资产阶级反动路线或者是资产阶级思想的反映。


\input{5-298-1967.5.2-关于大批制问题的指示(一九六七年五月二日).tex}
\section[关于军队整训的指示——给林彪同志的信(一九六七年五月七日)]{关于军队整训的指示——给林彪同志的信}
\datesubtitle{(一九六七年五月七日)}


\noindent 林彪同志:

各地军队都应整训一个短时期,时间10——14天为宜,已经整训过的,一个月或三个月再整训一次,全军三支两军人员,每一个月或两个月,都应整训一次,发扬成绩,纠正错误,以利再战。


\section[对广州、湖南军区报告的批示(一九六七年)]{对广州、湖南军区报告的批示}
\datesubtitle{(一九六七年)}


一、现将广州、湖南军区报告两个文件,发给你们希望遵照执行。

二、是错误的必须改正,如果不改正,越陷越深,到头来还得改正,威信损失就大了。及早改正威信只能比以前高。

三、不能动动摇摇,犹犹豫豫,听老婆孩子从保守派那里来的错误话,信以为真。\marginpar{\footnotesize 307}

四、要受得住工人、农民、学生、战士、干部的批评,加以分析,好的接受,错的改正,解释不通的,暂时搁下来将来再说。

五、要坚决相信群众大多数是好的,坏人是少数,不过百分之一、二、三,这样一想什么问题也想通了。


\input{5-301-1967.5.8-对《红旗》杂志、《人民日报》两编辑部一九六七年五月八日文章《〈修养〉的要害是背叛无产阶级专政》一文所加的两段话(一九六七年五月七日).tex}
\input{5-302-1967.5-要破除资产阶级法权思想(一九六七年五月).tex}
\section[对北京市革命委员会的指示(一九六七年五月)]{对北京市革命委员会的指示}
\datesubtitle{(一九六七年五月)}


机构不要庞大,要粉碎旧的官僚机构,旧市委继承了官僚的机构,官僚机构容易被走资本主义道路当权派所利用,所以建立革命委员会很必要。要破坏旧北京市适合走资本主义道路当权派利用的机构。搞了几个月了,这个经验是逐步取得的,要总结几条。

\kaoyouerziju{ (据谢富治同志一九六七年五月五日在北京市革命委员会全体会上的讲话)}


\section[干革命要有阶级感情(一九六七年五月)]{干革命要有阶级感情}
\datesubtitle{(一九六七年五月)}


要搞好文化大革命就要和工农相结合,要有无产阶级感情,要有工农兵感情。

\kaoyouerziju{ (据陈伯达同志一九六七年五月十一日在北京女六中讲话)}


\section[对上海革命派的号召(一九六七年五月)]{对上海革命派的号召}
\datesubtitle{(一九六七年五月)}


希望你们成为同党内一小撮走资本主义道路当权派作斗争的模范。

希望你们成为实行革命大联合的模范,成为反对小团体主义,反对无政府主义,反对经济主义,反对自私自利的模范。

\kaoyouerziju{ (据姚文元同志在上海整风报告会上的传达)}


\input{5-306-1967.5-对上海市革委会的号召(一九六七年五月).tex}
\section[关于国际形势的指示(一九六七年五月)]{关于国际形势的指示}
\datesubtitle{(一九六七年五月)}


有人说,美苏战略中心转移,我不赞成。他们是注意远东,但中心还在欧洲。欧洲七个师的兵力,并没有减少,只是调了几万老兵到远东。


\input{5-308-1967.5-关于处理军民关系的几点指示(一九六七年五月).tex}
\input{5-309-1967.5.18-在《伟大的历史文件》一文中所写的一段话(一九六七年五月十八日).tex}
\input{5-310-1967.5-接见阿尔巴尼亚军事代表团时的讲话(一九六七年五月).tex}
\section[在文化革命中立新功(一九六七年六月)]{在文化革命中立新功}
\datesubtitle{(一九六七年六月)}


老干部过去有功劳,但是不能靠吃老本,要很好地在无产阶级文化大革命运动中锻炼改造自己,要立新功,立新劳。

\kaoyouerziju{ (转引自六月一日《人民日报》)}


\section[关于夏收的指示(一九六七年六月三日)]{关于夏收的指示}
\datesubtitle{(一九六七年六月三日)}


积极行动起来夏收夏种,目前正进入两夏大忙季节,南方已开始收割。今年全国夏收作物生长很好。今年的夏收,各级领导同志应特别重视,立即投入夏收夏种。各地要组织学生、机关干部、工人、解放军,以劳力、畜力、技术力量,大力支持人民公社,做到颗粒归仓。严防阶级敌人破坏。


\section[关于中央各部运动的指示(一九六七年六月)]{关于中央各部运动的指示}
\datesubtitle{(一九六七年六月)}


“全国的形势比我们预料的要慢,中央各部要慢。”开始说三、四月出眉目,前几天主席说“比预料的晚了一个季度,不是三、四、五月了,而是六、七月了。与其估计快一点,不如估计慢一点,与其估计很容易,不如估计困难一点。’


\section[中国革命的伟大世界意义(一九六七年六月)]{中国革命的伟大世界意义}
\datesubtitle{(一九六七年六月)}


中国革命的胜利,主要不是搞地下斗争,而是武装斗争。

就是全世界都黑了,只要中国是光明的,那世界就有希望,没什么问题。

无产阶级文化大革命,首先要把北京、上海、天津这几个地方搞好。上海就是工人这个队伍比较好,所以上海的局势中央也比较放心。

\kaoyouerziju{(六月八日张春桥同志传达)}\marginpar{\footnotesize 314}


\input{5-315-1967.6-关于干部问题的指示(一九六七年六月).tex}
\section[有了错误就改(一九六七年六月)]{有了错误就改}
\datesubtitle{(一九六七年六月)}


如果有了错误定要痛痛快快承认错误,改正错误,改的越快、越彻底越好。绝不能扭扭捏捏,吞吞吐吐,更不能坚持错误,越走越远。(转引自六月二十七日河南军区检查报告)


\section[关于叛徒问题的指示(一九六七年六月)]{关于叛徒问题的指示}
\datesubtitle{(一九六七年六月)}


对于叛徒,除罪大恶极者外,在其不继续反共的条件下,予以自新出路。如能回头革命,还可以接待。但不能重新入党。


\section[关于对外宣传的指示(一九六七年六月十八日)]{关于对外宣传的指示}
\datesubtitle{(一九六七年六月十八日)}


有些外国人对我们《北京周报》、新华社的对外宣传有意见。宣传毛泽东思想发展了马克思主义,过去不搞,现在文化大革命以后大搞特搞,吹得太厉害,人家也接受不了。有些话何必要自己来说,我们要谦虚,特别是对外,出去要谦虚一点,当然就不要失掉原则。昨天氢弹公报,我就把“伟大的导师、伟大的领袖、伟大的统帅、伟大的舵手”统统勾掉了。“万分喜悦和激动的心情”我把“万分”也勾掉了。不是十分,也不是百分,也不是千分,而是万分,我一分也不要,统统勾掉了。\marginpar{\footnotesize 315}


\input{5-319-1967.6.26-对中央警卫团支左部队的三点指示(一九六七年六月二十六日).tex}
\section[对赞比亚总统的谈话(摘要)(一九六七年六月)]{对赞比亚总统的谈话(摘要)}
\datesubtitle{(一九六七年六月)}


非洲是很有生气的大洲。在亚洲,是东南亚比较突出,美国黑人运动正在蓬勃发展。

先独立的国家是要援助晚独立的国家的。你们不要感谢我们,这是为了世界革命,是关系到世界革命,不援助才是错误的。世界革命可能还要几百年,任务就落在我们身上了,特别是落在下一代,所以我们要好好培养接班人。


大学的斗、批、改是一项艰巨的工作。一种可能,改革彻底翻身;一种可能,是走回头路;一种可能,是改良。


\section[修改中国红卫兵代表团去阿尔巴尼亚的发言稿(一九六七年六月)]{修改中国红卫兵代表团去阿尔巴尼亚的发言稿}
\datesubtitle{(一九六七年六月)}


翻印者注:[]内为毛主席修改时删去的字句,黑体字为毛主席添加的。

一、在[最高统帅]毛泽东主席的教导下……

二、在[战无不胜的]毛泽东思想哺育下……

三、在以\textbf{伟大的共产主义战士}思维尔·霍查为首的光荣的阿尔巴尼亚劳动党……

四、领导阿尔巴尼亚青年胜利前进的阿尔巴尼亚劳动党及其敬爱的\textbf{伟大}领袖思维尔·霍查同志领导下……

五、伟大的战无不胜的马克思列宁主义[毛泽东思想]万岁!


\section[对姚文元同志访问阿尔巴尼亚的指示(一九六七年六月)]{对姚文元同志访问阿尔巴尼亚的指示}
\datesubtitle{(一九六七年六月)}


这次出去要注意谦虚啊!这不是一个代表团的问题,而且关系到中国红卫兵问题。


\section[关于绝食静坐的问题(一九六七年六月二十九日)]{关于绝食静坐的问题}
\datesubtitle{(一九六七年六月二十九日)}


社会主义制度下,无产阶级专政条件下,绝食静坐可以做为一种斗争手段。因为无产阶级内部有一小撮敌人,有官僚主义。他无非要让你答应几个条件。所以绝食静坐,允许。但不能提倡,一般不赞成。绝食可以开水放糖、放盐、打葡萄糖,可以轮班。


\section[看到西安师院造反派两张大字报和给中央文革的信后的批示(一九六七年六月)]{看到西安师院造反派两张大字报和给中央文革的信后的批示}
\datesubtitle{(一九六七年六月)}


现在有的人是“铺张闹革命”,“大少爷作风”,“掌权开始,是浪费开始。”如果是人民内部矛盾的话,批评人家宽一些,批评自己严一些。


\section[关于缅甸问题的指示(一九六七年七月一日)]{关于缅甸问题的指示}
\datesubtitle{(一九六七年七月一日)}


缅甸向题不怕断交,不怕决裂,甚至于这个时候断交更好,这样更有利于我们放开手干。


\section[关于建造主席塑像问题的指示(一九六七年七月五日)]{关于建造主席塑像问题的指示}
\datesubtitle{(一九六七年七月五日)}


\noindent 林彪、恩来及文革小组各同志:

此类事是劳民伤财,无利有害,如不制止,势必会刮起一阵浮夸风。碰头会讨论一下,发一指示,加以制止。

\kaitiqianming{毛泽东}
\kaoyouerziju{七月五日}



\section[对军队同志的教导(一九六七年七月)]{对军队同志的教导}
\datesubtitle{(一九六七年七月)}


有了错误就检讨就改正,改了就好了,只要讲三句话就行了。一句话:错了;二句话:是什么错就是什么错,在这个问题上或那个问题上是方向错了就承认这个问题上方向错了;第三句话:就改正。这样就好了。公开检查比不公开好,高姿态检查比低姿态好,早检查比晚检查好。

\kaoyouerziju{ (摘自××在1967年7月7日在昆明讲话)}

军队这几年群众工作做得少,错了改了就好,也不给处分,也不要治罪。

\kaoyouerziju{ (摘自周总理1967年7月7日接见山西代表时的讲话)}


\input{5-328-1967.7.7-在听了氢弹工作会议汇报后的讲话(一九六七年七月七日).tex}
\section[关于“农村包围城市”的口号(一九六七年七月)]{关于“农村包围城市”的口号}
\datesubtitle{(一九六七年七月)}


现在提出“农村包围城市”,这个口号是反动的。过去革命条件下,提出这个口号是对的。但现年的情况变了,城市住的无产阶级,你用农村包围城市干什么?包围城市就是包围无产阶级,包围革命造反派。

\section[关于不要“开快车”的指示(一九六七年七月十三日)]{关于不要“开快车”的指示}
\datesubtitle{(一九六七年七月十三日)}


开快车要翻车,要打招呼。当前主要搞大联合,三结合,坏人挖出来,牛鬼蛇神挖出来。党组织要恢复,各级党代表大会召开,人民代表大会召开,我看大体要到明年这个时候,大家不要有疲劳的感觉,不要想脱身。


\input{5-331-1967.7.13-接见军队领导干部时的讲话(一九六七年七月十三日).tex}
\input{5-332-1967.7-关于湖南军区的龙书金(一九六七年七月).tex}
\input{5-333-1967.7-关于教育革命的指示(一九六七年七月).tex}
\section[《中央对武汉军区公告的复电》(一九六七年七月二十五日)]{《中央对武汉军区公告的复电》}
\datesubtitle{(一九六七年七月二十五日)}


\noindent 林、周、中央文革小组:

拟复电如下,请讨论酌定。

\kaitiqianming{毛泽东}

一、你们现在所采取的立场和政策是正确的,公告可以发表。

二、对于犯了严重错误的干部,包括你们和广大革命群众所要打倒的陈再道同志在内,只要他们不再坚持错误,认真改正,并得到群众谅解了以后,仍然可以站起来参加革命行列。

三、要向思想不通的某些人员和百万雄师群众做工作,使他们转变过来。\marginpar{\footnotesize 320}

四、要向左派做工作,不要乘机报复。

五、要警惕坏人捣乱,不许破坏社会秩序。


\input{5-335-1967.7-对武汉事件的指示(一九六七年七月).tex}
\section[关于“八一”建军节的指示(一九六七年七月)]{关于“八一”建军节的指示}
\datesubtitle{(一九六七年七月)}


“八一”不能改,这是很重要的一天。我们打响了第一枪,为井冈山的斗争揭开了序幕。

这个问题是历史问题,历史问题是不容颠倒的。


\section[为一九六七年七月卅日《人民日报》社论加的一句话(一九六七年七月)]{为一九六七年七月卅日《人民日报》社论加的一句话}
\datesubtitle{(一九六七年七月)}


团结一致,同心同德,任何强大的敌人,任何困难的环境,都会向我们投降的。\marginpar{\footnotesize 321}


\input{5-338-1967.8-关于大联合的指示(一九六七年八月).tex}
\section[关于大串连的指示(一九六七年八月)]{关于大串连的指示}
\datesubtitle{(一九六七年八月)}


去年走是对的,今年就不是时候了,帮倒忙。

\kaoyouerziju{ (转摘自周总理1967年8月16日对北京红代会工作人员的讲话)}


\section[对《红旗》杂志一九六七年第十二期社论的批语(一九六七年八月)]{对《红旗》杂志一九六七年第十二期社论的批语}
\datesubtitle{(一九六七年八月)}


\textbf{还我长城。}

(注:这篇题为《向人民的主要敌人猛烈开火》的社论,是由林杰起草,王力、关锋批准发表的。社论中错误地提出了“军内一小撮”的口号,从而转移了斗争大方向,主席立即提出了严厉的批评。)


\section[对天津与河北问题的指示(一九六七年八月)]{对天津与河北问题的指示}
\datesubtitle{(一九六七年八月)}


天津、河北应该搞得好一些,因为天津、河北离北京近,特别是天津,容易影响北京。\marginpar{\footnotesize 322}


\section[无产阶级专政下革命的主要对象(一九六七年八月)]{无产阶级专政下革命的主要对象}
\datesubtitle{(一九六七年八月)}


在无产阶级专政下革命的主要对象。是暗藏在无产阶级专政机构内部的资产阶级司令部,我们就是对无产阶级专政机构内部的这一部分进行革命。从我们党和我们国家的整体来说,它们是不占统治地位的,但是必须打倒他们,才能巩固和强化无产阶级专政,防止资本主义复辟。

\kaoyouerziju{ 转摘自《红旗》杂志一九六七年第十三期社论《彻底摧毁资产阶级司合部——纪念党的八届十一中全会召开一周年》}


\section[关于武斗的指示(一九六七年八月)]{关于武斗的指示}
\datesubtitle{(一九六七年八月)}


对武斗不要看得太紧张,对形势不能看得太严重,不要急。那里有武斗,必然有后台,让他多表演一下,越表演越孤立,使群众看得更清楚,群众孤立他,就好办了。乱是暂时的,可以转化为好的。打架是支流,是暂时的支流。决不能转移斗争的大方向。


\section[对《中共中央、国务院、中央军委、中央文革小组给煤炭工业战线职工的一封信》的批示和修改(一九六七年八月十六日)]{对《中共中央、国务院、中央军委、中央文革小组给煤炭工业战线职工的一封信》的批示和修改}
\datesubtitle{(一九六七年八月十六日)}


\textbf{批示:已阅,照办。}

\kaitiqianming{毛泽东}

主要修改:

信的开头,“广大的职工群众们”一句是主席加的。

信的第七段,在“无产阶级革命派和革命职工必须把国家利益”一句后加了“工人阶级利益”几字。

第八段“凡是坚持下去、坚持生产,做出成绩的职工,不论属于哪个群众组织或者未参加组织,都应当给予表扬和适当奖励。”整段都是主席新加进去的。


\input{5-345-1967.8-关于军队支左问题的指示(一九六七年八月).tex}
\input{5-346-1967.8-要从政治上、思想上、理论上把走资派打倒(一九六七年八月).tex}
\section[8·27指示(一九六七年八月二十七日)]{8·27指示}
\datesubtitle{(一九六七年八月二十七日)}


一、许世友要帮助过关,他是一个战将,文化革命以来,落后了,跟不上;

二、要出马恩列斯语录;

三、文化大革命要搞三年,一年发动一年胜利,一年扫尾;

四、天津南开大学卫东红卫兵写了一篇好文章(指《要大胆使用革命干部》),提出了一个新问题,红卫兵小报是好东西;

五、不要把党内一小撮走资派和军内一小撮走资派并提,只提党内一小撮。把解放军搞垮了还要不要政府?

六、群众组织提以我为核心,这样提是极蠢的;

七、依靠青年人;\marginpar{\footnotesize 324}

八、依靠群众,加强专政,新疆办大型劳改农场不一定好,要研究;

九、要抓学习,把工作做好,要精兵简政。


\input{5-348-1967.8-对公安工作的指示(一九六七年八月).tex}
\section[关于陈毅的几点指示(一九六七年八月十一日)]{关于陈毅的几点指示}
\datesubtitle{(一九六七年八月十一日)}


材料不黑,性情直爽。


陈毅是个好同志,对陈毅要一批二保。


要保他,他是第三野战军司令,外交部长现在没人搞,还要他来搞。


陈毅怎么能打倒呢?陈毅跟了我四十年,功劳那么大。陈毅现在掉了二十斤肉,不然我带他接见外宾。


\section[对姚文元同志《评陶铸的两本书》一文的批示(一九六七年九月)]{对姚文元同志《评陶铸的两本书》一文的批示}
\datesubtitle{(一九六七年九月)}


好极,此文的要害是点破了“五·一六。”


\section[对新闻工作的指示(一九六七年九月)]{对新闻工作的指示}
\datesubtitle{(一九六七年九月)}


新闻单位要树立严肃作风、科学作风、战斗作风,要培养无产阶级革命者的战斗风格。

\kaoyouerziju{ (据陈伯达同志1967年9月3日对新闻人员讲话时传达)}\marginpar{\footnotesize 325}


\section[对庆祝国庆的指示(一九六七年九月)]{对庆祝国庆的指示}
\datesubtitle{(一九六七年九月)}


今年的国庆要充分的宣传文化大革命的成就。


游行不单是游行,也是大批判。用文艺武器批判党内最大的走资本主义道路当权派。


\input{5-353-2-中共中央通知.tex}
\input{5-353-毛主席视察华北、中南和华东地区时的重要指示 .tex}
\input{5-354-毛主席视察华北、中南和华东地区时谈话的主要精神的传达.tex}
\section[对林彪同志1967年国庆讲话的评语(一九六七年十月)]{对林彪同志1967年国庆讲话的评语}
\datesubtitle{(一九六七年十月)}


这个讲话是很好的讲话,气势磅礴,又没有夸张,是文化大革命以来很好的总结。


\section[造反派要听周总理的话(一九六七年十月)]{造反派要听周总理的话}
\datesubtitle{(一九六七年十月)}


造反派不听周总理的话,还叫什么造反派?矛头对准周总理,就是对准我、林彪。\marginpar{\footnotesize 339}


\section[关于机关革命化的指示(一九六七年十月)]{关于机关革命化的指示}
\datesubtitle{(一九六七年十月)}


不要调人。二十多个人就够了。要那么多人干什么呢?人少了,多下去,会也开的短一些,在下面解决问题,少搞些小报。


\section[接见刚果(布)总理努马扎莱的谈话(一九六七年十月三日下午)]{接见刚果(布)总理努马扎莱的谈话(一九六七年十月三日下午)}
\datesubtitle{(一九六七年十月三日)}

\begin{duihua}

\item[\textbf{努马:}] 我看到中国无产阶级文化大革命的伟大胜利,使中国人民政治觉悟大大提高了。

\item[\textbf{主席:}] 无政府主义也大大发展了。

\item[\textbf{努马:}] 也许是这样,但是我们还没有看到这些。

\item[\textbf{主席:}] 有那么个思潮暴露出来好教育。

\item[\textbf{努马:}] 你们的干部很谦虚。

\item[\textbf{主席:}] 非谦虚不可,否则群众斗他们。

\item[\textbf{努马:}] 你们的干部与我们的干部有很大的区别。

\item[\textbf{主席:}] 没有多大区别。都是官大了,薪水多了,坐小汽车了。大官还得有人做,大官没人做还得了!薪水多一点,房子好一点,坐汽车也可以,但不要摆架子,和工农群众平等相待。不要动不动就训人,骂人。有的大队书记,薪水不多,房子不好,没坐小汽车,官也不大,就是官架子不小。运动一开始,结果把他们吓了一跳。

\item[\textbf{努马:}] 外国讲中国很乱,我们怎么没看到?

\item[\textbf{主席:}] 乱一点,你们可以到处走走,乱了以后就不乱了。不闹够就不行。这时候差不多了。我们准备再乱一年。

\item[\textbf{努马:}] 什么叫越乱越好呢?

\item[\textbf{主席:}] 不乱胜负不分,湖南、湖北、江西、安徽、浙江,除安徽省外都好。到中国得一条经验。湖南煤矿动刀动枪了。生产几万吨下降到几吨,现在已产二万吨。

\item[\textbf{努马:}] 这样的矛盾,怎么解决得这么好呢?

\item[\textbf{主席:}] 后台揪出来了,群众打够了,这时中央讲几句话就行了。×××吹中国怎么好,不要听那一套,非洲人架子小,所以我们希望你们来。欧洲、亚洲就不行。

\item[\textbf{努马:}] 我们也开始反官架子。

\item[\textbf{主席:}] 我不建议你们也搞文化大革命。我们建军四十周年,建国十八年,打了二十二年,拥有打了几十年仗的解放军,所以搞文化大革命。

\item[\textbf{努马:}] 我们不搞文化大革命,但我们要研究文化大革命的理论和世界意义。

\item[\textbf{主席:}] 这次文化大革命要改变国家部分机构,包括军队。思克鲁玛那次来,没有料到推翻他政权的就是他的军队。我看你还是早点回去。

\item[\textbf{努马:}] 我们家里还有人,但我尽量早点回去。\marginpar{\footnotesize 340}
\end{duihua}

\section[关于按系统实现革命的大联合的指示(一九六七年十月十七日)]{关于按系统实现革命的大联合的指示}

\datesubtitle{(一九六七年十月十七日)}

各工厂、各学校、各部门、各企业单位,都必须在革命原则下,按照系统,按照行业,按照班级,实现革命的大联合,以利于促进革命三结合的建立,以利于大批判和各单位斗、批、改的进行,以利于抓革命、促生产、促工作、促战备。

\kaoyouerziju{(转引自1967年10月18日《人民日报》)}


\section[对海军学习毛主席著作积极分子代表大会代表的谈话(一九六七年十一月十九日)]{对海军学习毛主席著作积极分子代表大会代表的谈话}
\datesubtitle{(一九六七年十一月十九日)}


这个像章好(指海军学习毛主席著作积极分子代表大会敬献给毛主席的像章——编者),有海军又有空军,又有这么多群众,我就放心了。


\section[进行教育革命要依靠无产阶级革命派(一九六七年十月二十五日)]{进行教育革命要依靠无产阶级革命派}
\datesubtitle{(一九六七年十月二十五日)}


进行无产阶级教育革命,要依靠学校中广大革命的学生,革命的教员,革命的工人要依靠他们中间的积极分子,即决心把无产阶级文化大革命进行到底的无产阶级革命派。

\kaoyouerziju{ (转引自1967年10月27日《人民日报》在发表《关于教育革命的几个初步方案》时所加的“编者按”。)}


\section[对《解放军报》1967年11月9日社论《抓好形势教育》的重要修改(一九六七年十一月)]{对《解放军报》1967年11月9日社论《抓好形势教育》的重要修改}
\datesubtitle{(一九六七年十一月)}


在社论的第三段,毛主席特地重新引用了1938年10月他在中国共产党六届六中全会上的报告中的如下两段话:“当前的运动的特点是什么?它有什么规律性?如何指导这个运动?这些都是实际的问题。”“运动在发展中,又有新的东西在前头,新东西是层出不穷的。研究这个运动的全面及其发展,是我们要时刻注意的大课题。”

\kaoyouerziju{(见《中国共产党在民族战争中的地位》,《毛泽东选集》第二卷第二版523页)}\marginpar{\footnotesize 341}

在同一段“紧跟毛主席的战略部署”一句后,毛主席添加了“要经常把运动中出现的新动向、新成就、新经验、新问题,用毛泽东思想进行分析和总结,及时向广大指战员进行教育,使他们的思想能不断跟上发展着的新形势。”一句。

第四段,在“就是要用毛泽东思想统一人们对形势的认识”后,加了“识破阶级敌人的挑拨煽动”一句。

第六段,“无产阶级文化大革命的实践证明,只要毛泽东思想同亿万人民群众相结合”后,加了“除了叛徒、特务、顽固不化的党内走资派和社会上的牛鬼蛇神(即没有改造好的地、富、反、坏、右)以外”一句。


\section[关于党组织的指示(一九六七年七月)]{关于党组织的指示}
\datesubtitle{(一九六七年七月)}


党组织应是无产阶级先进分子所组成,应能领导无产阶级和革命群众对于阶级敌人进行战斗的朝气蓬勃的先锋队组织。

\kaoyouerziju{(转摘自《人民日报》、《红旗》、杂志、《解放军报》一九六八年元旦社论:《迎接无产阶级文化大革命的全面胜利》)}


\input{5-370-1967.11-关于正确对待犯过错误的老造反派的指示(一九六七年十一月).tex}
\input{5-371-1967.12.17-给林彪周恩来、中央及文革的信(一九六七年十二月十七日).tex}
\section[对姚文元同志一封信的批示(一九六七年十二月)]{对姚文元同志一封信的批示}
\datesubtitle{(一九六七年十二月)}


中央认为各地都应当这样做。但党组织内不应当再允许有查明证据的叛徒、特务和文化大革命中表现极坏而又死不悔改的那些人再过组织生活。党组织应是无产阶级先进分子所组成.应能领导无产阶级和革命群众对于阶级敌人进行战斗的朝气蓬勃的先锋队组织。


\input{5-373-1967.12.19-给阮友寿主席的贺电(一九六七年十二月十九日).tex}
\section[关于举办学习班的指示(一九六七年十月-一九六八年二月)]{关于举办学习班的指示(一九六七年十月-一九六八年二月)}
\datesubtitle{(一九六七年十月)}


中央应该开,主要是各省开,不仅军队开,地方党政文教也要集训。

造反派也要训,他们坐不下来,心野了。一期不行,可以训练两期、三期。造反派人很多,我看训练的办法好。

军队办学习班,要有战士参加。

办学习班,是个好办法,很多问题可以在学习班得到解决。

\kaoyouerziju{(转引自一九六八年二月五日《人民日报》、《解放军报》社论《华北山河一片红》)}\marginpar{\footnotesize 344}


\section[对广东问题的指示(一九六八年二月)]{对广东问题的指示}
\datesubtitle{(一九六八年二月)}


广东形势大好,省革筹要抓紧关键,立即成立革命委员会,带动广西、云南、福建、湖南等地区。因为此国防前线,有三条黑线。

主席对派性很有意见,大联合已经号召一年了,现在还讲派性。

文汇报的社论《论派性的反动性》主席看过说:“很好”。

\kaoyouerziju{(周总理给广州军区负责人温玉成同志电话传达的主席最新指示)}


\section[关于革命委员会等的指示(一九六八年二月)]{关于革命委员会等的指示}
\datesubtitle{(一九六八年二月)}


大办学习班。用学习班的方法斗私批修,提高认识,解决问题,狠抓思想革命化,组织革命化。已经成立革命委员会的应该巩固和发展,革命委员会就是好。应该总结经验,应该解放大批革命干部,干部只要不是三反分子、走资派、投敌叛变分子、特务分子,而是在运动中犯了路线错误认真检查,认识错误就可以三结合。在三结和中应该注意成份,但不能唯成份论,不要把坏人也结合进来。三结合要体现老、中、少,光小娃娃不行。

一般大学的革命委员会,解放军不结合进去,特殊情况下要结合,须经过市革命委员会批准。要警惕坏人,防止破坏。已经成立了革命委员会的应该爱护她,尊重她,帮助她,保卫她,维护她的无产阶级权威,严防阶级敌人破坏。革命委员会的成立,不是三支两军工作的完成而是进入一个新的阶段,巩固和发展革命委员会的无产阶级权威。

\kaoyouerziju{(据传达精神)}


\section[无产阶级文化大革命是共产党和国民党长期斗争的继续(一九六八年三月)]{无产阶级文化大革命是共产党和国民党长期斗争的继续}
\datesubtitle{(一九六八年三月)}


无产阶级文化大革命,实质上是在社会主义条件下,无产阶级反对资产阶级和一切剥削阶级的政治大革命,是中国共产党及其领导下的广大革命人民群众和国民党反动派长期斗争的继续,是无产阶级和资产阶级阶级斗争的继续。

\kaoyouerziju{见《人民日报》、《解放军报》一九六八年四月二十一日社论《芙蓉国里尽朝晖》}\marginpar{\footnotesize 345}


\section{革命委员会的三条基本经验}\datesubtitle{(一九六八年三月三十日)}

革命委员会的基本经验有三条:一条是有革命干部的代表,一条是有军队的代表,一条是有革命群众的代表,实现了革命的三结合。革命委员会要实行一元化的领导,打破重迭的行政机构,精兵简政,组织起一个革命化的联系群众的领导班子。

\kaoyouerziju{ (转摘自《人民日报》、《红旗》杂志、《解放军报》一九六八年三月三十日社论《革命委员会好》}
\input{5-379-1968.4.16-支持美国黑人抗暴斗争的声明(一九六八年四月十六日).tex}
\section[对派性要进行阶级分析(一九六八年四、五月)]{对派性要进行阶级分析\\{\large——几段最新指示}}
\datesubtitle{(一九六八年四、五月)}

对派性要进行\footnote{原文为“进在”,据《人民报纸》更正为“进行”}阶级分析。

\kaoyouerziju{《人民日报》、《解放军报》一九六八年四月二十日社论:\\《无产阶级革命派的胜利》}

党外有党,党内有派,历来如此。

除了沙漠,凡有人群的地方,都有左、中、右,一万年以后还是这样。

\kaoyouerziju{《红旗》杂志一九六八年四月二十六日评论员文章:\\《对派性要进行阶级分析》}

派别是阶级的一翼。

\kaoyouerziju{《人民日报》、《红旗》杂志、《解放军报》五一社论:\\《乘胜前进》}\marginpar{\footnotesize 347}

\end{document}