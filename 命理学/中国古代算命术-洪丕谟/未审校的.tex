

\section{入格八字举例}
在命理学家眼里,虽然人们出生时辰的八字千变万化,错综复杂,可是总得有个格局统帅全局,否则不就乱了套?这就是入格八字的由来。
关于八字的格,从来就为命书所重视。如《三命通会》卷六的《杂取各格以及《星平会海》卷十的取格析例,都不惜以整卷的篇幅,对八字的各种格局,作出了详尽的分析。
关于入格八字的看取办法,专以代表自身的日干为主,然后配合月令、年时,而以月令为重,其中逢官希财(财能生官);逢财看煞(财能生煞),逢煞看印(印能化煞),逢印看官(官印相生)。歌云:
一官二印三财位,四煞五食六伤官。
立法先详生与死,次分贵贱吉凶看。
在命书中,对于命的格局,有正格和变格的不同。凡是以官、煞、印、财、食伤等入局的,叫做正格。正格以外,叫做变格。现把命书所载有关格局,择要举例如下。
1.正格
①正'官格在用神中,正官是天地的正气,忠信的尊名,虽然治国齐家,劳苦功髙,可是八字中出现正官,只要一位就足够了,并且出现的部位,以月柱为正,又怕刑冲。如果官星太多,或官煞(偏官)相混,或部位®离月柱,或官M
算
逢冲,就难以入格了。这就是命书所说的:“正气官星,切忌刑冲,多则论煞,一位名真。”如果时柱上兼有财星的,更是贵不可言。
〔入格八字〕	金状元
(年)乙卯(月)	丁亥正官
(日)丁未(时)正财庚戌
八字月柱中亥宫壬水,克自身天干丁火为正官;并且四柱中只有一位官星,而时柱上又出现丁火的正财庚金,所以便把它取作正官格局了。诗曰:
正官须在月中求,无破无伤贵不休。
玉勒金鞍真赋态,两行旌节上星州。
②偏官格所谓偏官,就是七煞有制的称谓。如果八字中同时出现偏印、偏财,身煞平衡,就是大富大贵的命。如果七煞被制过头,或者八字中官煞混杂,那就退职离官,多致凶死。又如行运进入煞乡,也主不死而穷。此外,日柱天干无根而遇煞制,或煞重藏根,主人都有被煞制死的可能性。所说煞重藏根,就是七煞直接藏在自身日柱的地支中,比如乙酉日生的人,酉是辛金,克乙木为煞,这时如果年柱、时柱中又不见制煞的干支,就是很不吉利的命。
〔入格八字〕	沈郎中
(年)丙子子中癸水制脤偏官(月)甲午丁火偏宫i(日)辛亥—
(时)辛卯
月柱午中丁火克自身辛金既可称为偏官,也可称为七煞,然而这丁火因为被年柱子中癸水制服,根据有制为偏官,无制为七煞,以及取格看重月令的原则,所以这是一种偏官的格局。至于年干丙火克辛金为正官,因为不在月令之上,所以不取为格。此外,月干透出甲木作为辛金的正财,也在一定裎度上为这格局添了分数。诗曰.•
偏官有制化为权,唾手登云发少年》
岁运若行身旺地,功名大用福双全。
③七煞格
在命局中,七煞是克我的神,需要有制为福,好比恶宿小人,须要制伏,也可为我所用。在命书中,虽然有“‘七煞’有制,谓之偏官”的说法,可在举格局中,却也并不分得那么清楚。我们且看下面一个七煞格局:
〔入格八字〕	李寺丞
、(年)己巳(月)丁卯(日)丙午自旺(时)煞壬辰
这里,时上壬水克自身丙火为煞,然而周围却也不乏制水的土,可见“七煞”、“偏官”等格,原,也并不区分限制得那么明显,所以有的命书,索性把“偏官”、“七煞”统称为偏官格或七煞格,倒也来得干净》
按照“时上一位为贵”的原则,凡是上好的七煞格局,七煞的位置一定要出现在时柱上,并且只要一位,不可多见。
算
S如时柱出现七煞,而年、月、日柱上又重复出现的,那就非但不贵,反而成了辛苦劳碌的命。对于“时上一位为贵”的七煞格,只要本身自旺而有制伏,行运进入七煞旺乡,必定发迹。反之,如果命中七煞没有制伏,成了的的确确的弋煞格,那末只要行到对七煞有所制伏的运里,也可发迹;只怕•命里七煞没有制伏,而运又行到煞旺无制的境地,只就难免生出祸患了。诗说:	'	丫
时上七煞是偏官,有制身强好、/肴。
制伏喜逢煞旺运,三方得地发何难?
元无制伏运须看,不怕刑冲多煞攒。
若是身衰官煞旺,定知此命是贫寒。
④印绶格
在用神的名称中,印绶是生我的。符合这种格局的人,身旺为福,四柱中最喜透出官星七煞,以及行官煞的运,因为官煞能够生印。大忌柱中出现太多的财,这也因为财能伤克印绶。至于四柱纯都是印,由于印绶太过,反而走向事物的反面,所以注定主人是孤独的命。
〔入格八字〕陈都宪(年)-官癸未(月)正印乙卯正印(日)	丙子跑胎逢印
(时)官癸巳
八宇月柱中乙卯两个乙木,都是自身日干丙火的印绶,而日
支子对于日干丙来谇,在f生十二宵中叉正好处在天地气

交,氤氳造物的胎的状态,这就更加需要印绶来促成了。妙在年柱、时柱中透出的两颗官星癸水,也为这正印的格局增添了分数。诗曰:
月逢印绶喜官星,运入官乡福必清。
死绝运临身不利,后行财运百无成。
⑤正财格
•在格局中,正财最喜身旺印续,忌官星、忌倒印(偏印),忌身弱比肩、劫财。忌见官星的原因是怕盗财气,然而正财格中带有官星,又行上财旺生官的大运,则反而可以更加发迹。反过来说,如果柱中财多身弱,则怕行财旺生官的运,否则反而祸患临头。又如财神宜藏,藏则丰厚,蔣则浮荡,行运如果碰上比扃、劫财,非但分去财产,弄得不好,恐怕连命都要丢掉了。此外,也有些情况,比如身强财旺的逢财看煞,见官更好,所以命书乂有“财藏谣官者,当作贵推”的说法。
〔入格八宇〕	孛罗丞相
(年)壬申(月)丙午午中己土为财(日)甲午(时)壬申
这字月支午中己土,为自身甲木的正财,而自身日支又坐财知,所以在取格时把它看作是正财格。加之年柱、时柱的壬水申金,不是生我甲木的印绶就是制我甲木的七煞,所谓“逢财看煞”,对于印旺生财米说,可以说是致中和的最佳方案了。诗曰:	、
财星忌透只宜藏,身旺逢官大吉昌c怕逢比劫来相会,一生名利被分张。
⑥	偏财格
如果偏财出现在时上的,与时上七煞格局一样,只要一位,其他三柱不要重复出现。而这位时上的偏财,又怕逢冲,如果一旦行运进入财旺之乡,那就发福百端了。
〔入格八字〕李参政(年)庚寅(月)乙酉正官(日)甲子(时)戊辰戊日偏财z
这一命造,月支正官不透,时柱戊土下坐辰支透气通根,所以考虑取戊土偏财为格局。入偏财格的,除了菩行财运,最怕逢冲外,还大忌行到羊刃败财和劫财的运,因为这样偏财被分被劫,就全完了。诗说:
时上偏财一位佳,不逢冲破享荣华。
败财劫刃还无遇,富贵双全比石家。
⑦	食神格
食神如在月令提纲中出现的只要一位,并且要身旺,因为食神能够生财,如逢身弱,那就难以克财了y对于入食神格的人来说,四柱忌印绶、官煞,以及比肩、羊刃(劫财)为祸。如果大运一旦进入食神财旺运,便可发福,、
〔入格八字〕蜀王(年)己未(月)戊辰身旺•136
(日)丨戈辰(时)庚申
这里蜀王八字的食神虽然出现在时干上,可是因为得力,所以便把它取作食神格局。由于自身戊土,生在春天末一个月的三月辰月,土令得时,所以身旺。诗曰:
食神身旺窖生财,日主刚强福禄来。
•身弱食多反为害,或逢枭食主凶灾。
⑧伤官格
“伤官见官,为祸百辟”,因为在用神中伤官是正官的克星,如果官来乘旺,那就祸不可言了。所以入伤官格的,伤官一定要彻底伤尽才好。所谓伤尽,就是四柱中一点也不出现官星。八字中如伤官多,有财星,或行身旺运,或行财旺运,都是富贵发福的命。命理学家认为,“伤官乃小人之情,喜财而妒官,又行财运,反生富贵”。此外伤官身旺无财的凶,这种人如果一旦碰上官运,就会大祸临头,理应尽快退身避职。大凡伤官只喜财旺身旺,如果行运进入财衰和死绝等地,那就脱财无禄,不是官司打败,就是死期临头了。
〔入格八字〕通参政(年)甲寅
(月)庚午己土伤官	Z
(日)丙午(时)甲午
字月支午中己土对丙火来说,是我生的伤官。由于格中一点也没有丙火的官星癸水,所以伤官伤尽;加之伤官多,
月干透出庚金财星,自身丙午,午又是丙的帝旺之乡,所以是个发福宫贵的命。诗曰:
火土伤官伤宜尽,金水伤官要见官,
木火见官官有旺,土金官去返成官,
惟有水木伤官格,财官两见始为欢。
以上正官、偏官(包括七煞)、财、印、食、伤八种格局,命书称为正格。下入变格。
2.变格
①杂气印绶格
在月份中,辰、未、戌、丑等月,也就是三、六、九、十二月,月支辰中有乙木、癸水、戊土,未中有丁火、乙木、己土S戌中有辛金、丁火、戊土,丑中有癸水、辛金、己土,这里面包涵了天地驳杂不纯之气。比如以东方的甲乙木举例,甲则坐镇资位阳木,乙则坐镇卯位阴木,两者司管春令,面夺东方之气,可辰虽属于暮春三月.•然而这时已处于春a交接之处,方位已经偏向东南,所以受气不纯,禀命不一,有杂气之称。其他未、戌、丑三月,也照此原理类比。
在杂气印绶格中,如果自身日干是甲的,要出生在十二二月丑月的才称得上贵,因为丑中辛金是甲木的正官,丑中癸水是甲木的正印,丑中己土为甲木的正财。如果这时把握不住财、官、印中取哪一样来定格,可以观察月干中透出的是什么用神,然后再决定取舍。然而辰、戌、丑、未都是库藏,要有钥匙打开,才能发福,才能为我所用,而这种打开库藏的钥匙,就是刑冲破害。但这种刑冲破害,要恰到好处,否则冲破过头,反而伤了福份。大抵杂气霜要财多,便可为’
算
贵。假如在年、时等柱中有符合其他格局的,则当以其他格
局来论。	:、、
〔入格八字〕葛待诏(年)庚寅(月)丙戌丁火为印(日)戊子(时)癸丑
这种格局,忌行财运官运。八字的主人葛待诏,早先原是卖玳瑁梳子的,只因杂气中月令透出丙火,月支癩有丁火为印,所以行运一旦戌库冲破,就发迹了。可是毕竟由于日支子为癸水,属于戊土的财,而时支丑里又含有一定董的癸水为财,这样财能破印,水去克火,平时尚可维持过去,然而一旦行入子运,运中癸水和命中癸水一起呼应起来,那就泛滥成灾,火光灭没了。后来果然当行到子运时,这位葛待诏就寿终正寝了。这用算命的术语来说,就是“贪财坏印”。诗曰:
辰戌丑未为四季,印绶财官居杂气。
干头透出格为其,只问财多为尊贵。•
②杂气财官格
命书逢辰、戌、丑、未月出生的,有杂气之称。大抵杂气要财多透餒为贵,逢官也好。因为辰、戌、丑、未属于墓库,需要冲开,这样库中的财官印绶才能为我所用,否则官墓不显其名,财库不用于世,印墓不得为信,不就形同虚设?〔入格八字〕王尚书(年)正财戊子
算
(月)	壬戌辛金为官戊土为财
(日)乙亥(时)丁丑
自身乙亥,生于戌月,戌中辛金为官,戊土为财,而其中戊土又透出年干,所以就成了杂气财官的格局。诗曰:
杂气财官四库中,还须破害与刑冲。
•天干透出财源格,财多身旺禄相同。
③羊刃比肩格
所谓比肩,就是同类中阳见阳,阴见阴的称谓,好比兄弟姐妹的同类一样。同类中阳见阴则不称比肩而称败财,又称羊刃,阴见阳则不称败财而称劫财。八字中如果“印财身理见者,能夺伤官七煞,身弱见者,劫财分官见剥。”
〔入格八字〕髙太尉(年)庚午丁火(月)乙酉正官(日)甲寅(时)乙亥长生
本命天干甲木,生于八月,以西中辛金为正�����。然而年干出现庚金为七煞,这种官煞相混,辱不妙了。好在乙庚合金,甲木把妹妹乙木嫁给庚金为妻,命书中有《贪合忘煞”的说法,况且年支中又有丁火制服庚金,不致为灾。再看时支透出乙木,作为甲木的羊刃,而时支亥中癸水,又使甲木处于长生状态,所以行运一旦进入丑运,丑中辛金即抑乙木,又使自身甲木官运亨通,所以官至二品。诗曰:
春木夏火两相逢,秋金冬水一般同。
~140—
不宜羊刃天干透,运至重逢又反凶。
④	七煞羊刃格
所谓七煞,就是偏官,喜制伏,客羊刃。如命局中七煞、羊刃同时出现的,往往可以把它看作这种格局,但忌财多,否则便不成格局了。对于七煞羊刃格的人来说,最怕羊刃逢冲,替如丙日、戊日生人羊刃在午,因为午中丁火、己土分别屑于日干丙、戊的羊刃,这时如果行运进入正财子地,子午相冲,破了羊刃,就不妙了。同样,壬日生人羊刃在子,忌行午地正财的运,庚日生人羊刃在酉,忌行卯地正财的运,甲日生人羊刃在卯,忌行西地正财的运。如果格局中,羊刃不被冲破,那末碰上财运,问题不大
〔入格八字〕不花平章(年)乙卯(月)戊子羊刃(日)壬戌七煞(时)壬寅
局中命主生于壬日,月柱日柱分坐子、戌两支,子中癸水为壬水的羊刃,戌中戊土为壬水的七煞。壬水生于仲冬子月,得令身旺,七煞被时支寅中甲木所制,有制为吉。这样身强煞浅,七煞羊刃成格,所以是极贵的命。
⑤	金神格
金神只有三时,就是癸酉、己巳、乙丑,凡是四柱中时柱上出现这三个时辰的,就被认为是金神格。但也有认为,要逢六甲日出生的,才可入这格局,其中甲子、甲辰更好。金神原是破败之神,凡入这格局的,四柱中要火制伏为贵,或

行运进入火乡也好。如果运入水乡,水泄金气,大祸就临头
T。
〔入格八宇〕岳武穆(年)癸未(月)己卯〈日》甲子(时)己巳金神
N•1	•g1m一£
平会海》说:“甲、己为平头煞,生逢春月,身旺财弱,主骨肉参商,平生做事,弄巧成拙。己巳金神有火制伏,已酉丑合局,运行南方,名重禄高。柱不见火,残害化气,主凶恶暴亡。”“甲子日,己巳时,先贫后富,祖业轻微,妻勤子拗,诗说:
癸酉己巳并乙丑,时上逢之是福神。
傲物恃才宜制伏,交逢刃煞贵人其0
性多狠暴才明敏,遇水相生立穷困。
,丨ft运行逢火局,超迁贵显窗无伦。?	.、
⑥魁罡格
魁圼有四,就是庚辰、壬辰、戊戌、庚戌,其中辰是水库,属天罡,戌是火库,属地罡,辰戌相见,所以成了一种天冲地击之煞。凡是命造中日柱逢庚辰、壬辰、戊戌、庚戌的,就属于这一魁罡格局。《三命通会》说:“经云:魁罡聚众(四柱中出现魁罡的不只日柱一处),发福非常。主为人性格聪明,,文章振发,临事果断,秉权好杀。”如果“运行身旺,发福百螂,一见财.官,祸患立莫”9j于平总论》说:“身值天罡坶魁,.
了1钱,
衰则彻骨贫寒,强则绝伦贵显。”然而对于这种格局,也需活看,比如《三命通会》举例张时佥祺的八字是庚午、丁亥、戊戌、丙辰,刘大受少卿的八字是丁亥、癸丑、庚戌、戊寅,虽说出生的一天都是魁罡日,可是却不忌财官印,就是明证。诗0:
壬辰庚戌与庚辰,戊戌魁罜四座神。
不见财官刑煞併,身行旺地贵天伦。
⑦日德格	:1
入这一格局的只有阳干五天,就是甲寅、丙辰、戊辰、庚辰、壬戌。其中甲坐寅得禄,丙坐辰官库,庚坐辰财印两全,壬坐戌财官印具备,并且地支寅为三阳之首,辰、疼为魁罡之地,所以这五天的干支,就很奋点和其他日子的干支不同了。八字中出现日德的,不R其多,如果只有日柱一位日德的,那末取格时就要按照月柱中财官印食,另作别论了。平时口德除庚辰自兼魁罡二职外,不赞命中还是大运,最忌和魁罜同时出现,否则便认为是很不好的命运。
〔入格八字〕张学官(年)甲申(月)戊辰(日)戊辰(时)壬戌
命造中有三位日德,由学官而“腰金衣紫”,得五品官诰,很是不错。又如有庚辰、己卯、戊辰、平寅这样一命,按理有三位日德,该是好命。然而甲贫忌见身兼魁罡的庚辰,后来运行壬午财乡之地.午中己土为日干戊土的羊刃,犯了日德的忌讳,到丁巳年时,寅已相刑,四月死,寿只三十八岁。这是(三命通会》所记载的。诗说:
日德喜煞喜身强,不喜财星官旺乡,
为性温柔更慈善,一生福寿乐非常。
日德不毐见魁罡,化成煞曜最难当a局中重见还须疾,运限逢之必定亡。
⑧	日贵格
命造中出生在丁酉、丁亥、癸巳、癸卯四天的人,因为日干坐在天乙贵人星上,所以便把格局称为“日贵”。其中又有日贵、夜贵之分:丁亥、癸卯日生的,生时要在白天,叫做“日贵”,又叫“昼格”•,丁酉、癸巳日生的,生时要在黑夜,叫做“夜贵”,又叫“夜格”。《三命通会》说:“经云:贵人者,慈祥恺悌之号,德星尊重之命。遇财官印食则吉,值煞刃冲刑则凶。运遇魁罡,为害不浅。”对于日贵格的命,八字中如果聚上两三位贵人的,主人存粹仁义,贵不可言,怕就怕地支贵人逢冲受损,又怕日上空亡和魁罡加临,这样非但不贵,反而贫贱而夭了。诗说:
丁遇猪鸡癸兔蛇,刑冲破害漫咨嗟。
财临会合方成章,昼夜分之最为佳。
⑨	建禄格
八宇中自身日柱天干五行,与月建配合起来正好处于临官禄地,比如甲乙春生,丙丁夏旺,庚辛秋锐,壬癸冬长,以及己生于巳午月等就是。按照五行寄生十二宫的说法,甲禄临官)在寅,乙禄在卯,丙禄在已,丁禄在午,氏禄在已,
己禄在午,庚禄在申,辛禄在西,壬禄在亥,癸禄在子。因此如果日干甲木,月支在寅,年支如果没有其他大的破坏,就可认为入了建禄的格。
〔入格八字〕贺丞相(年)辛丑(月)庚寅甲禄在寅(日)甲辰(时)乙亥
八字中日干甲,月建在寅,寅为甲禄。月干和年干庚、辛金虽然分别为甲木的七煞和正官,官煞混见,可是当大运进入丙戌、丁亥制煞之乡时,那就大富大贵了。诗曰:
建禄生提月,财官喜透天。
不宜身再旺,官地是良缘。
⑬归禄格
这种格和建禄格不同的是,建禄格看日干和月支的配合是有禄地,而归禄格则把日干的禄归到时支中去找,如果时支中有禄,而四柱又没有官星七煞的,可认为入了归禄的格。经云:“日禄归时没官星,号曰青云得路,
〔入格八字〕林枢密(年)财戊子(月)甲寅(日)乙亥(时)己卯归祿
此格八字中四柱没有一点官星,所以富•贵入格。但如果在行运中遇到官星,就凶而不吉了。诗曰:
算
日禄归时格最ft,怕官嫌煞害自强。
若见比肩分劫禄,刑冲破害更难当。
⑪壬骑龙背格
这一格局以壬辰日出生为主,四柱又多见壬辰、壬寅,其中辰字多的贵,寅字多的富,如果纯见寅、辰两支,而没有别的地支掺杂进来,那就宫资双全了。经云阳水4逢辰位,是壬骑龙背之乡。”这种格大忌官星盛旺,若见戌和辰冲,也不为福。
〔入格八字〕王枢密••
(年)壬辰(月)甲辰(日)壬辰(时)壬萸为壬财官
王巨富
(年)壬寅(月)壬寅'(日)壬辰
(时)壬實遍地是财,以致巨富
以上两个八字,前一个辰多,所以贵过于富,后一个寅多,所以富过于贵,都是很典型的。诗曰:
壬骑龙背怕官居,重fi逢辰贵有余。•
假若寅f,辰字少,须应豪富比陶朱。
⑫六乙鼠贵格	•
•此格以六乙日生,时间逢子,而四柱中又没有官星,方才官髙名显。在神煞中,乙见V为贵人,而十二生肖又以子
属鼠,所以有41六乙鼠贵”之名。由于此格子为乙的贲人,所以大忌逢冲。若果八字或大运中子午相冲,那就通盘都坏了。
〔入格八字〕曹尚书(年)丁巳(月)壬寅(日)己卯(时)丙子贵人
《喜忌篇》说:“阴木独遇子时,为六乙鼠贵之地。”诗曰:
乙日生人得子时,名为鼠贵最为奇。
切嫌午宇来冲破,辛酉庚申总不宜。
⑬六甲趋乾格
此格为六甲日生,时间逢亥。亥宫属乾,为甲木的长生之地,所以有“六甲趋乾”的名称。入此格的,四柱和岁运中不客财星,同时又忌寅、已两字。因为甲财M土,能制亥水,寅与亥合,已与亥冲,都不佳妙。
〔入格八字〕新安伯(年)戊辰(月)癸亥(日)甲子(时)乙亥亥为乾
大凡逢甲日出生的,亥支不厌其多,又无已来相冲,这样就自然富贵了。诗曰:
趋乾六甲最为奇,甲日生人得亥时。
岁运若逢财旺地,官灾患难祸相随。
—U7—
⑭六壬趋艮格
入这个格的,要以六壬日生寅时为准,格中寅字多的,又叫做合禄格。在八字和岁运中,最怕逢申相冲,又忌财官。
〔入格八宇〕
(年)壬寅(月)壬寅(日)壬寅(时)壬寅
寅为艮宫,所以有“趋艮1的叫法。寅宫甲木能合己土,丙火能合辛金,这样就暗遨己土为壬水的正官,辛金为壬水的印绶了。行运不厌身旺之地,如遇申字相冲,那就大打折扣了。诗曰:
六窠趋艮喜非常,壬日寅时是赍乡。
大怕刑冲并克制,逢申岁运有灾殃。
⑮勾陈得位格
勾陈在五行中属戊己土。在六戊日和六己日中,遇上财官的有戊寅、戊子、戊申、己卯、己亥、己未等六日其中以申子辰水局为财,亥卯未木局为官。入这种格局的,最怕刑冲煞旺,反生灾害。
〔入格八字〕丁都督《年)丁亥(月)丁未(日)己卯(时)戊辰
八字中以己卯日为勾陈,遇亥卯未木局为官星得地,所以是个贵命。诗曰:
日千戊己坐财官,号曰勾陈得位看,
知有大才分瑞气,命中值此列朝班。
⑯玄武当权格
玄武在五行中属壬癸水。在六壬日和六癸日中,遇上财官的有壬寅、壬午、壬戌、癸未、癸丑、癸巳六日,其中以寅午戌火局为财,辰戌丑未土局为官。入这种格局的,在八宇和岁运中最忌身弱冲破。
〔入格八字〕季都司(年)庚戌(月)壬午(日)壬寅(时)辛亥
格中地支寅午戌火局为财,水火既济,所以窗贵荣华。诗曰:壬癸名为玄武神,财官两见始成真。
局无冲破当清贵,辅佐皇家一老臣。
⑫稼穑格
稼穑在五行中属中央戊己土。凡是入此格的,不仅日干要逢戊己土,并且还要地支辰戌丑未全是土局,加上无木克削,有水为用,自然便就福禄绵绵I.〔入格八字〕张真人(年)戊戌(月)己未(日)戊辱水
U9—
(时)癸丑水
此命地支辰戌丑未俱全,得水为财,又无木克,所以有搞。诗
戊己重逢杂气夭,土多只论木居全。
财星得遇堪为福,官煞如临有祸缠。
⑯曲直格
曲直在五行中属于东方甲乙木。凡是入此格的,不仅日干要逢甲乙木,并且地支还要会成亥卯未木局,或寅卯辰全,加上无金克削,有水为印,命主仁而搞寿。
[入格八字]李总兵(年)甲寅(月)丁卯
(日)乙未	7
(时)丙子
此格日干乙木与地支寅卯未会成木局,加上时支癸水生木,又无官煞相侵,所以盛而为官。诗曰:
甲乙生人寅卯辰,又名仁寿两堪评。
亥卯未全嫌白帝,若逢坎地必荣身。
诗中“嫌白帝”是说嫌弃庚辛金气,因为金属西方白帝I坎地是指水地,在八卦中坎属于水,所以古人常用坎指代水。⑲炎上格丨,	
炎上在五行中属于南方丙丁火。凡是入这格局的,不但日干要遇丙丁火,并且还要地支会成寅午戌火局,或巳午未全,加上身旺,运行东南,便就浑成文明,夫富大贵〔入格八字〕张太保	.

(年)乙未火(月)辛巳火(日)丙午火(时)平午火
局中自身丙火逢地支巳午未火局,一片炎上之性,故为朝中朱紫之贵。诗曰:
火多炎上气冲天,玄武无侵富贵全。
一路东方行运好,簪缨头顶带腰悬。
⑳润下格
润下在五行中属于北方壬癸水。凡是入这格局的,不但日干要遇壬癸水,并且还要地支会成申子辰水局,或亥子丑全。平生忌辰戌丑未官乡,喜西方印地,不宜东南,怕冲克。〔入格八字〕万宗人(年)庚子水(月)庚辰水	.•
(日)壬申水(时)辛亥水
这命非但地支申子辰全,子亥水局浑然,并且年、月、时柱又得庚辛生水,所以一片湛然,檫量广阔,为富贵之人。诗曰:天干壬癸喜冬生,更值申辰会成局,
或是全归亥子丑,等闲平步上青云。
㉑从革格
从革在五行中属于西方庚辛金。凡是入这格局的,不但日干要遇庚辛金,并且还要地支会成已酉丑全局,或申酉戌全。平生忌南方火运,冲刑克破,喜庚辛旺运。
(入格八字〕杨太尉(年)辛酉金、
(月)戊戌金(日)庚申金(时)辛巳金
命中自身庚金,地支申酉戌全,月干又透出戊土生金,所以福高禄深。诗曰:
秋月金居一类看,名为从革便相欢,
如无炎帝来临害,定作当朝宰辅官。
⑬弃命从财格
命中日干身弱,四柱中又全无印绶相生,比肩扶助,而天干地支却透财会财,造成财旺身弱的局势,这时就索性丢弃自身专以财论,所以叫弃命从财格。此格喜行财旺运,怕入煞印之乡。比如天干乙木而地支辰戌丑未土全,财神极旺,这时如果四柱无依,就只有以弃命从财格论。碰上这种格的,主人平生不是怕老婆,就是做人家的招女婿。因为财者妻也,自身既无依靠,托的全是妻子的福,所以作这样的分析。
〔入格八字〕王十万(年)庚申财(月)乙酉财(日)丙申财(时)己丑财
命中财多身弱,自身少有所助,所以只有舍命从之,才能有
搞。诗曰:
日主无根财犯重,全凭时印旺身官。
逢生必主兴家业,破印纷纷总是空。
®贪合忘煞格
四柱中财官两旺,这时如果柱中透出的煞被合去,就叫贪合忘煞。入这种格的,虽然衣禄丰厚,可却无官而好酒色。比如甲日柱中逢庚,庚就是甲的七煞,这时桂中如又透出乙木和庚相合,便就称作贪合忘煞。又如甲日柱中逢辛,辛是甲的正官,这时柱中如又透出丙火和辛相合,便也称作贪合忘官。其中煞要合,官不要合,所以有“合煞不为凶,合官真不美”的说法,又说:“煞无刃不威,刃无煞不显。”这里的刃,就是羊刃,也就是败财,可见刃煞一起出现,也并不是坏事。这又有点岔到别处去了。
〔入格八字〕王指挥•(年)	丙申
(月)合宫辛丑(日)甲辰(时)财戊辰
这命生于甲日,月柱中有官星照临,而戊辰时又财旺生官,本该大贵,可是偏偏年干丙火与官星辛金相合,命书说:“贪合忘官为颠邪。”现在官星既被合掉,只好作贱命看了。加之四十五岁入丙午运,火势太炎,申木干燥,木被火焚,则不禄矣。诗曰:	《
贪合忘官合不足,合煞不伤为己福。
堪叹身弱怕逢敗,更历官乡祸自逐。
⑭干辰一字
入此格的,年、月、日、时四柱,天干清一色而+杂,所以清贵。
〔入格八宇〕
(年)壬子戊辰甲子庚申丙寅•.;(月)壬子戊午甲戌庚辰丙申(B)壬子戊申甲寅庚戌丙午(时)壬寅戊午甲子庚辰丙申@支辰一字
这种格局,年、月、日、时地支一字不杂,一色纯清,也是一种贵命。
〔入格八字〕		
甲寅	戊辰	乙亥
丙寅	丙辰	丁亥
庚寅	甲辰	己亥
戊寅	戊辰	乙亥
根据宋代吴曾《能改斋漫录》记载,宋代宰相曾布的八字为乙亥、丁亥、辛亥、己亥,此外宰相萧注的八字则为癸丑、乙丑、乙丑、丁丑,这两人显然都是属于这一格局的贵人c⑯两干不杂	.■、/\久
这种格局,四柱天干只有两个天干,纯一不杂,所以取名为“两干不杂”。
〔入格A字〕
甲子
乙丑
这一命禄,局中天干只有甲、乙两字,不杂不乱。此外如丙寅、丁酉、丙辰、丁寅,局中天干也4有丙、丁两字,纯一不杂,也明显地属于这一格焉。赋云:“干头相类,铜臭官卑。”因为甲日生人逢乙,乙日生人逢甲,命书叫做“偏禄”,多半没有科名可言。
@天元一1
这种格局天—色纯潸,地支也一色纯清,所以除了行运地支被冲,大抵多贵人命
〔入格八字〕张贵妃
乙酉
乙酉
乙酉
乙酉
此外,庚辰或壬寅逢天元一气,位至三公,大富大贵。然而天元如果都是辛卯的话,那就财多身弱,反而是个身轻福浅的命了。
以上入格八字举例,命书尚有子遥已禄、卯未遥已、刑冲带合、刑合得禄、拱禄拱贵、冲合禄马、虎午奔已、羊击猪蛇、财官双美、福德秀气、青龙伏形、白虎持势、朱雀乘风、还魂借气、金白水清、木火交辉等格,名目不下百余种之多。
由于八字的格,少说也有百种之多,所以纵横开合,千变万化,有的甚至还玄妙百出,使人眼花缴乱,莫测商深。然而在大多数情况下,命理学家论定八字吉凶,还是以八种正格为主,从自身日千五行出发,结合八宇和岁运中五

行盛衰爯忌进行•总体分析,只是在一拽较为特殊清aK,对•少数几种明显入变格的,才以格论《
\section{关于女命的看法}
中国哲学中的阴阳学说,认为女人宠天地的阴柔之气,男人釆天地的阳刚之气,所以把女人说成属阴,男人说成M阳。阴和阳是世间一切事物矛盾对立统一中两个截然不同的面,这种思想反映在命理中是,非但在起运的岁数和大运的推排上,女性和男性有着截然相反的不同,并且在具体的算法上,两者也有着它明显相异的地方。
在书本前面的有关章权丨门得知,男性八字取我克的正财或偏财为妻子,可是女性八字中的丈夫,就要来个彻底的相反,取克我的官(正官)煞(偏官)为丈夫了。同样,在子女的看法上,男命取克我的偏官(七煞)为儿子,正官为女儿,而女命则取我生的食神为儿子,伤官为女儿。
由于封建社会妇人一切都要依赖于丈夫,“夫利其妇必利,夫困其妇必闲”,所以看女命的好坏,先要看夫M官煞的位置和盛衰,以定资贱,接着搏看子星,因为养儿防老,做妇人家的本身没有收入,因此晚年的荣辱,就全押在子星的好坏上了。
在通常W•况下,官、煞、财得地,有利于丈夫,食神得地,有利f孩子。丈火利则出身富货,一生享福;孩子利则晚年
,算
养庠,褒宠诰封。由于食神能够生财,M乂能够生官。比如有位夫人八字日干乙木、乙木所生的食神是r火,然后再由食神丁火生土,木能克土,所以土是乙木的财,接着又由土生出金来,金是克乙木的官,为了这个缘故,所以女命多取食神、财官作为八字的用神。如果八字中官煞财食生得既不得地,又不生旺,或者竟告缺如,行运时又没能够补上的,那就一生困苦潦倒,不用说了。
又如封建礼教崇尚妇人贞洁,从一而终,所以八字中如果见官的就不要见煞,如果见煞的也不要见官,总以一位为好。如果一旦八字中有两位官星,只要没有煞混在里面的,或者四柱里纯是煞,却没有混进官星的,也都是值得称羡的良家妇女。否则官煞相混,就被认为是个喜欢偷情而有外遇的淫荡女子。
在明代育吾山人所著《三命通会》里,曾详细论述了女命的八法。现不妨阐释如下:
①纯所谓纯,就枭纯一的意思。比如官星纯一,煞星纯一,有财(财能生宫)有印(印绶护身),又没遇上刑冲,这就是纯。我们且看下面的一个女命八字,
(年)癸巳(月)戊午(日)辛酉(时)丙申
/字中辛酉为自身,而酉对辛来说,由于正处在临官的禄地,所以自身生旺/古话说:“旺不从化。”按理天干合局,丙辛应该化水,现在本身专綠,.也就化而不化了。这里,辛金
的夫星是克我的芷官丙火,联系该命出生的戊午月,正偯农历五月火旺之时,所以夫星健旺。再联系年干癸水,恰恰与夫星丙火形成正官关系。在用神中,正官是个吉星,所以也对丈夫十分有利。若再联系月干戊土,又是夫星丙火的吉神食神,并且丙火和戊土,又一起归禄(临官)到年柱的地支已里,可谓难得。夫星看后再看子星。辛金生壬水为子,而位居时支子息宫的申里,恰好涵有壬水,而这壬水和申的关系,在十二宫中又正好处在万物发生向荣的长生之地。加之天干癸戊合火,丙辛合水,水火有既济之象;地支巳午酉申,已里庚金,申里庚金,酉里辛金,都是夫星丙火和月支午里丁火的财库,所以也就自然嫁夫为官而食天禄,属于夫荣子贵的命了。
②和所谓和,就是恬静的意思。比郊八字中自身柔弱,只有一位克我的夫星,而四柱又没有攻破冲击的神,这就是®其中和之气的“和”了。让我们看这样一个女命八字:(年)壬辰(月)辛亥(日)己卯(时)己巳
命中日柱天干己土为自身,月柱亥中甲木为夫星。亥对甲木来说,处在万物发生向荣的长生之地,既得天时,又得地利。甲木以辛为正官,现在月逢辛亥,对甲正属有利。己以金为子,时支已中庚金虽为伤官,但是也可活看,况且已对于庚金来说,也处在万物发生向荣的长生之地。以上这些,就叫做夫得官星,子得长生,所以主旺夫益子。至于日柱下卯中乙木,虽为自身己土的七煞,然而有时支已中庚金为制,所以“去煞留官”,为女命中的贵象。

③	清所谓清,就是洁净的意思。女命中或者只有一官,或者只有一煞,不相混杂,这就荪是清了。总要夫星得时,柱中有财有官,有印助身,没有一点混浊之气,方才来得清贵。比如这样一个女命:
(年)己未(月)壬申(日)乙未(时)甲申

自身日柱乙木,以月支时支申中所含庚金为夫星。申对于庚来说,处于临官的禄地,所以夫星得时,又乙木以我生的食神丁火为子星,而自身日支的未中正好含有丁火,而未对于丁来说,也正好处在临官的旺地,所以子星得地。乙木以壬水为正印,而月柱壬水又生于坐下的申金,水源不乏。加之日支未中己土,又为乙木送来偏财。这样财旺生官,四柱又没有刑冲破败。诗云,财官印绶三般物,女命逢之必旺夫。”所以这妇人有两国之封,夫人之命。
④	贵所谓贵,这是尊荣之号。命中有官星,并且得到财气的资生,而四柱中又没有刑冲破败,这就贵为女命中的尧舜了。经云无煞(偏官)女人之命,一贵可作夫人。”又说:“女命无煞逢二德,可二国之封。”所谓二德,不单只指天德,月德(见本书《八字中的神煞》篇),对于女命来说,财也是德,官也是德,如果再有印绶、食神,那就更加尊贵了。这里有这样一个女命:
(年)乙亥(月)丙戌(日)辛卯(时)癸巳
自身日柱天干辛金,不仅以我克的年干乙木为偏财,先获一德,并且又以月干中克我辛金的丙火为官人,而这官人乂坐在以万物成功而藏之的戌的憨痄上面,并且还把时支中的巳文拉来作为临官禄地,所以又得一德。二德之外,时干癸水货为夫星丙火的官,自身辛佥滋生癸水为子,而这为水的癸水恰又偏偏坐在临官的已上,可谐与“夫禄同位”。加之时千癸见日支卯,有天乙贵人之称(见前《八字中的神煞》篇)。这样又是贵人,又是财官双美,所以丈夫和儿子都贵,两遇褒封。
⑤浊所谓“浊”,就是混而不清。女命八字如出现五行失位,水土互伤,自身太旺,代表丈夫的官SM示不出来,而偏宫则一片丛杂,四柱里又没有财官印食,这就都是些下贱村浊,或娼妓婢妾,淫巧的妇人。这里良肴这样-个女命:
(年)己亥(月)乙亥(日)癸5(时)己未
自身癸水,生于十月亥月,太泛。癸水以戊土为正官,现在正官不显,而引时干己土为彳ik、可是日支丑和时支未中都
苻作为偏夫的己土混杂從驭,加之丨-丨柱中没有财,/.木本为癸水的食神,然而乙木生在有力的月干上面,己土受克,这就是五行失位,难免鬼败临身,而主先清后浊,不能享福了。

⑥	滥所谓滥,就是婪的意思。这是说四柱天干里明的有多个夫星,地支里暗的又财旺带煞,这就难免酒色猖狂,私暗得财。碰土此等命的,不是克夫再嫁,就是身为奴婢,因为太过或是不及,从而走向反面。比如这样一命:
(年)庚寅(月)丙戌(日)庚申(时)T:亥
自身庚金,生于秋月,日支又逢临官禄地,本身自旺。其中月柱重于时柱,理应丙火为夫,可是年支和月支寅戌会成火局,时干之上又透出丁火,难免爱火重m。再如自身庚申金暗克年支月支的寅亥木为财,而亥中壬水又为庚金的吉神食神,食神能够生财。因此这个女人虽说长得美貌有福,然而却又少不了属于滥淫而得财的一路。
⑦	娼所谓娼,就是娼妓。八字中如出现身旺夫绝,官衰食盛,或四柱里不见官煞,或有而被作为凶神的伤官伤尽,或官煞混杂而食神盛旺,这些如果不是娼妓的命,就是尼姑婢妾,克夫淫奔的命,两者必居其一,且看这样一命:
(年)丁亥	—
(月)庚戌(日)戊辰(时)庚申
1算
自身日干戊土,本应以年支亥中甲木为克我的夫星,可是由于其木正处在九秋的戌月,在旺相休囚死中属于失时无气的死,现在又逢月干庚金监临,所以必定克绝无疑再看时支申中庚金,理应属于戊土的食神,而申对庚来说又是临官禄地,所以食神有力,加之戊辰原属魁罡星辰,有利男子,不利女子,现在魁罡照临,又能生食,如再结合月干和时干的庚金,就未免使得食神旺过了头。虽说辰里乙木也为克我的夫星,但因位于戊土座下,不能透出,故不能为用。此外年支亥里的壬水,日支辰里的癸水,时支申里的壬水都是自身戊土的财。戊辰本属魁罡,自身强旺,现在克我的夫星既已死绝,而四周又充满一片我生的食神,所以叫做身旺逢生,贪食贪财,是个没有丈夫的秀丽娼妇。
⑧淫所谓淫,就是淫洙。这种人的八字,自身虽然得地,可是四柱夫星太过,明暗交集,人称日干身旺,四柱中都是官煞的便是。夫星出现在天干中的叫明,出现在地支里的叫暗。比如一丁三壬,或丁火同时碰上天干壬水,地支辰中癸水,子中癸水的,就都是四柱太过或明暗交集的典型。这样的女性对于男人,真是无所不纳的了。比如这样一个坤造:
(年)戊辰(月)壬辰(日)壬戌(时)癸亥
命中壬戌、癸亥一个处在临官禄地,一个处在万物成熟的帝胚状态,可谓自身得地。可是在夫星上,明有年柱戊土为正算
夫,暗有二泼一戌所含三个戊土为暗夫,这就属于命书所说的夫星交集,淫不可言了。
除了八法,女命的花样还有旺夫伤子,旺子伤夫,伤夫克子,安静守分,横夭少年,福寿两备,正偏自处,招嫁不定等八格,可谓名FI繁多,应接不暇。但是归纳起来,不外古赋所说的:
若观女命,则异乎男。富贵者一生官旺,纯粹者四柱休囚,浊溢者五行冲旺,娼淫者宫煞交差。
无官多合,此为不良。满柱煞多,不为克制丨印绶多而老无子,伤官旺而幼伤夫。四柱不见夫星,未为贞洁;五行多遇子曜(指食神多),难免荒淫。食神一位逢生旺,招子须当拜圣明;官煞不杂遇印扶,嫁夫定知登云路。守寒房而清洁,金猪木虎(指辛亥、甲寅日)相逢(此二日虽克夫而守正);对空帐而孤眠,土猴火蛇(指戊申、丁巳日)相遇(此二日克夫不正)。财旺生官,辅食无伤,而夫荣子贵;官食禄旺,一印有助,而后宠妃褒。伤官见无财印,败室刑夫•,官煞重逢遇三合(见前《天干地支的刑、冲、杳、化、合》篇),荒淫无耻。合多官重,
•贪淫好色之人•,官杂气衰,嗜欲刑夫之妾。身旺官凶,非师尼而为娼婢;食神变德,先贫贱而后荣华。
此外,在看女命时还盛行着一种克夫的说法。首先,“凡女命,生日在官、鬼、死、墓、绝上,主克夫”。比如丙戌、庚子等日出生的女命,査前《五行的旺相休囚死和寄生十二宫》篇,丙遇戌正处在人之终而归墓的状态,而庚遇子又处在万物死的状态,ra此部k苑夫:然而也w认为,辛復n生的女命,最然逢上绝地,可却“大炎小疵%这就难以••概而论了。犸之,“凡女命,生年生w同=位者,克夫”,“生年生日带六屮者,•名曰带中,主克夫,月共日俱带者亦然”。:举例说,如果中午年生的女命,苒碰上弔午n的生w,那就少不了十有九个要克丈夫。
有趣的是,命书中还多附有一首推算妇女怀孕,生男还是生女的歌诀。对此,《三命通会》记载说:
七'七四十九,问娘何月有,
除却母生年,单奇双是偶,•许男偁女•
•奇偶若不常,寿命不长久。•

根据歌诀,以49为基数,如果母亲岁数是31岁(虚岁),怀孕的月份是农历正月,那末算时49加1(正月)等于50,50再减掉母亲年龄31等于19,19属于单数,所以生男。如果算出来单数应生男,双数应生女的,而生下来的结果却单数生女,双数生男,和这截然相反,那就寿命不长而夭了。然而使人大惑不解的是,有的命书算法还要在末了再加上十九,这就把《三命通会》所载的推算怀孕男女法给彻底颠倒过来了。	

\section{合婚宜忌}
公元1988年7月H日,《新民晚报》所载周柏春《法国I公园相亲》一文的文末;作者写遒,隔天,有人送来了吴小姐的生辰八字》在我家灶上搁迓三天。三天中,家里琬未打碎一只,人未跌过一跤◊据说这预示着吴小姐将为这个家庭带来好运,这里,作者虽然没有进一步谈到男女双方八字的合还是不合,然而这种婚前看®女方八字的做法,却是我国民间、娶妻的一个内容。合婚在古代也叫合姓,就是合3姓为婚姻的意思。由于古代结婚娶妻,双方多半没有机会看	到对方本人,更不要说是了解对方的品德操行和性格脾气了。所以,在合婚过程中,除了周柏春文中提到的一环外,更多的是男方必先要请人看看女方八字是旺夫益子,还是伤夫克子?如者是旺夫益子,则男方兴高彩烈,阖第称庆,若是伤夫克子,则男方必定改辕易辙,另起炉灶。在封建社:	会或旧社会结婚幸与不幸纯凭运气的情况下,这种社会心理是可以理解的6且按下这种做法本身的荒诞无稽不说,举个财说,现在男女相亲,女方送来这样一个八字:

(年)丁丑
(月)壬寅
(日)丁酉
(时)己酉	:	:	
I命中日干丁火为女方自身,用月干克我的丰水为丁火夫星,而月支寅中甲木,既为丁火自身的印,又为夫星壬水的吉神食神。再如儿子寄居的时宫,一是丁火生出己土为子,二是夫ffi壬水得己土为官,三是子星己土得寅中甲木为官,四是丁火兌时支酉为财。综合以上分析,必定是个荣夫益子的命,所以男方高兴还来不及,哪有不满口应承下来的?再如女方如果送来这样一个八宇:
(年)甲辰
(月)癸酉
(日)丙子
(时)辛卯

其中以日柱丙子为自身,按照命书说法,女儿家丙子日出生是犯了阴阳煞,荣有挑诱男子的杨花水性。撇开这不说,而自身丙火既有月干癸水克我为夫,可地支辰子会水,又为暗夫。再如日时干支丙辛相合,子卯相刑,地支刑而天干合,命书认为是荒淫滚浪,酒色昏迷的命。加上丙火克酉中辛金财旺,而这财又正处在夫星座下,所以为卖奸得财无疑。这种淫败的女命,对于合婚的男方来说,又怎么接受得了呢?

由于我国封建社会,是个以男子为中心的社会,故而表现在合婚方面,更多的是男方对女方的挑剔。对此,有首古歌说:
择妇须沉静,细说与君听。
夫星要强健,日干当柔顺。
二德坐正舛,富贵自然来。
四柱带休囚,增命又增寿。

贵人一位正,两三做宠娉。
金水若相逢,必遭美丽容。
四贵一位煞,权家富贵说。
财官若藏库,冲开无不富。
寅申已亥全,孤淫肢便便。
子午并卯酉,定是随人走。
辰戌兼丑未,妇道必大忌。
有辰怕见戌,有戌怕见辰。
辰戌若相见,多是淫破人。
有煞不怕合,无煞却怕合。
合神若是多,非妓亦退歌。
羊刃带伤官,驳杂事多端。
满盘却是印,损子必须定。
天干一字连,孤破祸绵绵。
地支连一字,两度成婚事。

此是妇命诀千金英轻视。
话虽这样,然而反过来说,女方挑选丈夫,对于男方送来八字的研究分析,也是从来不肯马虎的。因为嫁鸡随鸡,嫁犬随犬,毕竟是件牵涉到女孩儿家终身幸福的大事,又怎能草率从事呢?在多数情况下,女孩儿家里对于男方八宇的要求是五行中和,不偏不倚,认为这样的男人不但一生丰衣足食,并且性格中和,寿命绵长。由于古代提倡女子嫁人,从一而终,如果只考虑到男方的荣华富贵,而不考虑到男方的性情脾气和生命寿夭,那往后的日子又怎么过下去呢?

所以,出于以上种种对嫁娶问题的看&和忧虑,命书综括男女合婚的要领是,男家择妇,八字贵看夫子二星,盖夫兴子益,其福必优也。女家择夫,八字I贵得中和之气》盖不偏不倚,其寿必长也,然而,世间男次的八字毕竟千变万化,数目繁多,而又琊来这么多夫荥子货,八字中和的命呢?对此,男女间八字如采有偏的,合婚时互相补偏救弊,转劣为优的学说便就出笼了.泛如男命自身的日干&木,而八字中比病、劫财的甲乙木较:但女方送来八字的自毖偏偏是戊己土,按理木克土,丈夫制约荽子,在封建伦理中最天经地义的班,可是到底因为男方木势太3L难兔中途克荽,所以这时就得看看女方的食神皮辛金如何了。如果食神重的,由于金能制木,因此招架得住,可以合婚;如果女方食神不足,只要戊己土多,能够生金的,也无伤大雅,同样可以合婚;只自身衰弱而又无食神庚金可以抵敌的,那就只得彼此说声再见,重新物色对象了。同样道理,反过来说,如果女命中庚辛金太多,那末找丈夫时最好对方火多,能够克制得住,否则找个木多的也好,因为木多金缺,女方砍伐就费力了。据说,如果按照这种原则配对起来的夫妻,虽然自身八字都有偏胜偏衰,这样那样的不足.•可是由于彼此取长补短,取得了动态的平衡,所以还是能够“琴瑟和谐,子嗣蕃衍3的。把这一男女间彼此救弊补偏的合婚原则归纳起来,就是:“男命木盛宜金者,得女命之刚金补之,则为尽美,得土生金者亦佳,得火者较次,得水木则无取矣。如女命刚金喜火者,得男命之烈火助之,则为尽美,得木生火者亦佳,得水者较次,得金土则无取矣。”其他五行偏盛偏衰的,•可以照此类推。

此外,在合婚中,还有一种用“骨破牌”、“铁扫带”等凶煞来作为避忌的。判定这些凶煞的办法是根据出生年份的地支,结合‘历出生月份的地支来定。例如子年出生的人生在五月(午月),如果是位女命,就被认为是犯了“再嫁”的神煞。在合婚时男方家里如果看到这种女命,就会退避三舍,敬而远之。现把这种合婚避忌列表如下:

流年	子	丑	寅	卯	辰	已	午	未	串	5	戌	亥
神	太岁剑锋伏尸	太由天空	丧门地丧	m	ff•五鬼	死符小耗	大耗	暴败天厄	飞廉白虎		天狗吊客	病符
此外还有一些其他避忌,听来煞是怕人,由于这种对于凶神避忌的办法不仅太简单了,而且还由于碰上的机会太频繁而屡屡失误,所以《命理探原》引西溪逸叟的话驳斥道:“但以人之所生年支硬配月支一字,尤为谬妄。夫以年月日时干枝八字及五行生克,论人吉凶,犹虞不足,岂可弃日时等六字,只论年月二字,即可妄断灾祥乎?”

\section{《金瓶梅》和《红楼梦》里的两次算命}
我国算命术自从五代的徐子平奠定基础以后,经过两宋元朝的氤氳作气,浸润茲延,到了明清之时,早已风靡了整个民间。社会上找人算命的,已经蔚成一种风气。百姓中间,不管是举士应考,商人经商,还是结婚生子,生老病死,都要找人问问算算,是吉是凶。到了这时,算命问卜,实际上已成了民俗中密不可分的一部分了。
对于我国这种算命术的土特产,由于它自始至终打着阴阳五行哲理的旗号,所以广泛地为知识分子所接受。社会上除了那些骗饭吃的专业算命先生外,文人学士会算命的也比比都是。正因为灯命术在文人学士中有着这样的基础,所以又常反映到他们的作品中去。这不仅反映在他们的一些集子或笔记中,并且还亳不例外地反映到一些优秀的小说中去。

《金瓶梅》明人小说中熠熠发光的佼佼者。由于作者学问浩瀚,兼通命相,所以小说里涉及算命看相,占卜问卦的竟有好几处之多。除了给西门庆算命,书中第六十一回黄先生为西门庆娇妾、身患重病的李瓶儿算的那个命,就是个很典型的例子。

连日来,李瓶儿的病愈来愈重,精彩消磨,月水淋漓,六脉沉细,一灵缥渺。一连请了好儿个医生,有的说是重情伤肝,肺火太旺,以致木旺土虚,血热妄行,犹如山崩而不能节制;有的说蹙精冲了血管而起,然后着了气恼,气与血相搏,则血如崩。这样药石乱投,你治你的,他治他的,早已乱了套儿。一天晚上,西门庆娘子吴月娘对西门庆道,你也省可与她药吃,她饮食先阻住了,肚腹中什么儿,只是拿药淘碌他。前者那吴神仙,算她三九上有血光之灾,今年却不整二十七岁了?你还使人寻这吴神仙去,却替他打算,算那禄马数上如何?只怕犯着什么星辰,替他襄保襄保。”西门庆听了,旋差人拿帖儿往周守备府里问去。那里回答,吴神仙云游之人,来去不定,但来,只在城南土地庙下。今岁从四月里,往武当山去了。要打数算命,真武庙外有个黄先生,打的好数,一数只要三钱银子,不上人家门。”西门庆随即使陈敬济佘三钱银子,径到北边真武庙门首黄先生衮门上贴着,妙算先天易数,每命卦金三钱。”陈敬济向前作揖,奉上卦金,说道,有一命,烦先生推算。”写与他八字,女命,二十七岁,正月十五日午时。这黄先生把算子一打,就说:“这个命,辛未年,庚寅月,辛卯日,甲午时,理取印绶之格。借四岁行运。四岁己未,十四岁戊午,二十四岁丁巳,三十四岁丙辰。今年流年丁酉,比肩用事。岁伤日干,计都星照命,又犯丧门五鬼,灾杀作吵。夫计都星者,阴晦之星也,其象犹如乱丝而无头,变异无常。大运逢之,多主暗昧之事;引惹疾病,主正二三七九月,病灾有损,小口凶殃,小人所算,口舌是非,主失财物。或是阴人(女人),大为不利。”抄毕数,敬济拿来家,西门庆正和应伯爵、温秀才坐的,见抄了数来,拿到后边,解说与月娘听。见命中凶多吉少,不觉:眉间带上三黄锁,腹内包藏一肚愁。
这里,李瓶儿八字和行运的情况是:
(年)辛未(月)庚寅(日)辛卯(时)甲午
黄先生认为这八字理应取印绶之格,他虽没有说清原因,想来月支寅中戊土,原是生她自身辛金的印绶,所以就取了这格。
再说流年丁酉,比肩用事。酉属辛金,和自身辛金都是同类的阴干,所以说是比肩用事。至于岁伤日干,就是流年丁火太岁,克伤了日干的辛金。按照命书的说法,岁伤日干,未必就会大祸临头。这里,黄先生把李瓶儿流年说得大不吉利的主要原因莛计都照命,又犯丧门五鬼,灾杀作吵。其中重点发挥了一通对于计都星照命不利的种种理由。原来命书认为》计都是星命家十一星中的-‘星,和罗喉星相对,十八天行一度,十八年行一周天。平时经常隐而不见,碰上日月行次即蚀,所以黄先生才有4夫计都者,阴晦之星也,其象犹如乱丝而无头,变异无常”等不吉的说法。在阴阳中,女人周阴,阴人再碰上这倒霉的阴星,也就难怪李瓶儿最终要一命呜呼了。

清朝人算命,不象明朝那样,把八字和神煞连得紧紧的,因为神煞一般都是硬套的,并且凶多吉少,从而为算命的准确性和灵活性设下了重重障碍。因此,单从本身八字出发结合岁运,论定吉凶,就成了清代命理学家的一大特色。当然,论神煞的也不是说绝对没有,如《红楼梦》曾用薛蝌的话说过:“既有这个神仙算命的,我想哥哥今年什么恶星照命,遭这么横祸?快开八字儿,我给他算去,看有妨害么?”然而和明朝人比较起来,比重要减轻多了。
有趣的是,在《红楼梦》这本封建社会百科全书中,普作者学识的光华,不仅表现在社会伦理、诗词歌赋、政治经济、琴棋书画、文物掌故、饮食烹调、儒学佛道等等多方面,并旦还深刻地表现在对医卜星相等三教九流的无所不通上。他在第八十六回中对于元妃八字的分析,就是很见算命功力的。在那个时代,知识分子探究命理,原是件十分普通的事,我们今天大可不必怕坏了他的声名而故作矜持。相反,触及一下他在这一领域里的探究,反而会使我们感到作

者的形象更加丰岱饱满,更加有血有肉,史加娃个生活中的活生生的一员。
书中宝钗说道:〃不但是外头的讹言舛错,便在家里的/一听见‘娘娘’两个宇,也就都忙了,过后才明白。这两天那府里头这些丫头婆子来说,他们早知道不是咱们家的娘娘3我说:‘你们那里汆得定呢?’他说逍:‘前几年正月,外省荐了一个算命的,说是很准的。老太太叫人将元妃八宇夹在丫头们八宇里头,送出去叫他推荪,他独说:‘这正月初一生口的那位姑娘,只怕时辰错了,不然真是个货人,也不能在这府中。’老爷和众人说:‘不管他错不错,照八字算去,那先生便说:f甲申年,正月丙寅,这四个字内,有伤官、败财,唯申宇内有正宫、禄马,这就趋家里养不住的,也不见什么好。这日子是乙卯,初春木旺,虽是比肩,哪里知道愈比愈好,就象那个好木料,愈经斫削,才成大器,独喜得时上什么辛金为贵,什么已中正官、禄马旺地:这叫作‘飞天禄马格’。又说什么日逢专禄,贵重的很。《天月二德’坐本命,货受椒房之宠。这位姑娘,若时辰准了,定是一位主子娘娘。这不是算准了么?我们还记得说:可惜荣华不久,只怕遇着寅年卯月,这就是比而又比,劫而又劫,慷如好木,太要做玲珑剔透,木质就不坚了他们把这些话都忘了,只管瞎忙。我才想起来,告诉我们大奶奶,今年那里是寅年卯月呢?”
可知,作者在书中给元妃安排的八宇年)甲申	•
〈月)丙K
•一,73一
(日)乙卯(时)辛巳
日柱乙卯是元妃的自身。在寄生十二宫中,卯是乙木的临官禄地,以说“日逢专禄”,这是一种很好的命。再如a辛金为贵”,命书指出,辛见寅为天乙贵人,贵重得很,现在时干和月支配合,就应了这命,如果这种配合在日柱和年、月、时之间,那就更好了。“已中正官、禄马独旺”,是说已中庚金,为日干乙木的正官,已支本身又处在已中丙火的临官禄地状态,加之时支已和日支卯相逢,应了驿马启动的命,所以算命的说元妃的命“真是个贵人,也不能在这府中'
那末不在这府中,又怎么料定非要受宫中椒房之宠呢?这是因为“天月二德坐本命”的缘故。这里,宝钗口中所说“天月二德坐本命”和命书里排定的天德、月德有所出入,看来当是指的“归禄逢二德”了。	-
至于所说的“飞天禄马格”,《喜忌篇》有云:“若逢伤官月建,如凶处未必为凶,内有倒禄飞冲。”元春生于乙卯日,乙为阳木,以庚金为官星,而月上丙火能克庚金,这就成了“伤官月建”。乙日既得丙火,又生在初寅木之月,日支上的卯木便可冲出辛时所含的申金,“倒禄飞冲'便成了“飞天禄马”格。
且丢开这些不说,无论如何,作者在这里用了一定童的篇幅,借宝钗的口转述了算命先生对命理的一番分析,说明他对命书有过兴趣,有过研究,则是肯定无疑的。更不要说他在书中所说,可惜荥华不久,只怕遇着寅年卯月,这就比174
而又比,劫而又劫,臀如好木,太要做玲珑剔透,木质就不坚了”的这一段话,还又是十分在行的呢。
《金瓶梅》、《¥楼梦》之外,明清小说中有关算命的比比都是,如极为著名的,就有吴敬梓《儒林外史》第五十四回《病佳人青楼算命,呆名士妓馆献诗》的有关算命的描述。文中,作者通过弹三弦瞎子为青楼女聘娘算命和陈木南和瞎子之间的谈话,从一个侧面反映了当时社会算命风气之盛和作者对算命术的了解。
I
\section{古代名人八字举要}
这是一个很有趣的课题,可供批判性研究。在袁树珊《命谱》里,他曾为诸葛亮算命道:
诸葛武侯相后汉灵奈光和四年七月二十三日已时生
〔命造八字〕	大运
(年)	辛酉辛金偏印,庚金正印		三岁乙未
(月)	丙申■	壬水劫财戊土正官癸水比肩	十三甲午
(日)	癸丑•	{辛金偏印己土偏官庚金正印	二十三癸巳
(时)	丁巳	{内火正财k土正官	三十三壬辰
乙木食神	四十三辛卯
(命宫)壬辰1戊土正官
、癸水比府	五十三庚寅
日元癸水,诞生立秋节后,白帝司权.金正当令,水得金生,正气充足,再逢年干辛金,年枝西金,及月枝申藏庚金,又藏壬水,日枝丑藏辛金,又藏癸水,卺香生之助之,其为金白水济,显而易见。仅恃月干单独丙火,不独不能制金,且亦不敷济水之用,况丙与幸合,同化为水,其火之成分,又复若有若无/没有生时丁巳之二火,决不能制当令之瓯金,济有余之相水。今既得此为正式之用神,其为雨旸时,若天地颐成可知。
接着作者笔锋一转,太致认为诸葛亮大运二十三岁后金水连环,和用神火背道而驰,虽说鞠躬尽力,也只能够事倍功半。五十四岁大运庚,流年甲寅,岁支寅和命中月支申相冲,与时支已相刑,所以一旦当生命进入当年八月癸西,二十日庚辰,金水汹浦,助纣为虐之时,也就难逃厄运,卒于军中了。
按下诸葛亮八字不表,我们这里再罗列古人八字一束•
稍作提示性分析,举要而已。
〔孔子〕
(年)庚戌(月)戊子(日)庚子(时)甲申
自身庚金,归禄在时支中中,并且时干甲木为庚金偏财,本•
算
属难得,借月文T/1,寒水当令,虽然金白水清,然而未免寒俭,况年支戌中官星丁火偏处一隅/旁受月支癸水制约,难以发挥。纵观孔子一生奔波,劳而无功,政界失意,直到晚年杏坛设教,弟子三千,说明金水流通,从次却好。孔子生于公元前551年庚戌,关于他的八字原有多种说法,录此以备一格。
〔关羽〕
(年)戊午(月)戊午(日)戊午(时)戊午
《三命通》说:“戊午日,戊午时,先刑后发,多不善终。”又说:“纯午,武职威权,名重藩镇。”基本和关羽的一生吻合C此外在格局中,这是一种a天元一气”的格局,又名凤凰池。据说,张飞的八字是“癸亥、癸亥、癸亥、癸亥”四个癸亥,一片铺天盖地的癸水,和关羽的一片火土正好来个一百八十度的大相反。火土红黄,癸水纯黑,小说中关羽面如重枣,张飞面如黝漆,大概和这不无关系。査史籍记载,有关关羽、张飞的生年,今已难以考知,但是对于他们的八字,命书却一直这样记载,这就使人费解了。
〔吕洞宾〕
(年)丙子(月)癸巳(日)辛巳(时)癸巳
算
这一八字,自身日干辛金,四柱中火多刑金,好在已中戊土生我,庚金助我,而年支、月干、时干又复一片癸水,有金白水清,水火既济之象。加之时柱癸巳,天乙贵人贴身而居,所以并不一般。今査吕洞宾的生平,生于公元798宇戊寅,卒年不知,传为唐代京兆人,后来被道教全真道尊为北五祖之一。可见命书所载这一命造,可靠性是很成问题的。
(邵雍〕
(年)辛亥(月)辛丑(日)甲子(时)甲戌
日元甲木,生于季冬丑月,时支逢戌,这是移根换叶,甲木逢养的迹象,属于学界一流的命备。邵雍是北宋著名哲学家和道学家,著作有《皇极经世》、《伊川击壤集》等。
〔蔡京〕
(年)丁亥(月)壬寅(日)壬辰(时)辛亥
从这命造的日柱看,既属于壬骑龙背的格局,又可属于魁罡的格局,所以命主生平潑好。然而,《三命通会》却又认为,“壬辰日,辛亥时,秀贵,恶死”。蔡京本是北宋权臣,后来金兵攻宋,他带着全家仓皇南被钦宗下令放逐岭南,结果在途中死于潭州(今湖南长沙)。据说那时京城里有个孩子的八字,也和蔡京生得一模一样,可他却只在十岁就淹死
算
To
〔贾似道〕	:	虼,
(年)癸酉	:
(月)庚申心	。:
(日)丙子(时)丙申
《三命通会》说:“丙子日,丙申时,若通火气及寅、卯月,再行身旺运,吉。年月纯金,弃命就财,亦以吉论贾似道为南宋权相,命书说他“奸臣”。南宋末年,元军沿江东下,他率兵抵抗,兵败革职,在放逐途中,被监送人郑虎臣所杀。
〔元世祖〕
(年)乙亥
•(月)乙酉	♦
(日)乙酉(时)乙酉
这一格局年、月、日、时,天干全是乙木,纯一不杂。按照命书说法,这是一种“干辰一字”的格局,属于大贵的命。‘元世祖名忽必烈,为元代的开貝皇帝,一生龙振虎威,功业十分显赫。
〔赵孟颊〕
(年)甲寅0!)甲戌(PO己酉(时)己巳
这一命造,时逢己巳,属于金神格局。金神原为破败之神,
算
要制伐入火乡为胜”,现在月支戌中;r火,年支寅中丙火一起制伐,加上日干己土遇印生我,比I&助我,所以一旦甲木制我,就格局平衡了。赵孟颊本是宋代宗室,入元后元世祖忽必烈搜访遗逸,经程钜夫荐举,官刑部主事,后又累官至翰林学士承旨,封魏国公,谥文敏。在艺术上,他的书画几乎一手笼軍了整个元代的天下,腐人学他的很多c明太祖〕
(年)戊辰(月)•壬戌(日)丁丑(时)丁未
这一命造,如果年、月地支不见辰、戌,单是日、时地支丑、未刑冲,大有不得善终的忧虑,妙在现在年、月、日、时的地支,辰、戌、丑、未四库一应俱全,这就非但无忧,并且贵为天子了。据说明太祖朱元璋登基以后,听到天下也有一个人和他的八字相同,这就使他大为忧虑,动了杀心。后来召来一瀞,原来是洛阳地区一个姓李的穷老头儿。朱元璋问他干什么活,,他说:“老民养蜂十三窠,以之度日•”朱元璋听后宽了口气•“这和我国家享有十三省布政司的税收正好一样,把十三省税收和十三窠蜂相比,除了表面数字相同,可•实质上却有着筲壤之别
〔张居正〕
(年)乙西(月)辛巳(日)辛酉
(时)辛卯
这一命造,按照《三命通会》说法:“辛酉日、辛酉时,出身孤苦,中年获福,末年封妻荫子,贵。”张居正是明朝有名的政治家,在他入阁当国的十年间,推行一条鞭法,颇有政绩。公元1582年壬午张居正虚龄五十七岁,这一年,大运丙子,流年壬午,岁和运子午相冲,张居正死。
〔戚继光〕
(年)戊子
(月)癸亥	v.
(日)己巳(时)乙亥
这一命造,日主己巳,虽然金神位®不在时上,未能作金神格局看,可是月干偏财,时干偏官,自身又得戊土之助,所以扶抑得宜。加之地支子亥年月,所以《三命通会》认为,“以财党煞,作弃哲,兵权1	〖UC
〔明神宗皇后〕	、
(年)甲子	•	'
(月)乙亥	、
(日)癸酉(时)壬子
命造中亥遇乙为天德,亥遇甲为月德,这天、月二德即使没有聚在身上,可菇托父祖荫庇,也贵为皇后

\section{无师自通的秤骨算命法}
出生年、月、日、时的秤骨份置	k
在旧时算命法中,有一种托名为唐代命相学家袁天罡先师的秤骨法。这种方法,只要对照一个人农历出,生年、月、日、时,分别査得年、月、日、时的称骨份童,然后再把这些份童汇总起来,便可在秤骨歌上找到有关自己一生荣枯的断语了。因为这种方法简便易行,所以有“无师自通”或“算命不求人”的说法。
1.出生年份六十花甲秤骨份量	
①〔甲子(鼠)〕	一两二钱
②〔乙丑(牛)〕	九钱
③〔丙寅(虎)〕	六钱
④〔丁卯(兔)〕	七钱
⑤〔戊辰(龙)〕	一两二钱
⑥〔己巳(蛇)〕	五钱
⑦〔庚午(马)〕	九钱:
⑧〔辛未(羊)〕	八钱
⑨〔壬申(猴)〕	七钱
⑩〔癸酉(鸡)〕	八钱
⑪〔甲戌(犬)〕	一两五钱
算
⑫〔乙亥(猪)〕九钱⑬〔丙子(鼠)〕一两六钱⑭〔丁丑(牛)〕八钱⑮〔戊寅(虎)〕八钱⑯〔B卯(兔)〕一两九钱⑫〔庚辰(龙)〕一两二钱⑬£辛巳(蛇)〕六钱⑲〔壬午(马)〕八钱⑳〔癸未(羊)〕七钱㉑.〔甲申(猴):)五钱@〔乙酉(鸡)〕一两五钱⑳〔丙戌(犬)〕六钱⑭〔丁亥(猪)〕一两六钱⑳〔戊子(鼠)〕一两五钱⑳〔己丑(牛)〕七钱@〔庚寅(虎)〕九钱⑬〔辛卯(兔)〕一两二钱⑳〔壬辰(龙)〕一两⑳〔癸巳(蛇)〕七钱㉛〔甲午(马)〕一两五钱©〔乙未(羊)〕六钱@〔丙中(猴)〕五钱©〔丁酉(鸡)〕两四钱@〔戊戌(犬)〕一两四钱@〔己亥(猪)〕九钱
@f庚子(鼠)〕	七钱
@〔辛丑(牛)〕	七钱
•㉝〔壬寅(虎)〕	九钱
@〔癸卯(兔)〕	一两二钱
⑪〔甲辰(龙)〕'	八钱
@〔乙巳(蛇)〕	七钱
⑬〔丙午(马)〕	•两5钱
⑭〔丁未(羊)〕	五钱
磕〔戊申(猴)〕	一两四钱
@斗己酉(鸡)〕	五饯
®〔庚戌(犬)〕	九钱
@〔辛亥(猪力	一两七钱
⑬〔壬子(鼠)〕	五饯
⑳〔癸丑(牛)〕	七钱
©〔甲寅(虎)〕	一两二钱
@〔乙卯(兔)〕	八钱
⑬〔丙辰(龙)〕	八钱
•@〔丁巳(蛇)〕	六钱
®〔戊午(马)〕	一两九钱
®〔己考(羊)〕	六钱
®〔庚申(猴)〕	八钱
⑯〔辛酉(鸡)〕	一两六钱
®〔壬戌(犬)〕	一两
⑯〔癸亥(猪)〕	六钱
I出生月份秤份fit
184—
①	〔正月〕六钱
②	〔二月〕七钱
③	〔三月〕一两八钱
④	〔四月〕九钱
⑤	〔五月〕五钱
⑥	〔六月〕一两六钱
⑦	〔七月〕九钱
⑧	〔八月〕一两五钱
⑨	〔九月〕一两八钱⑯〔十月〕八钱
⑪〔十一月〕九钱⑫〔十二月〕五钱
3.出生日期秤骨份
①	〔初一〕五钱
②	〔初二〕一两
③	〔初三〕八钱
④	〔初四〕一两五钱
⑤	〔初五〕一两六钱
⑥	〔初六〕一两五钱
⑦	〔初七〕八钱
⑧	〔初八〕一两六钱
⑨	〔初九〕八钱
⑩	〔初十〕一两六钱⑪〔十一〕九钱
⑫〔十二〕一两七钱
⑬〔十三〕	八钱
⑭〔十四〕	一两七钱
⑮〔十五〕	一两
⑯〔十六〕	八钱
⑫〔十七〕	九钱
⑬〔十八〕	一两八钱,
⑩〔十九〕	五钱
⑳〔二十〕	一两五钱
㉑〔二十一〕	一两
©〔二十二〕	九钱
@〔二十三〕	八钱
⑭〔二十四〕	九钱
⑮〔二十五〕	一两五钱
⑳〔二十六〕	一两八钱
©〔二十七〕	七钱
⑳〔二十八〕	八钱
@〔二十九〕	一两/、钱
⑳〔三十〕	六钱
4.出生时辰秤骨份S	.【:.,
①	〔子时(23—1时以前)〕一两六钱
②	〔丑时(1一3时以前)〕六钱
③	〔寅时(3—5时以前)〕七钱、
④	〔卯时(5—7时以前)〕一两
⑤	〔辰时(7—9时以前)〕九钱
⑥	〔已时(9一11时以前)〕一两六钱
186—
.⑦〔午时(11-13时以前)〕一两
⑧	〔杀时(1315时以前)〕八饯
⑨	〔申时(15-17时以前)〕八钱
⑩	〔西时(17-19时以前)〕九钱
⑪〔戌时(19-21时以前)〕六钱
纊
⑫〔亥时(21-23时以前)〕六钱
这里要注意的是,时辰如果逢正1点的,就算下••个时辰的丑时,正3点的,就猝下一个时辰的寅时,正五点的,就算下一个时辰卯时,其他类推。若逢日时制出生的,就提前一小时算。
通过自已出生的年、月、日、时,把从上表中分别奄出的秤骨份M总数加起来,然后冉把加起来份M的总数,对照下面列举的秤骨歌,就可轻而易举地揭晓每个人一生命运的穷通了。
秤骨歌
在对照:-个人农历出生的年、月、日、时的秤骨份M,并把它们一一汇总之后,我们就可根据下列份M•轻重,按图索骧地找出与每个人相关的秤骨歌诀。
〔二两二钱〕
身寒骨冷苦仃伶,此命推来行乞人。
碌碌巴巴无度日,终年打拱过乎生c
〔二两三钱〕
此命推来骨自轻,求谋作事爭难成。
妻儿兄弟应难许,别处他乡作散人〔二两四钱〕
此命推来福禄无,门庭困苦总难荣。六亲骨肉皆无靠,流到他乡作老翁。〔二两五钱〕
此命推详祖业微,门庭营度似稀奇。六亲骨肉如冰炭,一世助劳自把持。[二两六钱〕
平生衣禄苦中求,独自营谋事不休。离祖出门宜早计,晚年衣禄庶无忧。〔二两七钱〕
一生作事少商量,难靠祖宗作主张。
匹马单枪空做去,早年晚岁总无长。〔二两八钱〕
一生行事似飘蓬,祖宗产业在梦中。若不过房.改名姓,也当移徙二三通。〔二两九钱〕
初年运限未曾亨,纵有功名在后成。须到四旬方可立,移居改姓始为良。〔三两〕
劳劳碌碌苦中求,东奔西走何日休?若系终身勤与俭,老来稍可免忧愁。两一钱〕
忙忙碌碌苦屮求,何U云开见日头?.难得祖基家可立,中年衣食渐能周。,
[三两二钱;I
初年运蹇事难谋,渐有财源如水流c到得中年衣食旺,那时名利一齐收。〔三两三钱〕
早年作事事难成,百计勤劳枉用心。半世自如流水去,后来运至得黄金。•〔三两四钱;)	…
此命福气果如何?僧道门中衣禄多。离祖出家方为妙,终年拜佛念弥陀。〔三两五钱〕
生平福量不周全,祖业根基觉少传。营事生涯宜守旧,时来衣食胜从前。〔三两六钱〕
不须劳碌过平生,独自成家福不轻。早有福星常照命,任君行去百般成。〔三两七钱〕
此命般般事不成,弟兄少力自菰行。虽然祖业须微有,来得明时去不明。
C三两八钱〕
一身骨肉最濟高,早入黉门姓氏标。
待到年将三十六,蓝衫脱去换红袍。〔三两九钱〕
此命终身运不通,劳劳作事尽皆空。苦心竭力成家计,到得那时在梦中。〔四两〕
夺生衣禄是缔长,件件心中自主张:。前面风霜多受过,后来必定享安康《〔四两一钱〕
此命推来自不同,为人能千异凡庸。中年还有逍遥福,不比前时运未通。
L四两二钱〕
得宽怀处且宽怀,何用双眉皱不开。若使中年命运济,那时名利一齐来。
(四两三钱〕
为人心性最聪明,作事轩昂近贵人。衣禄一生天数定,不须劳碌是丰亨。(四两四钱〕
万事由天莫苦求,须知福禄赖人修。当年财帛难如意,晚景欣然便不忧。〔四两五钱〕•
名利推求竟若何?前番辛苦后奔波。命中难养男和女,骨肉扶持也不多。〔四两六钱〕
东西南北尽皆通,出姓移居更觉隆。衣禄无穷天数定,中年晚景一般同。〔吗两七钱〕
此命推求旺末年,妻荣子贵自怡然。平生源有滔滔福,可卜财源若水泉。〔四两八钱〕	'
初年运道未普通,几许蹉跎命亦穷《•
190—
兄弟六亲无依靠,一生事业晚来隆。〔四两九钱〕
此命推来福不轻,目成自立显门庭。从来富贵人钦敬,使婢差奴过一生。〔五两〕
为利为名终日劳,中年福禄也多遭。老来自有财星照,不比前番目下高。〔五两一钱〕
一世荣华事事通,不须劳碌自亨通。弟兄叔侄皆如意,家业成时福禄宏。〔五两二钱y
一世亨通事寧能,不须劳苦自然宁。宗族有光欣喜甚,家产丰盈自称心。〔五两三钱〕
此格推来福泽宏,兴家立业在其中。一生衣食安排定,却是人间一富翁。〔五两四钱〕
此格详来福泽宏,诗书满腹看功成。丰衣足食多安稳,正是人间有福人。〔五两五钱〕
走马扬鞭���利���,少年作事费评论。一朝福禄源湃至,富贵荣华显六亲。〔五两六钱〕
此格推来礼义通,一身福禄用无穷。甜酸苦辣皆尝过,滚滚分源稳而丰。
〔五两七钱〕
福樣丰盈万事全,一身荣耀乐天年。名扬威震人争羡,处世逍遥宛似油。〔五两八钱〕
平生衣食自然来,名利双全窝贵偕。金榜题名登甲第,紫袍玉带走金阶。〔五两九钱〕
细推此格秀而清,必亨才髙学业成。甲第之中应有分,扬鞭走马显威荣。〔六两〕	,
一朝金榜快题名,显祖荣宗大器成。衣禄定然无欠缺,田园财帛更丰盈。〔六两一钱〕
不作朝中金榜客,定为世上大财翁。聪明天付经书熟,名显高科自是荣。〔六两二钱〕
此命生来福不穷,读书必定显亲宗。紫衣金带为卿相,窗贵荣华孰与同?
C六两三钱〕
命主为官福禄长,得来富贵&非常。名题雁塔传金榜,大显门庭天下扬。〔六两四钱〕
此格威权不可当,紫袍金带坐高堂。荣华富贵谁能及?万古留名姓氏扬。〔六两五钱〕
192—
细推此命福非轻,富贵荣华孰与争?
定国安邦人极品,威声S赫®寰瀛。f六两六钱〕
此格人间一福人,堆金积玉满堂春,
从来富贵由天定,金榜题名更显亲。
〔六两七钱〕
此命生来福自宏,田园家业最高爸。
乎生衣禄盈丰足,一路荣华万事通。
〔六两八钱〕	.
窗贤甴天莫苦求,万金家计不须谋a十年不比前卓,祖业根基千古留。
〔六两九钱〕
君是人间衣禄星,一生富贵众人钦。
总然福禄由天定,安享荣华过一生。
〔七两〕
此命推来福不轻,何须愁虑苦劳心。
荣华富贵已天定,正笏垂绅拜紫廣。
〔七两一钱〕
此命生成大不同,公侯卿相在其中。
一生自有逍遥福,富贵荣华极品隆。
从以上秤骨算命法看,难免简单粗糙。然而有时为了多一蜇参考依据,算命先生常常喜欢把八字算命和秤骨算命合在一起推断。据说这样相互参照,•可以取长补短。而民间则因为这种算法简便而容易掌握,所以一直流传不歇。
第四章算命术的批判

\section{墨子的“非命”观	}
所调“非命”,就是否定、反对世界上有所谓夭命的-•种观点。作为学说的一种,“非命”观在先秦诸子的墨家学派中,也和“兼爱”、“非攻”一样,同样占着十分重要的地位。可以这样说,在当时一片弥搜着天命观的混浊空气中,墨家学派打出的这一旗帜鲜明的观点,无疑为当时的学坛和意识形态领域,吹进了一阵醒人耳目的清风。
《墨子》一书,《非命》共有上、中、下三篇,文中集中体现了墨家代表人物墨翟对于“天命”观的批判,是我国古代反对天命论的辉煌篇章C
在《非命》上篇中,墨子说道,古代治理国家的王公大人,都希望国家窗足,人民众多,政局安定,然而他们所得到的,不是窗足而是贫困,不是众多而是稀少,不是安定而是
祸乱。这就是说,实际上他们没有得到原先所希望得到的,而是得到了他们原先所不希望得到的,这是什么原因呢7
算
回答是“执(主张)有命(命运)者”混在民间的太多了。这些“执W命者y认为:“命里注定®足就宵足,命里注走贫困就贫困;命里注定人多就人多,命里注定人少就人少;命里注定安定就安定,命里注定祸乱就祸乱;命里注定长寿就长寿,命里注定短命就短命。即使你使出多大的力气,又有什I	么用处呢?”他们既把这一套向上兜错给王公大人,又向下
影响了百姓干活的积极性。所以,“执有命者”是不仁的。对于他们这些惑乱视听的言论,不可不辨个彻底的水落石I出。
那末,怎样才能辨个彻底的水落布出呢?墨子的说法是,立论必定先要有个标准。立论如果没有标准,就好象在运转着的制陶转轮上去辨别方向,是怎么也辨不清楚的。正因为这样,所以墨子提出了立论一定要有“三表”(三项标准)的原则。什么叫“三表”?一指推究本源,二逛指弄清过程,三是指检验实践。怎样才能推究本源呢?这就要上推古代圣王的事迹了。怎样才能弄清过程呢?这就要下考百姓耳闻目睹的实情。怎样才能检验实践呢?这就要在刑政实施中检验是不是符合国家和人民的利益。
如今天下士君子中,认为有命运的,何不往上观察一下圣王的事迹呢?古代夏桀搞乱了天下,商汤把它接过来治理好了;商纣搞乱了天下,周武王把它接过来治理好了。这期间社会没有改变,老百姓没有变换,由桀、纣统治则天下大乱,由商汤、周武王统治却天下大治,这难道可以说是归之于命运吗?
如今天下士君子中,认为有命运的,何不往上翻猗一下
算
先王的典籍?先王的典籍,原是国家定出来,公布施行到百姓中去的一种有关宪制/先王的宪制,何曾说过“幸福不可求得而灾祸不可避免,ft没奋好处而凶残没有害处”的话?先王用来审判案件制裁犯罪的,是国家的刑律。先王的刑律,又何曾说过“幸福不可求得而灾祸不可避免,善良没有好处而凶残没有害处”的话?先王用来整治军队,指挥军队进退的,‘先王的军令。先王的军令,又何曾有过“幸福不可求得而灾祸不可避免,莕良没有好处而凶残没有害处”的话?所以墨子说道:我们还没有完全统计过天底下的好书,即使统计起来也统计不完,可蒞从大的方面来狩,雄本就数宪制、刑律、军令这三个大&了。现在只要一希那些“执天命者”的言论,都是些古代先王與籍里找不到的,这不明摆着可以丢弃了吗?二看那些“执天命者”的言论,是违背天下道义的。而又正是那些违背天下道义的言论,使得百姓困苦不堪而不能自拔。把百姓弄得困苦不堪而不能自拔当作自己快乐的,就是残畨天下的人。
再看,人们为什么要坚守正义的人治理国家呢?回答是义人在上,天下必治,上帝山川鬼神有了正统的继承人,万民就会得到莫大的好处。怎么才能证实这一点呢?古代商汤封在亳邑(今河南省商丘县),这地方长短大小总计起来,不过百里见方的土地,可是商汤却能和百姓“兼相爱,交相利,侈(多余)则分”,率领百姓尊敬上天,事奉鬼神,这样上天鬼神就使商汤的天下富裕起来,结果诸侯归附,百姓亲近,贤士投奔,没有终结他的一生就称王天下,做了诸侯的头头。又如古代周文王封在岐周(今陕西省岐山县),这地
算
方让短大小总计起来.不过百里见的土地,可是周文王却和百姓“兼相爱,交相利,侈则分”,所以住在近处的百姓乐意受他的统治,住在远处的百姓也听说他的德政而竟相ik附。当时只要听到周文王名字的,不K人们都会立即起身投奔到他那里,就是连体弱病或的,也都会守候在自己的住处热切盼望说:“要是文王的土地扩展到我们这里,那末我们岂不也成了文王的百姓?”也就因为这个原因,上天鬼神就使文王的天下富裕起来,结果诸侯归附,百姓亲近,贤士投奔,没有终结他的一生就称王天下,做了诸侯的头头。刚才我们不是说过:“义人在上',天下必治,上帝山川鬼神有了正统的继承人,万民就会得到莫大的好处。”这结论就是根据这些事实推论出来的纟
所以,古代圣王制定法律颁布政令,设立奖惩条例,原是用来鼓励好人,制约坏人的。刑政赏罚明百姓们在家就孝顺父母,出门就埤敬师长,进进出出都有一定的规矩礼节,男男女女都不杂处。国家如果派这样的人治理官府就不会偷盗,守护城池就不会背叛,国君有难就皙死保卫,国君流亡就跟随护送。而这些美德,正是国君赞赏,百姓称誉的。可是对于这些,“执天命者”却说:“君王要赞赏,是这些人命里本来就该得到的,并不是因为做了好事才被赏的。”在这种思想支配下,有些人可以在家不孝父母,出门不敬师长,进进出出都不讲规矩礼节,男男女女都杂在一起。国家如果派这样的人去治理官府就会偷盗,守护城池就会背叛,国君有难就不肯死节,国君流亡就不肯护送。而这些劣迹,却正是国君惩戒,W姓责备的。可是对于这种劣迹,“执天
算
命者”又会说:“国君要惩罚,是这些人命里本该招致的,并不是因为劣迹斑斑才被惩罚的。”在这种思想支配下,为君的就可以不守正义,为臣的就可以不忠国君,为父的就可以不爱护子女,为子女的就可以不孝顺父母,为兄的就可以不关心弟弟,为弟的就可以不尊敬兄长。所以那些“执天命者”,简直就是一切谬论和劣迹的制造者。
那么,怎么说明“执天命者”是一切谬论和劣迹的制造者呢?且看上古时候那些不开化的百姓,吃起来贪心不足,干起活来却偷懒得很,所以吃的和穿的都告匮缺,这就难免担心自己受饿挨冻。可是对于这种受饿挨冻的简单原理,他们非但不从“我这个人太懒惰不中用,千活不得力”去考虑,反而坚持认为我的命本来就是个受饿挨冻的命”。再看上古时候那些暴虐的君王,既不克制他们耳目声色的欲望和心里头的邪念,又不孝顺他们的父母,这就难免最终导致国破家亡。可是对于这种国破家亡的简单原理,他们非但不从“我这个人太懒惰不中用,不善于处理国政家事”去考虑,反而坚持认为:“我的命本来就是个要国破家亡的命,《书经•仲虺之告》中说:“我听说a朝人假托天命,发布命令于天下,于是上帝便就怒而讨伐他们的罪行,因此夏朝就失掉了他们的军队。”这是商汤在否定夏桀天命观时所说的话。又如《书经•太誓》中说:“商纣在平时不肯事奉上帝鬼神,丢开他们的祖先神祇不去祭祀,竟说:‘我有好命,不必尽力做事Z这样上帝也就放弃了商纣而不去保佑他了。”这些,又是周武王在否定商纣王天命观时所说的话。
眼下,由于这些“执天命者”的言论,弄得国君不治理国
算
家,百姓不好好干活。国君不治理国家就政局混乱,百姓不好好干活就财用不足,结果弄得上对上帝鬼神没有米饭甜酒可供祭祀.•下对天下贤人达士无法收容安抚,外对诸侯宾客不能应接招待,内对贫民百姓无以充饥御寒,更不要说是将息调养老弱病残了。所以天命观“上不利于天,中不利于鬼,下不利于人”。这样“执天命者”不简直就是一切谬论和劣迹的制造者?
所以最后墨子在篇末总结道:“今天下之士君子,忠实(真心实意)欲天下之富而恶其贫,欲天下之治而恶其乱,执有命者之言不可不非,此天下之大害也。”
文中,墨翟通过推究本源,弄清过程,检验实践的“三表”原则,采用层层列举事实,步步讲清道理的方法,有力地驳斥了“执有命者之言”对于天下的严重危害性,可谓入木三分。这里面虽然也有由于时代对作者所造成的局限,使得墨翟在驳斥反对万事命萣的同时,又搬出了天帝鬼神的一套,可是无论如何,这在当时一片烟障雾隔,信命如狂的社会风气中,其文章思想的光辉瑰奇,却是无可置疑的。
\section{古人都相信算命吗?}
自从唐李虚中发明用年、月、日三柱和徐子平奠定用年、月、日、时四柱推算命理以来,一时算命术大行天下,学者风从。就以那位为李虚中作墓志的大文豪韩愈来说,就
-199-
是个十分信命的人。此后宋元明清,学者君子,信命的代不乏人,更不要说是一般民间的平民百姓了。
明代张瀚曾在《松窗梦语》卷六中说,有一年,作者有个名叫。孙子泉的朋友,遨乡人同來一起饮酒。席间,孙季泉一个个地询问各人出生的年月日时,心里暗暗推算,可就是:不出声。后来洒酣席散,孙季泉暗里拉作者到一旁说我和你为同年友,现在只有我们两人。我看你中年运限不利,然而不知到底怎样,现在再仔细为你推算一下。”算后,孙季泉很有把握地说:“中年虽然运行西方,只是宦途淹滞不利,对身家性命却没有多大影响。行过西方金运,进入南方火运,那就豁然通达了。”当年孙季泶高中一甲,作者中二甲。后来孙季泉官至宗伯的髙位,过了十几年后,作者也爬上了冢宰的髙位。至此,作者不禁深深感叹夫以数十年之迟速显晦,决于八字之间,公之楮于术数如此!”
《玉堂丛语》是明朝学者焦竑的一本笔记杂著。书中卷七,作者说了这样一件事,提学萧鸣凤楮于子平之术,正德丁丑年间廷试,有人拿了好多考生八字前来求教说请你算算这次廷试,谁是状元?”萧鸣风把各人八字一一看过后说:“这位舒梓溪可以尚中廷试第一,状元爷的桂冠他摘定了。”结果廷试下来,果然应验了他的话。..
类似于以上记载的,在明清人的著述中,真是难以一一枚举,可以作为古人信命的铁证。可是,是不是古人都信命呢?答案自然是否定的。
一个人生活中贫富寿夭的遛遇,原要受着历史、社会、政治、文化、境遇等等多种因索的制约,而命理学家撇开这
-200—
些不谈,大谈其命,&然是荒谬而站不住脚的对此,在不相信命的古人中,除了春秋战国时期《墨子•书打出C非命》三篇,对儒家的天命观念作无愤批驳的那位墨家学派创始人墨翟外,唐太宗时著名哲学家吕才也是其中一个。吕才在他所著的《荪命篇》中说:“汶宋忠、贾谊讥司马季主(占卜术士)曰"卜筮者高人禄命(说好别人的命),以悦人心,矫言祸描,以规(图谋)人财。’王充曰:‘见骨体,知命禄,见命禄,知骨体。’此则言禄命尚矣。推索本原,固其不然。积善之家,•必有余庆,岂建禄(命中有禄)而后吉乎?积恶之家,必有余殃,岂劫杀(命中有劫财、七杀)而后灾乎?”接着他又举例说:“文王忧勤损寿,非初值空亡(一种凶煞);长平坑降卒,非俱犯三刑(被秦坑死的四十万赵国降卒,并不都是命里犯了三刑);南阳(汉光武帝)多近亲,非俱当六合(命中地支),与此同时,他还详论锊庄公的命说,如果按照他出生的年、月、日去笄,应该是个穷贱的命,可实际上,他却当上了一国之君。有意味的是,偏偏这个说命不准的人,却是个历史上对阴阳、舆地有着极深研究的专家。在《旧唐书》本传中,至今还保存着他的《叙宅经》、《叙禄命》、《叙葬书》等多篇。看来,大概正因为他曾深入这个营垒,所以才能反戈一击,致强敌于死命。
赵宋之时,有费衮著《梁溪溲志》十卷。在书的第九卷中,有《谈命》一则说:“近世士大夫多喜谈命,往往自能推步,有精绝者。予尝见人言,日者阅人命,盖未始见年月R时同者,纵有一二,必唱言于人以为异。尝略计之,若时无同者,则一时(时辰)生一人(一种),一日当生十二人;以岁计之,四千三百二十人;以一甲子计之,止有二十五万九千二百人而已。今只以一大郡计,其户口之数尚不减数十万,况举天下之大,自王公大人以至小民,何啻亿兆?虽明于数者,有不能历算,则生时同者,必不为少矣。其间王公大人始生之时,则必有庶民同时而生者,又何贵贱贫當之不同也V’”末文作者虽然自谦“予不晓命术,姑记之,以俟深于五行&折衷焉”,可是却又认为“此说似有理”。可见这里,作者的天平是倾向于不信命的。

《鸡肋编南北宋之际的著述,作者庄绰在书中卷上说道,世上以五行星历论命者多矣,这里抄录先贵而后凶的命几则:“张邦昌,元丰四年辛酉七月十六日亥时。王黻,元丰二年己未十一月初二日卯时。燕瑛,熙宁十年丁巳五月二十六日寅时。聂山,元丰元年戊午八月初十日卯时。赵野,元丰七年甲子正月十九日丑时。朱勐,熙宁八年乙卯十月二十六日申时。王衆,元丰元年戊午正月初六日子时。蔡攸,熙宁十年丁巳三月三十日寅时。邓绍密,熙宁六年癸丑九月二十三日戌时。童贯,皇祐六年三月初五卯时。”对于这些人的命造,作者又说,当他们处在全盛期时,算命的都没能够说出他们未来的灾祸,由此可见,阴阳家的话是不可太听信的,只有端正身心,好好做人,才是立身处世的唯一办法。

清代大诗人王士祺,对于算命这种玩意,他在《池北偶谈》卷二十一中引陆象的话批判说:“五行书以人始生年月日时所值辰,推贵贱夭寿祸福甚详,乃独略于智愚贤不肖,曰纯粹清明,则归之窗贵福寿,曰驳杂浊晦,则归之贱贫夭祸。《易》有否泰,君子小人之道,迭相消长,各有盛衰。纯驳清浊明晦之辨,不在盛衰,而在君子小人。今顾略于智愚贤不肖,而必归之富贵贫贱寿夭祸福,何耶7”这种从君子小人,智愚贤不肖角度入手,对算命术只推人贵贱贫宫夭寿祸福所做的批判,却也别具一格。

清代名士袁枚,也是不信算命的一位。在《随园随笔5中,他说,当初大挠作¥子,原來不过为了记数而已,就好比数一二三四一样,并没有什么多大意义,更谈不上什么五行生克配合了。听到袁枚的这种说法,当时曾有人反对道:“人在社会上本来也没姓名,可是一旦取名以后,人家一叫他就应了。天干地支这玩愆既然古人早就给它”上了阴阳五行,不也和人应答一样道理吗?”对于这种暗中偷换概念的狡辩,袁枚也用同样的狡辩术驳斥道:“人是天地万物之灵,所以一叫就应,如果派给草木禽兽什么姓名,就叫不应了,又何况天干地支这种本来就是子虚乌有的东西呢?”
清代道光年间的文人笔记中,吴炽昌的《客窗闲话》,以其流畅的文笔,多彩的内容而为读者所熟悉。书中“续集”卷七有《禄命》一则道:近来有个姓赵的算命先生,精于子平之术,自己推算下来应该得个四品的官,可是因为读书不多,难求功名。后来赵姓来到京师,看到做官的都下级承奉上级,以谄誉获利,心里感到很不是味儿,于是出都回到扬州推牌算命。平时他住在楼上,前来算命的先要用钱挂号登记次序,然后再用箩筐把人家的八字从楼下吊到楼上,除1大富贵人,一般人很难和他见而,所以名噪一时。一次,本郡太守派仆人去赵姓那里算命,赵姓漪到他的八字和自己
算
一模一样,心甩很是诧异,就用纸条放下楼去洵问來人说:“如果生在南方,和我的命差不多,如果生在北方,就有四品的官职。”后来仆人回答:“我家老爷是北方旗籍。”采然被他算中。.		:-v
可是在实际中,八字完全一样而命运不一样的到处都有。那时浙江巡抚的儿子和镇江一个卖豆腐人家的儿子出生在同年同月同日同时,后来巡抚的儿子因为荫袭得官,当老子逝世之后,这做儿子的也当上了浙江巡抚的官,可是那卖豆腐人家昨儿子,却仍接替他的先人,做着卖豆腐的勾当。又如《消K录》载纪晓岚学士的侄子,和家里奴仆的儿子刘云鹏一起降生人间,其侄十六岁而夭,刘云鹏却依然健在。当时出生,只隔着一扇窗子,两个孩子同时产出,连分秒都一样,可就是一尊一卑,一夭一寿,这又怎么解释呢?可见唐朝太常博士吕才驳论命理,千古不移。
末了,吴炽昌总结:“天下之大,每日万生万死。帝皇夭寿之日,岂无同者?昔明太祖密谕各布政,确搜与同八字之人(和明太祖朱元璋八字一样的人)。乃进三人:一僧、一丐、一市侩。帝以问刘青田,亦无以对。故曰命之理微,圣人罕见之。”这里吴炽昌的结论虽有保守成分,可是对于命理所持的怀疑态度,则是显而易见的。
对于不信命或对命理持怀疑态度的学者,除了以上几位,当然还有很多的人。可见在信命和不信命之间,从古以来就有着针锋相对的斗争。
算
\section{算命术中海亩蜃楼的象征律和风雨飘摇的演绎法}
中国算命术尽管博大梢深,万化千变,搞得神乎其神,好象果真泄了什么天地造化之秘似的,可是剖开来看,它也有一个最基本、最原则的运算法,自始至终地贯穿在整个算命术的中间,那就是象征律和在象征律基础上推进演化出来的演绎法。
尽管阴阳五行在我国古代哲学体系中,有着它朴索的、唯物的一面,但是它也不可避免地存在着种种不足,尤其是五行:用木、火、土、金、水五种物质包罗自然界的万物,难免粗糙生硬;五行相生用金生水,只是•就金属在高温下的液化状态来说的,金属的液化状态又怎能和水划上等号呢?再如土为万物之母,能生木、生金、生火、生水,却把五种原先本属各自独立并列的元索,说成了母子关系。其实,举其一点,土就是土,水就是水,土里所含的水,本是水的一种存在形式,又怎么能认为是水生出来的呢?
当然,我们不能用现代的科学水平来要求古人,否则社会就没有前进了。不过,这至少可以提供给我们这样的思考,就是建立在这种纯看生辰八字五行理论基础上的算命术,完全弃后天人为和社会因素于不顾,它的科学性又到底有多强?它的揭示人生命历程的预测又到底有多可靠?
算
现在暂时先不.管这些,言归象征律本身。
C人身一小天地〕把人象征天地宇宙或自然界,是古人的一种普遍认识。天地自然运周不休,人也运周不休;天地自然有阴阳五行,人也有阴阳五行。人既有阴阳五行,那末,推究每个人出生年月日时干支所乘受的阴阳五行之气的不同,从而推测他们一生的人生历程,就被引进到算命术中来了。
〔人与四时合序〕这是由人身一小天地生发出来的一种象征律。阴阳家把十天干和十二地支分为阴阳五行,按照日与天会的原理而记年,月与日会的原理而记月。这样一年有四季十£个月,笄命的就把一个人降生时碰上的天干地支,分为年、月、日、时四柱,从而推定他一生的吉凶。此外,结合出生季节看五行的旺相休囚死,用的也是一种人与四时合序的象征方法。
〔五行寄生十二宫〕这是一种完全模拟自然界生物在一年十二个月中生长衰绝的象征律。在五行寄生十二宫的理论中,分别有长生、沐浴、冠带、临官、帝旺、衰、病、死、墓、绝、胎、养等状态。这些状态,循环无端,周而复始,而秉有五行之气的人,在生趣上同样也有着这种相似。其中首先是绝,绝又叫受气或胞,好比万物处在地里,还没有成形,又象母亲腹空,没有怀孕•,二是受胎,这时天地气交,氤氳造物,物在地下萌芽,好比人受父母之气;三是养而成形,万物在地里成形,就象孩子在母腹成形一样;四是长生,万物发生向荣,象人初成形而生长;五是沐浴,沐浴又叫作败,因为万物始生,形体柔脆,容易损伤,似嬰儿刚出母腹三天,给他
洗浴,容易困绝;六是冠带,这时万物渐渐秀荣,有如人开始穿起了衣冠;七是临官,万物既已渐趋秀实,就象人临官似的;八是帝旺,天地万物至此成熟,人亦至此精力健旺;九是衰,万物由成熟开始转向形衰,好比人由盛壮转向衰老;十.是病,万物有病,如同人有病一样;十〜是死,万物死亡,和人的死亡没有什么两样;十二是墓,墓又叫库,万物成功藏进仓库,就象人死进入坟墓。归墓以后,万物又受气胞胎而生,就这样周而复始,直到无穷。
〔五行秉性和相貌性情〕人既秉受天地五行之气而生,那末按照命理学家的说法,不同五行乘性的人,也就自然有着不同的相貌性情了。命主以木为主的人瘦长清朴,因为木形修长,木质清朴;命主以火为主的面赤聪明,因为火色红赤,火性闪烁;命主以土为主的人面黄忠淳,因为土属黄色,土性敦厚;命主以金为主的人面白刚毅,因为金属色白,性质坚硬;命主以水为主的人面黑机灵,因为水色沉黑,水性流动。
〔象征社会伦理纲常的用神〕命理学家论命看重用神,而用神名称的由来,则多半象征着封建社会的伦理纲常。《三命通会》探索古人立印、食、官、财名义时说:“生我者有父母之义,故立名印绶。印,荫也;绶,授也。替父母有恩德,荫庇子孙,子孙得受其福,朝廷设官分职,畀以印绶,使之掌管。官而无印,何所凭据?人无父母,何所怙恃?其理通一无二,故曰印绶。我生者有子孙之义,故立名食神。食者如虫吃物,盖伤之也。虫得食物则饱,人得食物则益。被食则损,造化以子成而致养,即人子致养父母之道也,故曰食神。克我者,我受制于人之义,故立名官、煞。官者梢也,煞者害也。朝廷以官与人,此身属于公家,任其驱使,赴汤蹈火,不敢有违,至于盖棺而已,是官害之也。凡人梦棺则得官,亦是此义)故曰官、煞。我克者逛人受制于我之义,故立名妻财。如人娶妻,而妻有妆奁田土,赍以事我,终身无违,我得自然享用,不致诏乏,况人成家立产,须得妻室内助,故曰妻财。是四者,术家立名之大义。”奄无疑问,这些算命家立名的大义,是深深打上了封建社会伦理纲常烙印的。

以上所举五种,只是就算命术象征律中一些主要方面而说的c其实,算命术中的象征律还远远不止这些。说实在的,几乎在所有的命理观念中,都是或多或少地浸透着这种象征律的。而这种象征律,在某些局部还有点类似于机械类比推理的味迫。虽然这种类比推理的办法比较粗糙。
在这种天地阴阳五行象征基础上建立起来的算命术,在具体的推算过程中,采用设广的趟一种演绎的推理方法。在这种演绎推理中,命理学家先把一个人的出生年月日时的干支五行和他的大运推排出来,然后又把这作为推理的前提,一步一步地演绎推算下去,如命中五行自身属金,并且金多水多土多火少,就可演算出这人采有金的诚性,生性刚正不阿,由于金水相生,汩汩流通,又应聪明过人,技术超群,.在运行中,不喜比劫的金、食伤的水和印绶的土,因为这在八字中已经够多的了,倒是正财偏财的木和官星的火,可以为我所用,所以一行到木弋运,就必当发迹无疑。
在实际过程中,算命先生对一个人八字的浈算要远比
1:面所说的复杂得多。比如一个命的日亍是金,在推演时不仅要考虑到整个八字五行对日主金的影啤关系,并且还要综合考虑到出生的月份和寄生十二宫对自身的利弊,八字与八字之间的刑冲化合,以及干支彼此交通往来所形成的种种吉神和凶煞等等。所以这种独特的演绎法,虽然有些类似于我们今天逻辑学上的演绎推理法,然而又不能完全等同起来。因为演绎推理要求,推理的前提必须是正确的,否则就推不出正确的结论来。然而对于命理学家来说,他们所信奉或假设的前提有多大的可靠性,那就得打上几个大大的问号了。

通过说理分析,我们已不难想见,算命术象征律所象征的,不过是通过“人身一小天地”、“天人感应”等观念而作出的对自然现象的一些表面象征。然后,又通过出生年、月、日、时所得五行和这些表面象征挂起钩来进行千变万化,无穷无尽的演绎推算。这种推算,既不顾个人对自己前程的努力如何,又不顾整个社会发展对人类所造成的种种重大影晌,单从海市蜃楼的徒有堂皇表面的象征出发,那末建立在这基础上的演绎法,岂非是不攻自破?算准是偶然的,算不准是必然的,现在我们到了为笄命术下这种绪论的对候了。

\section{学术乎?迷信乎?}
中国算命术比起世界上任何算命术来,因为有着一个表面看去似乎十分完整严密的学术体系,所以远比其他国家的种种算命术复杂得多,难学得多。也就因为这个原因,所以一千多年来,非但一直盛行民间不衰,并且还傅得了一些学者大儒的普遍青睐。撇开算命术发明以前已深信天命•的孔子、列子等名满天下的巨子不说,单就算命术发明以后,南宋的大儒朱蔌、明代的闻人刘基、清代的学者俞曲园等,都是信命而且自己又会算命的一些代表人物。

由于这些学者巨儒的介入,侦猝命术这种玩意,更为社会上一些学问不髙的平民百姓所深信不疑。因为这些学者大儒的崇高声望,确实影响了一大枇人    学者大倘的介入,主要是因为算命术所依据的,是我国天人感应和阴阳五行的哲学理论。在这种貌似科学的理论支配下,东汉大学者王充尽管可以不信鬼神,但却坚信命运。他认为:“人麽气而生,含气而长,得贺则炎,得贱则贱。”“或贵或贱,或贫或贫,宫或累金,贫或乞食,贵至封侯,贱至奴仆,非天琪施有左右也,人物受性有厚薄也,并JTf言:“富贵贫贱皆在初拔之时,不在长大之后随操行而至也。”为什么他要坚持提出这种“人受命在父母施气之时,以(已)得吉凶矣”的说法呢?他说得很清楚,就是“天施气而众星布精,天所施气,众星之气在其中矣”。原来自然界中充满了一种天气和众星的梢气,人在结胎之初受了这种气的或厚或薄的影响,就会影响到他今后的一生。邊种宇宙之间的气,自然也包涵了金、木、水、火、土布施出来的五行之气。

差不多和王充同时,《白虎通•五行篇》中还把自古以來的阴阳五行思想,和社会人事作了种种紧密的联系。书中煞有介琳地说:“父母生子养长子,何法?法水生木,木长大也9”a男不离父母,何法?法不离木也。女离父母,何法?
.法水流去金也,“不娶同姓,何法?法五行离类乃相生也。”“子丧父母,何法?法木不见水则憔悴也。”“父死子继,何法?法木终火旺也。”“臣谏君,何法?法金正木也。子谏父,何法?法火揉直木也。”真还被说得头头是道。

后来,唐朝的李虚中和五代的徐子平接过东汉学者论命论五行的学说,在论命中大加发挥,并从而形成了一套完整的学术体系。
从上所述,由于算命术采用了阴阳五行哲学和天文星象中的一些现象作为立命的理论基础,加上通人达士学者大儒的加入和肯定,就使得神秘的中国算命术在无意中披上了一层学术的堂皇外衣。说实在的,如果撇开符•命目的,单就这种弈命术本身葙眼,它确实有着一整套完憋的体系,若要深入探究下去,也够你一辈子研究的,可是算命术的发明,毕竟趋有着它的目的的,这目的就是探究每个人一生未来的吉凶荣枯、寿夭贫從等等,属于一种多少年来人们一直向往着的预知术,
预知术在我国有史以来,一直是一门人们前赴后继,悉心探索着的学问,它的内容除了四柱算命外,还广泛地包括着占卜、星相、拆字、起课详梦、扶乩等术,这里面尽管掺杂着种种江湖骗子和浓重的迷信色彩,可从另一方面来说,毕竟也注进了一些通儒学者历时绵久的研究探索,因为这种预知术对于整个人类来说,确实是太神秘,f太具吸引力了,尽管到头来,这种探索努力只能是失败的。
算
那末说到这里,算命术是不是就不迷信了呢?答案是,就算命术本身的一整套完整体系来说,里面多少蕴涵着一种学术思想,但就预知术的目的而言,由于算命术作为演算蕋本的大前提,绝口不谈个人因素和社会因素,却空谈什么采自先天的五行生克之类的哲理,其本身的合理程度由此可见。加上结合天文星象,又混进了好多宇宙中并不存在的凶神恶煞等等,所以非但推断不准,并且还在它长期的存在过程中,不可避免地给江湖骗子以可乘之机,从而使算命术更加笼上了一层浓m迷信的神秘外衣。
在目前,社会上迷信算命术的仍然不乏其人,这说明它的存在,有猗极其a杂的社会因素和历史根源。要彻底根除人们对筲命术的迷信,单纯靠国家禁令和行政的堵,是无济于事的。看来,最好的办法还是静下心来,对它进行无情的解剖,把它的原原本本暴鍩在光天化日之下,让人们自己来作一番科学而又深入的评判,这样,不仅这种预知术究竟灵验不灵验可以尽人皆知,并且还使那些江湖骗子失去了混饭吃的资本,岂不美哉?
李虚中发明算命术之初,原是根据一个人出生年、月、日所碰上的干支进行推算的,由于推算的结果,相同的人实在太多,于是五代的徐子平才又加上时间的干支,从而奠定了年、月、R、时四柱八字推命的基础。可是这样下来,据说也只有五十一万几千种命,因此社会上八宇相同的人还是不少。并且在一些相同的八字中,有的还命运截然不同,判若泾消。按照宋人有关资料记载,蔡京的八字是丁亥、壬资、壬辰、辛亥,按理在八宇的格中,这是属于壬骑龙背的格局,该娃大宵大贵的,可是偏偏在京城里郑粉儿子的八字,也和蔡亨一横•一样,但在人生的道路上,这小子却潦倒不烟,没能混好。这就给我们的命理学家出了个不小的难题。尽管前面我们说过,算命先生在碰上问题棘手时,自会举出种种遁辞,自跖其说,然而这种局面的出现,毕竟是很严峻的。

对于这种难以解释的现象,就是命书本身,也不得不作彻底的承认。《三命通会'》始明末以来最为权威的一木命书,内中卷六曾载《十干十二年生大资人例》一篇,说是只要在六帀年丁卯月乙未R戊寅时,六乙年己卯月甲戌日乙亥时,六丙年庞寅月丁巳內午时,六丁年丙午月壬辰日丁未时,六戊年壬戌月己丑日戊寅时,六己年辛未月己未日丙寅时,六庚竽甲申月庾申日辛巳时,六辛年丙申月庚午日辛巳时,六壬年辛亥月壬辰日丁未时,六癸年丙M月丙辰日戊子时,这六十个时辰出生的人,必定是建功立业大贵的人,不然至少也得是个出尘的神仙。以上这六十个时辰出生的人分配到六十花甲中去,每年只有一日一时,才有大贵人应世。可是对于这种说法,《三命通会》作者育吾山人无可奈何地感叹道:“大贺人莫过帝王。考历代创业之君,及明朝诸帝,无一合者。余尝谓天下之大,兆民之众,如此年、月、日、时生者,岂无其人,然未必皆大贵人。要之天生大贵人,必有冥数气运以主之,年、月、日、时多不足凭。”
好一句“必有冥数气运以主之,年、月、日、时多不足凭”,其中“冥数气运以主之”7是虚晃一枪的遁辞,只有“年、月、日、时多不足凭”一句,才是书主人多年来为人卷命的甘苦之言。讲实上,作者在一生命理研究的生涯中,符到的缙
算
绅人家和凡夫俗子同命的,多得数也数不过来。就是在缙绅和缙绅之中,八字相同而命运不同的也大有人在。因此作者接着说道:“如黄懋官侍郎,与申价副使同命,黄死于兵祸,申死牖下。申先黄死,官之大小,又不论也。朱衡与李庭龙同命,朱发科壬辰,李发科癸丑(两人没同一年登軏)。朱官至尚书,李止大参,寿又不永。其子孙之多寡贤否,又不论也纟万窠与饶才同命,万举进士,官至卿贰,饶止举人,官至太守。然饶多子而万则少,又万以谪戍死,而饶则否,其寿夭得丧又难论也。三河黄且斋兄弟同产,而功名先后,亦自不同。”为之,作者不由感叹:“况天下之大,九州之广,兆民之众,其八字同者何限,又乌以例论耶?”
罗大经的《鹤林玉藤》,堪称宋人笔记中的佼佼者。,书中记有“大算数”一则道,一天有人拜访黄直卿,说是善算星数,能够预知吉凶祸福。对此,黄则卿回答道:“我也有个大算数,《书》说:‘湛迪吉,从逆凶。作善,降之百祥,作不善,降之百殃。大学》也说:f言悖而出者,亦悖而入。货悖而入者,亦悖而出。’这个数,从古到今没有差错,难道不比你的算数强吗?”这ffl,黄直卿引《尚书》和《大学》的话,大意是说一个人做善事就吉,做恶事就凶,做善事,上天就会賜福给你,做坏事,上天就会降灾给你。说话背理伤人的,也会被人家所伤。用不正当手段弄进货物的,也会被人家用不正当的手段弄去。

这里,黄直卿把《尚书》、《大学》的这段话当作为人处世的大算数,从而风趣生动地批判了客人的星数,可谓笔力扛的确,社会上立身处世,最要紧的还是“大算数”,因为这是自己给自己算命的S佳方案。种瓜得瓜,种豆得豆,佛家的因果报应论在这里和中国的传统逍德,在某种程度上该说是一线吻合的。平时佛家反对猝命,这也主要是他们信奉“众善奉行,诸恶奠作”的信条,好事fl我做,至于上天如何安排处置,不是我应该过问的事。

事实上,偏信命运安排,忽视“大算数”而栽跟斗的也大有人在。据说明清之际有个染坊儿子,八宇算下来是个大富大贵,高官厚禄的命。家里人听说孩子生了这么个高的命,都大喜过望,从小开始就什么都听他的。后来孩子长大酗酒游荡,不务正业,结果酒醉落水而死,死时才十九岁。这难道不是偏信算命,从小失于教宵所招致的祸患?再如从前文推子息的歌诀来?h也是很荒谬的。歌中谈到最多的是五子,而封建社会里达官货人广莕姬妾,生子在五个以上的比比皆是。又如目前社会上实行计划生育,提倡一对夫妻只生一个孩子,而歌里却三个四个五个迭出,又作如何解释?这不使命理学家犯愁了么?如果硬说现在人工流产也应包括在内,可到底又有点硬解的味道了。
因此结论是:对于中国文化中的算命术,我们先要了解它,剖析它,因为了解、剖析的全过程,就是批判的全过程。堵不如导,这&一条早已被证宍了的历史规律。


