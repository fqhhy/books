% 对齐环境:
%\begin{center} … \end{center}
%\begin{flushleft} … \end{flushleft}
%\begin{flushright} … \end{flushright}
% 对齐命令
% \raggedright \centering  \raggedleft
% 想使用这三个命令,只对齐一段文本,不对后续文本生效,可以如下
% {\raggedleft second line to be right aligned.\par} 在结尾使用 \par

% 环境会在上下文产生一个额外间距,而 \centering 等命令不产生。
% 一行的文本对齐方式  环境 \leftline{左对齐} \centerline{居中} \rightline{右对齐}

\documentclass{ctexart}

% 计算文字宽度(\widthof命令)
\usepackage{calc}
\usepackage{multirow}
% 使用\foreach命令
\usepackage{tikz}

% 记录字符串长度的变量


\newcommand\college{学院}
\newcommand\majorID{专业班级}
\newcommand\thesisauthor{学生姓名}
\newcommand\adviser{指导教师}
\newcommand\coadviser{协助指导教师}
\newcommand\applydate{完成日期}
\newlength{\lablen}

\newcommand\test[5]{{
  % \setlength{\lablen}{\maxof{\widthof{#1}}{\lablen}}
  % \setlength{\lablen}{\maxof{\widthof{#2}}{\lablen}}
  % \setlength{\lablen}{\maxof{\widthof{#3}}{\lablen}}
  % \setlength{\lablen}{\maxof{\widthof{#4}}{\lablen}}
  % \setlength{\lablen}{\maxof{\widthof{#5}}{\lablen}}
  % \mbox{\the\lablen}

  % \makebox[#1em][s]{测试}\\
  % \makebox[#1em][s]{#3}\\
  \makebox[#1em][s]{#3\\#4}\hspace{1em}\mbox{\raise#2\baselineskip\hbox{毛泽东}}\\
}}

\begin{document} %在document环境中撰写文档

\test{13}{1}{中华人民共和国}{中国共产党中央委员会主席}{}{毛泽东}
\test{13}{1}{测试测试}{中国共产党中央委员会副主席}{}{周恩来}

\mbox{\raise2.25pt\hbox{正文}} \Large{测试}



% 找到最长的字符串(用\settowidth命令实现)
% \settowidth{\lablen}{\coadviser} % 不需要calc宏包
% 用\makebox生成分散对齐的定宽盒子
% \begin{tabular} {cc}%
%   \makebox[\the\lablen][s]{\college} :     & \makebox[\the\lablen][s]{} \\ \cline{2-2}%
%   \makebox[\the\lablen][s]{\majorID} :     & \makebox[\the\lablen][s]{} \\ \cline{2-2}
%   \makebox[\the\lablen][s]{\thesisauthor} :& \makebox[\the\lablen][s]{} \\ \cline{2-2}
%   \makebox[\the\lablen][s]{\adviser} :     & \makebox[\the\lablen][s]{} \\ \cline{2-2}
%   \makebox[\the\lablen][s]{\coadviser} :   & \makebox[\the\lablen][s]{} \\ \cline{2-2}
%   \makebox[\the\lablen][s]{\applydate} :   & \makebox[\the\lablen][s]{} \\ \cline{2-2}
% \end{tabular}

% \vspace{4ex}

% 找到最长的字符串(用\widthof命令实现)
% \setlength{\lablen}{\widthof{\coadviser}} % 需要calc宏包
% 用\makebox生成分散对齐的定宽盒子
% \begin{tabular} {cc}%
%   \makebox[\the\lablen][s]{\college} :     & \makebox[\the\lablen][s]{} \\ \cline{2-2}%
%   \makebox[\the\lablen][s]{\majorID} :     & \makebox[\the\lablen][s]{} \\ \cline{2-2}
%   \makebox[\the\lablen][s]{\thesisauthor} :& \makebox[\the\lablen][s]{} \\ \cline{2-2}
%   \makebox[\the\lablen][s]{\adviser} :     & \makebox[\the\lablen][s]{} \\ \cline{2-2}
%   \makebox[\the\lablen][s]{\coadviser} :   & \makebox[\the\lablen][s]{} \\ \cline{2-2}
%   \makebox[\the\lablen][s]{\applydate} :   & \makebox[\the\lablen][s]{} \\ \cline{2-2}
% \end{tabular}

% \vspace{4ex}

% 找到最长的字符串(用循环实现,需要TikZ宏包)
% \foreach \x in {\college, \majorID, \thesisauthor, \adviser, \coadviser, \applydate}
% {
%   \settowidth{\textlen}{\x}
%   \ifdim \textlen > \lablen
%     \setlength{\lablen}{\textlen}
%   \else
%     \relax
%   \fi
% }
% 用\makebox生成分散对齐的定宽盒子
% \begin{tabular} {cc}%
%   \makebox[\the\lablen][s]{\college} :     & \makebox[\the\lablen][s]{} \\ \cline{2-2}%
%   \makebox[\the\lablen][s]{\majorID} :     & \makebox[\the\lablen][s]{} \\ \cline{2-2}
%   \makebox[\the\lablen][s]{\thesisauthor} :& \makebox[\the\lablen][s]{} \\ \cline{2-2}
%   \makebox[\the\lablen][s]{\adviser} :     & \makebox[\the\lablen][s]{} \\ \cline{2-2}
%   \makebox[\the\lablen][s]{\coadviser} :   & \makebox[\the\lablen][s]{} \\ \cline{2-2}
%   \makebox[\the\lablen][s]{\applydate} :   & \makebox[\the\lablen][s]{} \\ \cline{2-2}
% \end{tabular}
  

\end{document}
